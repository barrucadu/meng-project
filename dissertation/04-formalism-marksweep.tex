\chapter{Correctness of Mark-Sweep Collectors}
\label{sec:marksweep}

This chapter talks about the necessary formalisms and proof
obligations for mark-sweep collectors. Firstly, we define the basic
language necessary for talking about garbage collection in Section
\ref{sec:marksweep-defns}. The problem of correctness is decomposed
into marking and sweeping in Sections \ref{sec:marksweep-marking} and
\ref{sec:marksweep-sweeping}, and finally we build up to a proof of
the Armstrong/Virding\cite{Armstrong95} collector in Section
\ref{sec:marksweep-example}.

\section{The Language of Garbage Collection}
\label{sec:marksweep-defns}

\paragraph{Preliminary Definitions}
The following terms are used throughout the dissertation: $h$, a heap;
$h'$ the state of a heap $h$ before garbage collection; $r$, the root
set; $a \pointsto b$, the cell $a$ contains a pointer to the cell $b$;
$\free{c}$, true iff the cell $c$ is on the free list; $\reach{h}{r}$,
the set of cells reachable from $r$ in $h$; and $\type{w}$, the type
of a word. Furthermore, quantifiers are used as follows: $\forall c
\in h$ ranges over pointers to cells in the heap $h$; and $\forall w
\in c$ ranges over addresses within a cell $c$. The definitions of
$\free{c}$ and $\reach{h}{r}$ are expressed more formally later on.

\begin{definition}[Heap]
  \label{def:ms-heap}
  The heap is an array of words in the form (type, data).
\end{definition}

Firstly, we need a heap. The convention that the heap is a flat array
has been used, and so pointers are just indices into this array. This
allows the use of Gries' array formalism\cite{Gries87}. Furthermore,
type data is needed for each word, as there must be some way of
distinguishing pointer data from non-pointer data. It may seem that
needing to tag every single word with a type is wasteful, but in the
case where the collector is able to determine this automatically, the
type can be stored, as it were, in a ghost variable.

\begin{definition}[Cell]
  \label{def:ms-cell}
  A cell is a contiguous sequence of one or more words in the heap,
  starting with a sequence of one or more metadata words used by the
  garbage collector, known as the header of the cell.
\end{definition}

Next we move on to cells, at least an entire word has been reserved
for the use of the garbage collector, but this is quite generous, and
in practice would be reduced as much as possible. Reserving space
simplifies proof, and can be worked around in practice by transforming
a more efficient garbage collector into a less efficient one, and then
performing the proof: if the transformation is correct, the proof will
hold for the original, albeit with slightly different space
characteristics.

\begin{definition}[Pointer]
  \label{def:ms-pointer}
  A pointer is an index into the heap array, pointing to the start of
  the data section of a cell.
\end{definition}

Pointers have been restricted to only pointing at the start of the
data portion of a cell,\footnote{It would seem, at first glance, that
  this restriction in conjunction with the definition of words in the
  heap does not allow null pointers, and this would be the case if we
  interpreted the definition of the heap to mean ``all words have a
  static type'', but if we allow the type of a word to be determined
  at runtime, then this is not the case. The type rule for pointers
  could, in practice, be something like ``resides in the nth word of a
  cell and is the address of the start of the data portion of another
  cell'', in which case invalid pointers (such as null pointers) would
  be considered to be data.} as then going from an internal pointer to
a cell header does not need to be considered at all, furthermore, in
practice the mutator does not need to know about the metadata, and
should not know: it should be for purely internal use. If the size and
layout of the header is known in advance, any word in it can be
accessed by pointer arithmetic. Now we'll define a helper function
which will be of use later on: given a pointer, $p$, let $\gchead{p}$
return the heap index for the start of the garbage collection
metadata.

\begin{definition}[Word Preservation]
  \label{def:ms-word-preservation}
  After garbage collection, no allocated cells have been mutated.
   \[\forall c \in h,\ \forall w \in c,\ \lnot\free{c} \implies h[w] =
   h'[w]\]
\end{definition}

A short word needs to be said on what it means for a cell to be
allocated. Conceptually, cells are either allocated, in which case
they may or may not be garbage, or deallocated, in which case the
allocator can hand them out to the mutator. However, this is a
temporal property, and difficult to define as a predicate. Thus,
rather than talk about a cell being allocated or not, we shall talk
about it being on the free list or not: this is what the predicate
$\free{c}$ expresses.

Word preservation is a necessary property of an entire mark-sweep
collector, but may be broken by the marker or the sweeper, as long as
it is restored at the end. As such, it has not included it in the
definitions of correctness in the following two sections, but it is
still essential.

Finally, in order to ensure that pointers only point to sane
locations, we must show that the live cells and the free cells are
disjoint,

\begin{definition}[Live Cell Invariant]
  \label{def:ms-live-cell-invariant}
  No live cells contain a pointer to a cell on the free list.
  \[\forall c \in \reach{h}{r},\ \lnot\free{c}\]
\end{definition}

\section{Marking}
\label{sec:marksweep-marking}

This section shall attack the problem of correct marking, firstly as a
problem in graph theory, and then as a more concrete problem when
dealing with an array of memory with pointers.

Let's start out by saying exactly what goes in that garbage collection
metadata. The first word is a mark flag, equal to 0 if the cell has
not been marked, and any other value if it has.

\begin{definition}[Reachable]
  \label{def:ms-reachable}
  A cell is reachable if it is pointed to by a root, or pointed to by
  a cell which is reachable.

  \begin{minipage}{.5\textwidth}
    \begin{prooftree}
      \AxiomC{$r \in roots$}
      \AxiomC{$r \pointsto a$}
      \BinaryInfC{$a \in \reach{h}{roots}$}
    \end{prooftree}
  \end{minipage}
  \begin{minipage}{.5\textwidth}
    \begin{prooftree}
      \AxiomC{$a \in \reach{h}{roots}$}
      \AxiomC{$a \pointsto b$}
      \BinaryInfC{$b \in \reach{h}{roots}$}
    \end{prooftree}
  \end{minipage}
\end{definition}

If we abandon our array-based interpretation of memory for the moment,
and instead consider the heap to be a digraph of cells connected by
pointers, then reachabilty can be defined inductively over the nodes
of the graph very simply, and this lends itself to implementation with
a recursive function. If we denote the set of all marked, and
allocated, cells by $\markset{h}{roots}$, then what we want after
marking is $\markset{h}{roots} = \reach{h}{roots}$.

Returning to the array formalism, what we thus want from a marker is
the following property:

\begin{definition}[Correct Marking]
  \label{def:ms-correct-marking}
  After marking, for all allocated cells, the mark flag is unset if
  and only if the cell is not reachable.

  \[\forall c \in h,\ \lnot\free{c} \implies
  \left(h\left[\gchead{c}\right] = 0 \iff c \notin
    \reach{h}{roots}\right)\]
\end{definition}

Note that this does not say anything about unallocated cells. An
unallocated cell can contain any garbage, but then it is the job of
the allocator to clean it up before handing it out to the mutator. An
alternative formulation would be $\lnot\free{c} \iff
\left(\ldots\right)$, but this would then require the garbage
collector to unset the mark flag of any cell it frees, and as cells
need to be modified before being given to the mutator anyway,
confining this to the allocator seems reasonable.

As we'll be referring to the property of a cell being marked multiple
times, define a shorthand, $\marked{h}{c} \equiv
h\left[\gchead{c}\right] \neq 0$.

\section{Sweeping}
\label{sec:marksweep-sweeping}

This section shall consider sweeping, with an assumption that correct
marking has already been performed. Then, we shall briefly consider
interleaved marking and sweeping, such as happens in some incremental
and other special-case collectors.

What is meant when it is said that a sweeper is correct? After a
little thought, we will realise that we mean it doesn't deallocate any
cells which were marked, and it unmarks all of the cells. In the case
of a correct marker, this is equivalent to preserving and unmarking
exactly the reachable cells, which (with the additional constraint of
word preservation) is exactly what we want our collector to do.

\begin{definition}[Correct Sweeping]
  \label{def:ms-correct-sweeping}
  All cells which are marked are preserved and unmarked, no garbage
  cells are preserved, and nothing other than the garbage collector
  metadata gets mutated.
  \[\forall c \in h',\ \left(\lnot\free{c} \iff c \in
    \markset{h'}{roots}\right) \land \left(\lnot\free{c} \implies
    \lnot\marked{h}{c}\right)\]
\end{definition}

The essential component here is the $\iff$, the effect of which is to
identify the set of allocated cells with the set of cells which were
marked.

\subsection{Interleaved Marking and Sweeping}
\label{sec:marksweep-sweeping-interleaved}

Hitherto it has been assumed that marking happens in its entirety, and
then sweeping happens in its entirety. However, sometimes marking and
sweeping are interleaved, such that in each iteration of the collector
a little more of each is done.

In order for correctness to hold here, we can conceptually separate it
out into a marker and a sweeper, but rather than talking about the
entire heap, we talk about the subheap considered so far. Thus, the
correctness of each remains the same, but we have an additional proof
obligation.

\begin{definition}[Interleaved Marking/Sweeping Correctness]
  \label{def:ms-interleaved}
  When the sweeper considers a cell, the marker has already marked it
  and everything it points to (if it was reachable).
\end{definition}

Thus, the marker always remains ahead of the sweeper, and so no
problems arise. In fact, we could consider normal mark-sweep
collectors a special case of this type, where this property trivially
holds.

\section{Case Study: A Garbage Collector for Erlang}
\label{sec:marksweep-example}

Armstrong and Virding\cite{Armstrong95} introduced a simple mark-sweep
collector for Erlang, making use of the immutability of the language
in order to combine the mark and sweep stages. The collector operates
on a heap of cons cells, and uses a pointer named \texttt{SCAV} to
keep track of the current position of the collector in the heap. The
algorithm is presented in Figure \ref{fig:marksweep-example-algo}.

\begin{figure}[t]
  \centering
  \begin{lstlisting}
last = current
SCAV = hist(last)
while (SCAV != first) {
    if (marked(SCAV)) {
        possibly_mark(car(SCAV));
        possibly_mark(cdr(SCAV));
        unmark(SCAV);
        last = SCAV;
        SCAV = hist(last);
    } else {
        tmp = SCAV;
        SCAV = hist(SCAV);
        set_history(last, SCAV);
        free_cons(tmp);
    }
}
  \end{lstlisting}
  \captionsetup{format=default}
  \caption{The Armstrong/Virding garbage collector.}
  \label{fig:marksweep-example-algo}
\end{figure}

The \texttt{possibly\_mark}\footnote{Notational aside: variable and
  function names lifted directly from the algorithms are typeset in
  \texttt{typewriter font}, even in proofs, to distinguish them.}
function follows and marks its argument if it is a pointer;
\texttt{first} and \texttt{current} point to the first allocated cell
and the head of the free list, respectively; and the \texttt{hist}
function returns a pointer to the cell allocated before its
argument. It is assumed that all cells pointed to by roots have been
marked before calling the collector.

This collector works because the history fields form a linked list of
allocated cells, going back to the beginning of time. As the language
is immutable, pointers must always point back in time, and so if a
cell is unmarked by the time it is reached by following the history
list, then it must be garbage. We can express this property as a loop
invariant: the portion of the heap which has been considered consists
only of unmodified, reachable cells, and all cells which are directly
reachable from a cell in the considered region are marked.

\subsubsection{Partial Correctness}
\label{sec:marksweep-example-partial}

Let $\id~x$ be the \textit{allocation ID} of a cell $x$. The first
cell has an ID of zero, and the ID of every other cell is one greater
than the ID of the previously allocated cell: we can then express the
loop invariant as in Figure
\ref{fig:marksweep-example-partial-invariant}.

\begin{figure}[t]
  \centering
  \begin{align*}
    \forall c \in h',\ \forall w \in c,\ &first \in \reach{h'}{roots}\\
%
    &\land \id~c > \id~\mathtt{SCAV} \implies (\free{c}
      \iff c \notin \reach{h'}{roots})\\
%
    &\land \id~c > \id~\mathtt{SCAV} \land \lnot\free{c}
      \implies h[w] = h'[w] \land \lnot \marked{h}{c}\\
%
    &\land \id~c > \id~\mathtt{SCAV} \land \lnot\free{c}
      \implies (\forall c \pointsto x,\ \id~x < \id~\mathtt{SCAV}
      \implies \marked{h}{x})\\
%
    &\land \left(\id~c < \id~\mathtt{SCAV} \land \nexists x \pointsto
      c,\ \id~x > \id~\mathtt{SCAV}\right) \implies \lnot\marked{h}{c}
  \end{align*}  
  \caption{Loop invariant for the Armstrong/Virding collector}
  \label{fig:marksweep-example-partial-invariant}
\end{figure}

As the collector never touches the \texttt{first} cell, in order for
there to be no garbage left over at the end, then it must necessarily
be reachable. This is somewhat unusual, and arises purely because this
collector arguably has a bug, in that this one cell of garbage can
survive collection.

Let's begin by rewriting the algorithm in terms of arrays, and expand
out all of the functions within it. Furthermore, let's say that a cell
in this scheme consists of two words of metadata and two of data,
arranged in the order $\langle mark,\ history,\ car,\ cdr
\rangle$. The rewritten algorithm is displayed in Figure
\ref{fig:marksweep-example-partial-arrays}.

\begin{figure}[t]
  \centering
  \begin{lstlisting}
last = current
SCAV = h[gchead(last) + 1]
while (SCAV != first) {
    if (marked(h, SCAV)) {
        car = h[SCAV];
        cdr = h[SCAV + 1];

        if(type(car) == pointer) {
            h[gchead(car)] = 1
        } else {
            skip
        };

        if(type(cdr) == pointer) {
            h[gchead(cdr)] = 1
        } else {
            skip
        };

        h[gchead(SCAV)] = 0;

        last = SCAV;
        SCAV = h[gchead(last) + 1];
    } else {
        tmp = SCAV;
        SCAV = h[gchead(SCAV) + 1];
        h[gchead(last) + 1] = SCAV;
        free_cons(tmp);
    }
}
  \end{lstlisting}
  \caption{Array-based version of the Armstrong/Virding collector}
  \label{fig:marksweep-example-partial-arrays}
\end{figure}

Combining the correct marking, correct sweeping, and word preservation
requirements, we arrive at the following postcondition for a correct
garbage collector: \[\forall c \in h',\ \forall w \in c,\
(\free{c} \iff c \notin \reach{h'}{roots}) \land (\lnot\free{c} \implies
h[w] = h'[w] \land \lnot\marked{h}{c})\]

\begin{lemma}[Loop Invariant]
  \label{lem:li}
  The loop invariant holds.
\end{lemma}

\begin{proof}
  \begin{figure}[t]
    \centering
    \begin{align*}
      \forall c \in h',\ \forall w \in c,\ &\mathtt{SCAV} = first \land
        first \in \reach{h'}{roots}\\
%
      &\land \id~c > \id~\mathtt{SCAV} \implies (\free{c}
        \iff c \notin \reach{h'}{roots})\\
%
      &\land \id~c > \id~\mathtt{SCAV} \land \lnot\free{c}
        \implies h[w] = h'[w] \land \lnot \marked{h}{c}\\
%
      &\land \id~c > \id~\mathtt{SCAV} \land \lnot\free{c}
        \implies (\forall c \pointsto x,\ \id~x < \id~\mathtt{SCAV}
        \implies \marked{h}{x})\\
%
      &\land \left(\id~c < \id~\mathtt{SCAV} \land \nexists x \pointsto
        c,\ \id~x > \id~\mathtt{SCAV}\right) \implies
        \lnot\marked{h}{c}\\\\
%
%
      \forall c \in h',\ \forall w \in c,\ & \id~c > \id~first \implies
        (\free{c} \iff c \notin \reach{h'}{roots})\\
%
      &\land \id~c > \id~first \land \lnot\free{c}
        \implies h[w] = h'[w] \land \lnot \marked{h}{c}\\
%
      &\land \id~c > \id~first \land \lnot\free{c}
        \implies (\forall c \pointsto x,\ \id~x < \id~first
        \implies \marked{h}{x})\\
%
      &\land \left(\id~c < \id~first \land \nexists x \pointsto
        c,\ \id~x > \id~first\right) \implies
        \lnot\marked{h}{c}\\\\
%
%
      \forall c \in h',\ \forall w \in c,\ & \free{c} \iff c \notin
        \reach{h'}{roots}\\
%
      &\land \lnot\free{c} \implies h[w] = h'[w] \land \lnot \marked{h}{c}\\
%
      &\land \lnot\free{c} \implies (\forall c \pointsto x,\
        \lfalse \implies \marked{h}{x})\\
%
      &\land \left(\lfalse \land \nexists x \pointsto
        c,\ \id~x > \id~first\right) \implies
        \lnot\marked{h}{c}\\\\
%
%
      \forall c \in h',\ \forall w \in c,\ & (\free{c} \iff c \notin
        \reach{h'}{roots}) \land (\lnot\free{c} \implies h[w] = h'[w] \land
        \lnot \marked{h}{c})
    \end{align*}
    \caption{Sufficiency of the loop invariant for the Armstrong/Virding collector}
    \label{fig:marksweep-example-partial-sufficient}
  \end{figure}

  The sufficiency of the loop invariant is shown in Figure
  \ref{fig:marksweep-example-partial-sufficient}, and now it remains
  to be shown that the invariant holds,
  \begin{prooftree}
    \AxiomC{$\htriple{I \land C \land M}{\ldots}{I}$}
    \AxiomC{$\htriple{I \land C \land \lnot M}{\ldots}{I}$}
    \BinaryInfC{$\htriple{I \land C}{if(marked(h, SCAV)) \ldots}{I}$}
    \UnaryInfC{$\htriple{I}{while \ldots}{I \land \lnot C}$}
  \end{prooftree}
  where $C = \mathtt{SCAV} \neq first$ and $M =
  \marked{h}{\mathtt{SCAV}}$.

  The proof naturally separates into two cases, one for each branch of
  the if statement. These are proven separately in Lemmas
  \ref{lem:lia} and \ref{lem:lib}.
\end{proof}

\begin{lemma}[Loop Invariant (first branch)]
  \label{lem:lia}
  The loop invariant holds in the marked case of the loop.
\end{lemma}

\begin{proof}
  Knowing that nothing in this branch deallocates cells, the proof
  goals can immediately be simplified to these three conditions:
  \begin{enumerate}
    \item $\mathtt{SCAV} \in \reach{h'}{roots}$
    \item $\forall w \in \mathtt{SCAV},\ h[w] = h'[w] \land
      \lnot\marked{h}{\mathtt{SCAV}}$
    \item $\forall \mathtt{SCAV} \pointsto x,\ \marked{h}{x}$
  \end{enumerate}

  Firstly, let's look at the two expansions of
  \texttt{possibly\_mark}, as they are almost identical. Let's refer
  to the variable holding the possible pointer as $x$, and let $X
  \equiv \type{x} = \typointer$.
  \begin{prooftree}
    \AxiomC{$\htriple{X \land P}{h[gchead(x)] = 1}{Q}$}

    \AxiomC{$\lnot X \land P \implies Q$}
    \UnaryInfC{$\htriple{\lnot X \land
        P}{skip}{Q}$}
    \BinaryInfC{$\htriple{P}{if(type(x) == pointer)
        \{\ldots\} else \{skip\}}{Q}$}
  \end{prooftree}

  The desired postcondition, of course, is \[(\type{x} =
  \typointer \land \id~x < \id~SCAV) \implies \marked{h}{x}\]

  Fortunately, we know that if $x$ is a pointer, then the allocation
  ID of its cell must be lower than that of the current cell because
  of the constraint over the entire system: data is immutable. Thus,
  only backward pointers are possible, and so we can simplify the
  postcondition to just \[\type{x} = \typointer \implies
  \marked{h}{x}\]

  If $x$ is a pointer, then the cell it points to gets marked. If the
  gaps in the proof tree are filled in as follows, then this can be
  achieved:
  \begin{align*}
    P &\equiv \type{x} = \typointer \implies \marked{(h;
      \gchead{x}:1)}{x}\\
    Q &\equiv \type{x} = \typointer \implies \marked{h}{x}
  \end{align*}

  Now it must just be shown that the assignment works:
  \begin{align*}
    (X \land P)[h/(h; \gchead{x}:1)] &\iff (\type{x} =
      \typointer\\
    &\quad\land (\type{x} = \typointer \implies
      \marked{(h;\gchead{x}:1)}{x}))\\
    &\quad[h/(h; \gchead{x}:1)]\\
%
    &\iff \type{x} = \typointer \land (\type{x} =
    \typointer \implies \marked{h}{x})\\
%
    &\implies Q
  \end{align*}

  It can now be seen that by simply combining lines 5 to 18, we end up
  in the state, \[(\type{car} = \typointer \implies
  \marked{h}{car}) \land (\type{cdr} = \typointer \implies
  \marked{h}{cdr})\] which is exactly what is wanted. Part (3) of the
  invariant has now been handled. This result can now be inductively
  applied, combined with the precondition of this entire block,
  $\marked{h}{\mathtt{SCAV}}$ to trivially see that part (1)
  holds. Finally, it can be seen that we never assign to anything
  other than the mark flags, and the mark flag of \texttt{SCAV} is
  unset on line 20, and so part (2) holds.
\end{proof}

\begin{lemma}[Loop Invariant (second branch)]
  \label{lem:lib}
  The loop invariant holds in the unmarked case of the loop.
\end{lemma}

\begin{proof}
  Informally, as we have assumed correct marking, we know that this
  cell is not reachable from the roots. The invariant is broken on
  line 26, as an unreachable cell is left in the list ahead of
  \texttt{SCAV}, but then it is restored in lines 27 and 28, as the
  cell is removed and deallocated. Thus, the invariant still holds.

  Knowing that the loop invariant holds and is sufficient for
  correctness, as established by Lemma \ref{lem:li}, it only remains
  to be shown that it holds at the start of the loop.

  \begin{prooftree}
    \AxiomC{$\htriple{I''}{last = current}{I'}$}
    \AxiomC{$\htriple{I'}{\texttt{SCAV} = h[gchead(last) + 1]}{I}$}
    \BinaryInfC{$\htriple{I''}{last = current; \texttt{SCAV} = h[gchead(last) + 1]}{I}$}
  \end{prooftree}

  Furthermore, as $h$ refers to the current heap and $h'$ to the
  original heap (being a ghost variable), and as the heap has not been
  mutated yet, $h = h'$, which allows further simplifications to be
  made.

  \begin{align*}
    I' &= I[h[\gchead{last} + 1] / \mathtt{SCAV}]\\
    &=\forall c \in h,\ \forall w \in c,\\
    &\quad\quad first \in \reach{h}{roots}\\
    &\quad\quad \land \id~c > \id~h[\gchead{last} + 1] \implies
    (\free{c} \iff c \notin \reach{h}{roots})\\
    &\quad\quad \land \id~c > \id~h[\gchead{last} + 1] \land
    \lnot\free{c} \implies h[w] = h[w] \land \lnot \marked{h}{c}\\
    &\quad\quad \land \id~c > \id~h[\gchead{last} + 1] \land
    \lnot\free{c} \implies (\forall c \pointsto x,\ \id~x <
    \id~h[\gchead{last} + 1]\\&\quad\quad\quad\quad\implies \marked{h}{x})\\
    &\quad\quad \land \left(\id~c < \id~h[\gchead{last} + 1] \land
    \nexists x \pointsto c,\ \id~x > \id~h[\gchead{last} + 1]\right)
    \implies \lnot\marked{h}{c}\\\\
%
    I'' &= I'[current / last]\\
    &=\forall c \in h,\ \forall w \in c,\\
    &\quad\quad first \in \reach{h}{roots}\\
    &\quad\quad \land \id~c > \id~h[\gchead{current} + 1] \implies
    (\free{c} \iff c \notin \reach{h}{roots})\\
    &\quad\quad \land \id~c > \id~h[\gchead{current} + 1] \land
    \lnot\free{c} \implies h[w] = h[w] \land \lnot \marked{h}{c}\\
    &\quad\quad \land \id~c > \id~h[\gchead{current} + 1] \land
    \lnot\free{c} \implies (\forall c \pointsto x,\ \id~x <
    \id~h[\gchead{current} + 1]\\&\quad\quad\quad\quad\implies \marked{h}{x})\\
    &\quad\quad \land \left(\id~c < \id~h[\gchead{current} + 1] \land
    \nexists x \pointsto c,\ \id~x > \id~h[\gchead{current} + 1]\right)
    \\&\quad\quad\quad\quad\implies \lnot\marked{h}{c}\\
%
    &=\forall c \in h,\ \forall w \in c,\\
    &\quad\quad first \in \reach{h}{roots}\\
    &\quad\quad \land \lfalse \implies (\free{c} \iff c \notin
    \reach{h}{roots})\\
    &\quad\quad \land \lfalse \land \lnot\free{c} \implies
    \ltrue \land \lnot \marked{h}{c}\\
    &\quad\quad \land \lfalse \land \lnot\free{c} \implies
    (\forall c \pointsto x,\ \id~x < \id~h[\gchead{current} + 1]
    \implies \marked{h}{x})\\
    &\quad\quad \land \left(\ltrue \land \nexists x \pointsto
    c,\ \lfalse\right) \implies \lnot\marked{h}{c}\\
%
    &= \forall c \in h,\ first \in \reach{h}{r} \land (\lnot\free{c}
    \implies \lnot \marked{h}{c})
  \end{align*}

  Thus, we can see that the algorithm is correct if, when it starts,
  no cells are marked. As no cells are marked after it completes, this
  becomes a constraint on the allocator, to unmark recycled cells
  before giving them to the mutator. It can also be seen from the
  precondition that the collector has an overhead of one cell: the
  \texttt{first} cell is never collected, and so correctness does not
  strictly hold if \texttt{first} is not reachable, but the collector
  is close enough such that it doesn't really matter.
\end{proof}

\subsubsection{Total Correctness}
\label{sec:marksweep-example-total}

We can use allocation IDs to express that the history list forms an
unbroken chain of cells, all the way back to the first cell, as
follows:

\[\forall x,\ \left(x = \mathrm{first} \implies x =
  \mathrm{hist}(x)\right) \land \left(x \neq \mathrm{first} \implies
  \exists n \in \mathbb N_{1},\ \id~x = n +
  \id~\mathrm{hist}(x)\right)\]

The loop condition can be restated as $\id~\mathtt{SCAV} \neq 0$. As
there are no IDs below 0 ($\mathrm{hist}(\mathrm{first}) =
\mathrm{first}$), that can be further rewritten to $\id~\mathtt{SCAV}
> 0$. \texttt{SCAV} starts out as the last-allocated cell, and so has
an ID $\geq 0$. Now, in order to prove termination, we simply need to
show that $\id~\mathtt{SCAV}$ decreases at every iteration of the
loop.

To simplify, let us throw away everything which does not relate to
mutating \texttt{SCAV}, giving the following loop:

\begin{lstlisting}
while (SCAV != first) {
    if (marked(SCAV)) {
        last = SCAV;
        SCAV = hist(last);
    } else {
        SCAV = hist(SCAV);
    }
}
\end{lstlisting}

Substituting variables, we get

\begin{lstlisting}
while (SCAV != first) {
    if (marked(SCAV)) {
        SCAV = hist(SCAV);
    } else {
        SCAV = hist(SCAV);
    }
}
\end{lstlisting}

And then we can eliminate the if statement:

\begin{lstlisting}
while (SCAV != first) {
    SCAV = hist(SCAV);
}
\end{lstlisting}

Trivially, we can now see that there are two cases:

\begin{description}
  \item[Case 1, $\mathtt{SCAV} = \mathrm{first}$] In this case,
    $\mathrm{hist}(\mathtt{SCAV}) = \mathtt{SCAV}$, and so
    $\id~\mathtt{SCAV}$ does not decrease, but this situation is
    precisely the loop termination condition.

  \item[Case 2, $\mathtt{SCAV} \neq \mathrm{first}$] In this case, we
    know by the history list assumption that
    $\id~\mathrm{hist}(\mathtt{SCAV}) < \id~\mathtt{SCAV}$, and so
    $\id~\mathtt{SCAV}$ does decrease. Furthermore, we know that the
    ID cannot decrease below zero, and so we get a strictly decreasing
    sequence of positive natural numbers. Clearly, this sequence is
    not infinite, and so the loop terminates.
\end{description}

\section{Summary}
\label{sec:marksweep-summary}

This chapter has defined the heap, cells, pointers, word preservation,
and reachability (defns.  \ref{def:ms-heap}, \ref{def:ms-cell},
\ref{def:ms-pointer}, \ref{def:ms-word-preservation}, and
\ref{def:ms-reachable}), and used these to then define correct marking
and sweeping (defns.  \ref{def:ms-correct-marking} and
\ref{def:ms-correct-sweeping}), giving some thought to the correctness
of interleaved marking and sweeping (defn. \ref{def:ms-interleaved}).

It has been argued that correctness for a mark-sweep collector comes
down to marking all (and only) of the reachable cells, freeing all
others, and then unmarking. And, furthermore, that no words should be
modified in non-garbage cells after garbage collection.

Then the Armstrong/Virding\cite{Armstrong95} collector was used as an
example of the developed formalism, with proofs of its partial and
total correctness provided. These proofs shall not be referenced in
future chapters, and merely serve as an illustration of the utility
of the definition of correctness presented here. Definitions, however,
will be reused.
