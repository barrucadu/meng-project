\chapter{Conclusion}
\label{sec:conclusion}

In this chapter I shall discuss further work suggested to me by
undertaking this project, and close with some final thoughts on the
matter.

\section{Further Work}
\label{sec:conclusion-further}

\subsection{Machine Verification}
\label{sec:conclusion-further-machineverification}

Whilst it would have been desirable to produce machine-verified
proofs, I did not do so for two reasons: firstly, learning to use a
proof assistant would have been a significant amount of work to add to
the project, and secondly I believe the proofs are good enough in
their current form. Parts of the proofs may be amenable to automation,
however, which would have simplified the work whilst simultaneously
making it more rigorous.

Given the two proofs in the current forms, however, it would not be
infeasible to translate them into a form suitable for machine
verification, and to ``fill in the blanks'', resulting in completely
machine-verified proofs. This endeavour would also highlight any
assumptions I have unknowingly made in this project, and bring to
light any further obligations, particularly those on the mutator or
allocator.

\subsection{Run-time Invariant Checking}
\label{sec:conclusion-further-invariants}

The garbage collector is only a part of a larger system, consisting of
the allocator, mutator, and a possible intermediary memory-management
subsystem. Proving the garbage collector correct is all well and good,
but if the other components break the invariants or preconditions upon
which the garbage collector relies, it may fail, no matter how
rigorous the proof of correctness.

Unfortunately, the task of verifying a nontrivial system is very
difficult, and this is the reason why formal verification is so rare
in practice. Thus, it may be worthwhile to implement run-time checking
of invariants into the system, and to either halt or emit an error
when such an invariant is broken.

If this could be implemented efficiently, it would only be a small
overhead for increased reliability, as garbage collection bugs can not
arise in a verified collector where all of the expectations
hold. Furthermore, if this could be implemented such that invariants
are checked after every operation which could break them, it would
produce a powerful debugging tool, as a programmer would know exactly
where errors were introduced into the system.

\section{Final Thoughts}
\label{sec:conclusion-thoughts}

\todo{How I feel, briefly, about the whole thing. Try to avoid
  references to mounting insanity.}