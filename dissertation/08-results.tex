\chapter{Results \& Evaluation}
\label{sec:results}

Previously, I have attempted to provide proofs of correctness for two
extant garbage collection algorithms, and from this extract a more
generalised notion of correctness. In this chapter, I shall discuss
and evaluate the results of the two facets of my work, and how
successful I have been in achieving the goal of this project. I shall
also discuss an experimental validation of my proofs involving an
implementation of the two collectors discussed.

\section{Proofs of Collectors}
\label{sec:results-collectors}

In the process of producing a generalised notion of correctness for
garbage collectors, I first provided proofs of the Armstrong/Virding
and Fenichel/Yochelson collectors: two very different approaches, but
both for systems based on fixed-size cons cells. The proofs for
neither collector I considered could be machine verified in their
current form, however I believe that they remain convincing despite
this.

\section{Collector Implementation}
\label{sec:results-impl}

As evidence of the correctness of the proofs, I
implemented\footnote{All of the code produced can be found in Appendix
  \ref{sec:gc-impl}, and online at
  \url{https://github.com/barrucadu/meng-project}.} both of the
collectors in C, and wrote a small test program with a very
constrained heap which constructs a variety of linked lists, and
results in over 200 garbage collections occurring. I then attached
assertions to points in the programs to verify the preconditions,
postconditions, and invariants of the collectors.

The implementations are written in C99, and have been tested with both
clang 3.4 and gcc 4.8.2. They should work under any
standards-compliant compiler, but I have not verified this.

The implementation does have some limitations imposed upon it for
simplicity: there is a fixed-size array of roots, which the programmer
must ensure contains all roots, and the requirement of immutable cells
in the Armstrong/Virding collector is not implemented. Resolving both
of these would have been significantly more work, and are not
necessary in order to demonstrate that the garbage collection works.

Both collectors manage a heap of \texttt{NUM\_CELLS} cells (in the
case of Fenichel/Yochelson, both semispaces are this big), in order to
have identical characteristics. This then allows identical assertions
to be used in the test program. As neither collector comes with an
allocator, I have had to produce allocators, \texttt{alloc()}, which
work with them. The way I have done so is not necessarily the only, or
best, way, it is just what occurred to me and appears to work. In both
cases, the collector is invoked by the allocator upon not having
enough memory left for an allocation. The preconditions are checked
before calling the \texttt{gc()} function, and the postconditions
checked after. In each of the garbage collector loops, the invariants
are checked after each iteration.

Due to using a low level language, the implementations of the
collectors look somewhat more complex than what is specified in their
respective papers, but is as close as I could reasonably obtain.

The assertions, and their implementation are discussed below.

\subsection{Armstrong/Virding}
\label{sec:results-impl-ms}

\subsubsection{Preconditions}
\label{sec:results-impl-ms-pre}

The Armstrong/Virding collector, as I discussed, has a precondition
that the $first$ cell is always reachable, as otherwise it would be
garbage, and due to it never being freed, garbage would survive a
collection. Furthermore, it is assumed that when the collector is
called, all cells pointed to by roots have been marked.

\begin{lstlisting}[language=C,caption={Armstrong/Virding Preconditions}]
assert(reachable(gchead(first)));

for(unsigned int i = 0; i < NUM_ROOTS; i++)
  if(roots[i] != NULL)
    gchead(roots[i])->mark = MARKED;
\end{lstlisting}

\subsubsection{Postconditions}
\label{sec:results-impl-ms-post}

The postcondition express simply the correct mark-sweep postcondition:
after collection, all cells which are allocated (not on the free list)
are reachable from the roots. Furthermore, the live cell invariant is
checked here as a postcondition, although arguably it should be
checked with the rest of the invariants, in the garbage collection
loop.

\begin{lstlisting}[language=C,caption={Armstrong/Virding Postconditions}]
for(unsigned int i = 0; i < NUM_CELLS; i++)
  if(reachable(&heap[i]))
    assert(!on_free_list(&heap[i]));

for(unsigned int i = 0; i < NUM_CELLS; i++)
  if(!on_free_list(&heap[i]))
    assert(reachable(&heap[i]));
\end{lstlisting}

\subsubsection{Invariants}
\label{sec:results-impl-ms-invariants}

The invariants appear somewhat more complex. The first loop expresses
$\forall c \in h,\ \alloc{c} \land \id~c > \id~\mathtt{SCAV} \implies
(\forall c \pointsto x, \id~x < \id~\mathtt{SCAV} \implies
\marked{h}{c})$: all cells outside of the region that has been
collected, but which are pointed to by a cell in that region, have
been marked. The latter two express the simpler properties that all
allocated cells in the collected region are live, and all allocated
cells in the collected region are unmarked.

\begin{lstlisting}[language=C,caption={Armstrong/Virding Invariants}]
for(unsigned int i = 0; i < NUM_CELLS; i++)
  if(reachable(&heap[i]) && cell_id(&heap[i]) > cell_id(gchead(SCAV)))
    {
      if(heap[i].cell.car.tag == REFERENCE &&
         heap[i].cell.car.val.ptr != NULL &&
         cell_id(gchead(heap[i].cell.car.val.ptr)) < cell_id(gchead(SCAV)))
        assert(gchead(heap[i].cell.car.val.ptr)->mark == MARKED);

      if(heap[i].cell.cdr.tag == REFERENCE &&
         heap[i].cell.cdr.val.ptr != NULL &&
         cell_id(gchead(heap[i].cell.cdr.val.ptr)) < cell_id(gchead(SCAV)))
        assert(gchead(heap[i].cell.cdr.val.ptr)->mark == MARKED);
    }

for(unsigned int i = 0; i < NUM_CELLS; i++)
  if(reachable(&heap[i]) && cell_id(&heap[i]) >= cell_id(gchead(SCAV)))
    assert(reachable(&heap[i]));

for(unsigned int i = 0; i < NUM_CELLS; i++)
  if(reachable(&heap[i]) && cell_id(&heap[i]) > cell_id(gchead(SCAV)))
    assert(heap[i].mark == UNMARKED);
\end{lstlisting}

\subsection{Fenichel/Yochelson}
\label{sec:results-impl-c}

\subsubsection{Preconditions}
\label{sec:results-impl-c-pre}

An invariant which I did not mention whilst discussing my proof, but
which the garbage collector, mutator, and allocator must obey, is that
there are no inter-semispace pointers. Due to a bug in my linked list
code (which has since been remedied), a garbage collection occuring
whilst appending a value to a list would result in this invariant
being broken, and resulted in my being stuck for some time before I
realised what the issue was. I then decided to enforce this as a
precondition of the collector.

\begin{lstlisting}[language=C,caption={Fenichel/Yochelson Preconditions}]
cell *sspace = (cons_space == FIRST) ? semispace1 : semispace2;

for(unsigned int i = 0; i < NUM_CELLS; i++)
  if(reachable(&sspace[i]))
    {
      if(sspace[i].car.tag == REFERENCE &&
         sspace[i].car.val.ptr != NULL)
        assert(sspace[i].car.val.ptr >= sspace &&
               sspace[i].car.val.ptr <= &sspace[NUM_CELLS]);

      if(sspace[i].cdr.tag == REFERENCE &&
         sspace[i].cdr.val.ptr != NULL)
        assert(sspace[i].cdr.val.ptr >= sspace &&
               sspace[i].cdr.val.ptr <= &sspace[NUM_CELLS]);
    }
\end{lstlisting}

\subsubsection{Postconditions}
\label{sec:results-impl-c-post}

As a postcondition I again checked the live cell invariant, and also
asserted that all allocated cells are reachable.

\begin{lstlisting}[language=C,caption={Fenichel/Yochelson Postconditions}]
cell* sspace = (cons_space == FIRST) ? semispace1 : semispace2;

for(unsigned int i = 0; i < NUM_CELLS; i++)
  if(reachable(&sspace[i]))
    assert(!on_free_list(&sspace[i]));

for(unsigned int i = 0; i < NUM_CELLS; i++)
  if(!on_free_list(&sspace[i]))
    assert(reachable(&sspace[i]));
\end{lstlisting}

\subsubsection{Invariants}
\label{sec:results-impl-c-invariants}

The invariant I decided to check is that, after collecting every root,
all pointers in cells reachable from that root have been
updated. Actually, I went for the weaker property that pointers
pointed into the correct semispace. Unfortunately, proving the
stronger property would require having access to the pre-state of the
heap, which would require copying it before initiating a garbage
collection.

\begin{lstlisting}[language=C,caption={Fenichel/Yochelson GC Loop}]
for(unsigned int i = 0; i < NUM_ROOTS; i ++)
  if(roots[i] != NULL)
    {
      roots[i] = collect(roots[i]);
      check_pointers_updated(roots[i], NULL);
    }
\end{lstlisting}

\begin{lstlisting}[language=C,caption={Fenichel/Yochelson Pointer Checking}]
static void check_pointers_updated(const cell* root, const cell* cur)
{
  if(root == cur)
    return;

  if(cur == NULL)
    cur = root;

  cell *sspace = (cons_space == FIRST) ? semispace1 : semispace2;

  if(cur->car.tag == REFERENCE &&
     cur->car.val.ptr != ALREADYCOPIED &&
     cur->car.val.ptr != NULL)
    {
      assert(cur->car.val.ptr >= sspace &&
             cur->car.val.ptr <= &sspace[NUM_CELLS]);
      check_pointers_updated(root, cur->car.val.ptr);
    }

  if(cur->cdr.tag == REFERENCE &&
     cur->cdr.val.ptr != NULL)
    {
      assert(cur->cdr.val.ptr >= sspace &&
             cur->cdr.val.ptr <= &sspace[NUM_CELLS]);
      check_pointers_updated(root, cur->cdr.val.ptr);
    }
}
\end{lstlisting}

\section{Generalised Garbage Collection Correctness}
\label{sec:results-correctness}

\todo{Evaluation 2}

\section{Summary}
\label{sec:results-summary}

\todo{Key strengths and weaknesses of work}