\chapter{Introduction}
\label{sec:intro}

As programs become more complicated, we have a greater need for
high-level languages which abstract away from much of the underlying
machine. A key part of such a language is a run-time memory management
system, and in particular the automatic disposal of unneeded allocated
memory: that is, garbage collection.

\section{Motivation}
\label{sec:intro-motivation}

Garbage collection is an important part of many modern programming
languages, however there is an implicit assumption that it never goes
wrong. It would not be feasible for the programmer to account for all
the ways in which the run-time support system may fail, and doing so
would remove the benefit of using such a language in the first
place. But these systems are not perfect: they are written and tested
by humans, with formal verification being very uncommon.

An arguably better approach is to design a garbage collector which is
then formally verified, in order to be able to definitively eliminate
all bugs from the collector, and so release the programmer from the
possible worry of how it may fail. For total certainty, any
implementation must be proven to be a faithful translation of the
algorithm to machine code, however merely having a verified algorithm
is a good stepping-stone towards this.

\section{Aims}
\label{sec:intro-goals}

Reviewing the literature reveals that much of the recent work in
garbage collector development has been focused on concurrency, and
obtaining real-time guarantees. The topic of verified garbage
collection has received comparatively little interest, and is perhaps
viewed as something which would be nice in an ideal world, but not as
important as making garbage collectors as fast as possible. It is
unfortunate that verification has received little attention, as
garbage collection bugs can be a very difficult problem to detect
through testing alone.

This dissertation explores the route of using generic and flexible
formalisms, not tied to any particular class of collector, and which
can be used for proving a particular collector correct, or extracting
a new collector from the formalism. This dissertation presents the
following contributions,

\begin{enumerate}
  \item A formalism for a notion of partial correctness for mark-sweep
    collectors, making as few assumptions as possible; and a proof for
    an established collector correct with respect to the formalism.

  \item A formalism for a notion of partial correctness for copying
    collectors, reusing as much as possible from the mark-sweep
    formalism, but diverging where necessary; and a proof for an
    established collector correct with respect to the formalism.

  \item From these two formalisms, an extraction of a common formalism
    applicable to both mark-sweep, copying, and some other types of
    collectors, and some guidance to aid proof.

  \item An implementation of the two proven collectors, with key
    assertions verified at runtime whilst inducing a large number of
    collection cycles.
\end{enumerate}

\subsection{Development Methodology}
\label{sec:intro-goals-development}

The formalisms shall be developed by building everything up from first
principles, by firstly defining all of the necessities to talk about
garbage collection. Then what it means for mark-sweep and copying
collectors to be correct shall be considered, and appropriate
invariants and postconditions produced to ensure correctness.

The two approaches shall then be unified, possibly branching out to
include other properties, in to the general formalism, and it shall be
shown how it could be specialised to the previously proven collectors.

\subsection{Evaluation Methodology}
\label{sec:intro-goals-evaluation}

The specific formalisms shall be evaluated by considering how
convincing the proofs are, and what issues, if any, are raised by the
implementations with checked assertions. The implementations shall be
evaluated by discussing which properties were checked, and which
others could possibly have been. To a lesser extend, the evaluation
will also consider how easy the proofs were: measures such as length
of proof may not be totally accurate when comparing them, but can
provide a useful, albeit subjective, intuition.

The generic formalism shall be evaluated by considering how different
it is to the specific formalisms, and how easy it was to produce given
them. Furthermore, I shall consider how obvious an extension it is of
the specific formalisms, such as whether it is just a unification of
the two, or whether it allows anything totally different to what the
two specific formalisms permit.

\section{Current Work}
\label{sec:intro-current}

The current work in proving garbage collectors correct tends to either
be a proof for one specific collector, or a new collector extracted
from a formalism, with little work on providing a generic way to prove
collectors correct. An example of the former type is Birkedal
\textit{et al}'s\cite{Birkedal04} proof for Cheney's collector using
separation logic, and an example of the latter is
Myreen's\cite{Myreen10} construction of a copying collector using
HOL4: both of these shall be discussed further in the literature
review.
\section{Outline}
\label{sec:intro-outline}

Chapter \ref{sec:lit} summarises the historical context and recent
developments in the field of the project.

An understanding of Gries' array reasoning\cite{Gries87} is required
to follow the proofs, and an introduction to it is presented in
Chapter \ref{sec:heap}, along with a justification of using that
particular model of the heap.

Chapters \ref{sec:marksweep} to \ref{sec:gc} proceed to formalise
partial correctness for garbage collection, by considering
independently mark-sweep and copying collectors, and then abstracting
from the specifics of both to produce a general formalism.

Finally, Chapters \ref{sec:results} and \ref{sec:conclusion} summarise
the contributions of the project and what remains to be done, as well
as discussing limitations of the work.
