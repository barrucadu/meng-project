\documentclass[10pt,a4paper,twoside,openright]{report}

% Input
\usepackage[utf8x]{inputenc}

% Nicer fonts
\usepackage[T1]{fontenc}
\usepackage{lmodern}

% Graphics support
\usepackage{graphicx}
\usepackage{epstopdf}
\DeclareGraphicsRule{.eps}{pdf}{.pdf}{`epstopdf #1}

% Hyperlinks, URLs etc.
\usepackage{hyperref}
\usepackage{url}
\hypersetup{
    colorlinks=true,
    citecolor=black,
    urlcolor=black,
    linkcolor=black,
    pagecolor=black,
    anchorcolor=black
}

% Stuff at the end
\usepackage[toc,page]{appendix}
\usepackage[xindy]{glossaries}
\usepackage[xindy]{imakeidx}
\loadglsentries{glossary.glo}
\renewcommand*{\glossaryentrynumbers}[1]{}
\makeglossaries
\makeindex

% Advanced maths features
\usepackage{amsmath}
\usepackage{amssymb}
\usepackage{amsfonts}
% Conditional macros
\usepackage{xifthen}

% Definitions and theorems
\usepackage{amsthm}
\newtheorem{definition}{Definition}

% Drawings
\usepackage{tikz}
\usetikzlibrary{trees}

% Red underlined TODO messages!
\usepackage[normalem]{ulem}
\newcommand{\todo}[1]{{\color{red}{\uline{#1}}}}
% [citation needed]
\newcommand{\cn}{{\color{red}$^{\text{[citation needed]}}$}}

% Page layout jiggery pokery
\usepackage{pdflscape}
\usepackage{multicol}

\newenvironment{wide}{%
  \begin{list}{}{%
      \setlength{\topsep}{0pt}%
      \addtolength{\leftmargin}{-1.5cm}%
      \addtolength{\rightmargin}{-1.5cm}%
      \setlength{\listparindent}{\parindent}%
      \setlength{\itemindent}{\parindent}%
      \setlength{\parsep}{\parskip}}%
  \item[]%
}{%
  \end{list}%
}

\newenvironment{lscape}{\begin{landscape}\begin{wide}}{\end{wide}\end{landscape}}

% Logic
\usepackage{bussproofs}

\newcommand{\htriple}[3]{\left\{#1\right\}\ \mbox{#2}\ \left\{#3\right\}}
\newcommand{\len}{\mathrm{len}~}
\newcommand{\true}{\mathbf{true}}
\newcommand{\false}{\mathbf{false}}
\renewcommand{\iff}{\Leftrightarrow}
\renewcommand{\implies}{\Rightarrow}

\title{%
{\normalsize\scshape University of York}\\
\vspace{2em}
\hrule
\vspace{1em}
{\huge\bfseries Memory Management}\\
\vspace{1em}
{\bfseries Verified Garbage Collection}\\
\vspace{1em}
\hrule
\vspace{1em}
{\small Submitted for part completion of MEng Computer Systems and Software Engineering}
\vfill%
{\normalsize Number of words: TODO as counted by ``detex | wc -w'',\\
excluding code listings and appendices.}
}

\author{%
Author: Michael Walker\\
Supervisor: Colin Runciman
}

\date{}

\begin{document}

\pagestyle{empty}

\maketitle

\cleardoublepage
\begin{abstract}
\todo{Not a mini introduction. Write better abstract after having done
  more work}

Programming languages with managed runtimes abstract away from much,
sometimes even all, of the need for manual memory management. Whilst
a lot of work has gone into how to make these critical systems
performant, there has not been so much work on formal verification,
and developers often rely on traditional testing and bug
reporting. This is far from ideal, as source-level verification of
software is pointless if the runtime may have arbitrary bugs lurking
within.

In this project, I propose a formalism of partial correctness for
garbage collection, and then show how some common algorithms are
correct under this formalism. Finally, I propose a new garbage
collection algorithm based on a review of the literature, prove that
it is correct, and then measure the performance of a (non-verified)
Java implementation in Jikes RVM.  \vfill

\paragraph{Statement of Ethics}

This project concerns only the development and implementation of
algorithms. No human participation is involved at any point and as
such there are no specific ethical implications.
\end{abstract}


% Table of contents, roman numeral page numbers
\cleardoublepage
\pagestyle{plain}
\pagenumbering{roman}
\setcounter{page}{1}
\tableofcontents

%\listoffigures
%\listoftables
%\lstlistoflistings

% Reset page numbering for content
\newpage
\cleardoublepage
\pagenumbering{arabic}

\chapter{Introduction}

As systems become more complicated, we have a greater need for
higher-level languages which abstract away from much of the underlying
machine. A key part of such a language is a run-time \gls{memory
  management} system, and in particular the automatic disposal of
unneeded allocated memory: that is, \gls{garbage collection}.

\section{Motivation}

Garbage collection is an important part of many modern programming
languages, however there is an implicit assumption that it never goes
wrong. It would not be feasible for the programmer to account for all
the ways in which the run-time support system may fail, and doing so
would remove the benefit of using a managed language in the first
place. But these systems are not perfect: they are written and tested
by humans, with formal verification being very uncommon.

An arguably better approach is to design a \gls{garbage collector}
which is then formally verified, in order to be able to definitively
eliminate all bugs from the \gls{collector}, and so release the
programmer from the possible worry of how it may fail. For total
certainty, any implementation must be proven to be a faithful
translation of the algorithm to machine code, however merely having a
verified algorithm is a good stepping-stone towards this.

\section{Outline}

Chapter 2 summarises the historical context and recent developments in
the field of the project are presented, and is built upon in Chapter 3
where the full aims and motivations of the project are presented.

Chapter 4 proceeds to formalise partial correctness for
garbage collection, and justifies the formalism chosen by proving
correct two different collectors.

Chapter 5 develops a garbage collection algorithm, proves it
correct using the formalism, and then shows an implementation of the
algorithm in a language runtime. The relation between algorithm and
code is then discussed.

Finally, Chapters 6 and 7 summarise the contributions of the project
and what remains to be done, as well as discussing limitations of the
work.

\chapter{Literature Review}

\todo{This is all very general: perhaps make more specific as more
  details of the project get worked out}

This chapter reviews the background of the project: the traditional
methods of garbage collection, more complicated collectors, methods of
algorithm and software verification, and current work in the
production of verified garbage collectors.

\section{Garbage Collection}

Garbage collection is the process of automatically reclaiming unneeded
memory from a program. In languages with manual memory management,
such as C, a programmer can allocate new memory with \texttt{malloc},
but then must remember to release it with \texttt{free}. If this is
not done, a memory leak occurs, and the program may gradually consume
more and more memory.

In a garbage collected language, however, the programmer can allocate
new memory, but typically cannot (and does not need to) explicitly
deallocate it. Instead, some runtime system determines when blocks of
memory are no longer needed, typically by determining if they can be
reached by following pointers from the ``roots''---the set of
variables in scope---or not.

\subsection{The Traditional Algorithms}

We can divide most garbage collection algorithms into four camps:
reference counting, mark-sweep, copying, and mark-compact. Each is
suited to different use-cases, and the choice of which to use is often
determined by the behaviour of the particular system in which they
will be used.

\subsubsection{Reference Counting}

Reference counting is a very simple scheme where each block contains a
\textit{reference count}, some integer field which keeps track of the
number of references to the block. In the most basic form of the
algorithm, the compiler maintains a write barrier, which increments
and decrements the count immediately as references are created or
destroyed. Furthermore, as soon as the count reaches zero, the block
is deallocated\cite{Collins60}.

This strategy seems good, as it distributes the load of memory
management throughout the computation, however it results in the
problem of unbounded time taken to free a block, as freeing it may
cause other things to be freed, and so on\cite{GarbageCollection}.
This problem can be offset by pushing to a free list, rather than
doing the recursive free, or by deferring the reference counting.

The Deutsch--Bobrow algorithm is an example of this kind, where
reference count adjustments are stored in a transaction file, which is
routinely processed to adjust the state of the entire
system\cite{Deutsch76}.

A fatal flaw, alas, with reference counting is that it cannot reclaim
self-referential structures, even if they are
garbage\cite{McBeth63}. Some systems make use of ``weak references'',
references which do not cause the count to be modified, to offset
this. However, this places a burden on the programmer and so loses
some of the convenience of the system.

\subsubsection{Mark-Sweep}

The mark-sweep method was invented for the early Lisp
systems\cite{McCarthy60}, and it functions by giving every block a
\textit{mark bit}, which is initially unset. Upon running out of
memory, all cells reachable from the roots are marked (this can be
done recursively quite simply), and then all blocks in the heap are
examined: those which are unmarked are freed, and those which are
marked are unmarked, ready for the next collection.

Compared to reference counting, mark-sweep is a very different beast:
it handles cycles easily and there is no overhead on pointer
operations. It does, however, ``stop the world'' in order to perform a
collection\cite{GarbageCollection}. Furthermore, garbage collection
becomes more frequent as heap residency increases.

These problems can be offset somewhat by interleaving the sweeping
with the mutator (the user program, so called because it mutates the
heap). Hughes's lazy sweep algorithm\cite{Hughes82} does this by
performing a fixed amount of sweeping at each allocation, and so the
only long garbage collection pause happens when the heap needs to be
marked.

\subsubsection{Copying}

Copying collectors divide the heap into two \textit{semispaces}, where
allocation is only done in one of them at a time. In garbage
collection, all live block are copied to the other semispace, and the
roles of the semispaces are swapped\cite{Fenichel69}. Unlike
mark-sweep collectors, the pause time of a copying collector is
proportional only to the number of live blocks in the heap, rather
than the size of the entire heap.

Copying collectors have become fairly popular due to the low cost of
allocation and reduction of fragmentation, however it has a cost of a
half of the heap space\cite{GarbageCollection}.

\subsubsection{Mark-Compact}

Heap fragmentation can be a great problem: there may be more than
enough space to allocate something, but not enough contiguous space to
do so. Furthermore, high fragmentation can increase the rate of page
faults and cache misses, reducing the performance of the mutator.

Thus, we come to the mark-compact collectors. These first mark the
live portion of the heap, and then copy marked blocks over garbage
ones, moving everything towards one end of the heap in order to remove
fragmentation. There are three ways of performing this compacting:
blocks can be moved without regard for their original order; blocks
which point to each other can be placed next to each other; and blocks
can simply ``slide'' towards one end of the heap.

Unfortunately, these collectors require multiple traversals over the
heap, and so are potentially even slower than mark-sweep
collectors\cite{GarbageCollection}.

\subsection{Hybrid Collectors}

Sometimes any one of the traditional algorithms is not quite good
enough for a particular problem, and a better collector can be formed
by combining them. For example, reference counting could be combined
with periodic copying, in order to reclaim cyclic structures and
reduce fragmentation. For another example, Immix\cite{Blackburn08} is
a hybrid mark-region (memory is divided into large regions, inside
which allocation occurs) and copying collector.

\subsubsection{Generational Collectors}

Generational garbage collection is a type of hybrid collector: the
heap is divided into a number of \textit{generations}, where each
generation holds progressively older blocks. This is based on
empirical observations of block lifetimes resulting in the
generational hypothesis: young objects die quickly, whereas old
objects stick around\cite{Ungar84}.

Allocation happens in the youngest generation, and when that fills up
a \textit{minor collection} is started. The generation is garbage
collected, and old enough blocks get copied to the next generation. In
the simplest case, all live blocks at the time of a minor collection
get promoted. If the entire heap is full, a \textit{major collection}
is started, which uses some other garbage collection
algorithm\cite{GarbageCollection}.

Generational garbage collection tends to perform well for languages
where old-to-young pointers are rare, such as most functional
languages, and so has become fairly popular amongst implementers of
such languages.

\section{Algorithm Verification}

\todo{What is verification}

\subsection{Verification by Proof}

\todo{How to take some code and a specification and prove the two equivalent}

\subsection{Verification by Extraction}

\todo{How to refine a specification until we can pluck out code}

\subsection{Proof Assistants}

\todo{If I decide to use one, brief summary of what they are, and then
  a slightly more detailed subsubsection about the one I will use}

\section{Verified Garbage Collection}

Finally, we look at two examples of garbage collectors which have been
verified in different ways.

\subsection{Reusable verification of a copying collector}

Myreen\cite{Myreen10} has constructed a verified garbage collector
using the extraction method: correctness was formalised using the HOL4
theorem prover, was refined to produce an algorithm for correct
copying collectors, and was then refined further to the level of
verified machine code.

The collectors produced have been named L1, which is a specification
of garbage collection; L2, which is an implementation of L1 as the
transitive closure of a step relation; L3, which is a deterministic
implementation of L2 with more realistic memory semantics; L4, which
uses actual implementation types; and L5, which is verified machine
code. L5 is constructed by earlier work of Myreen, on constructing a
certifying compiler.

Other than the obvious, the main contribution of Myreen is that his
proof is quite generic, as only the lowest level of refinement (L5)
makes use of a specific programming logic, and could be applied to many
different copying collectors, just by changing the specific refinement
steps taken.

It is possible that his L1 collector could be applied to non-copying
collectors, providing an already formalised notion of correctness
``for free''.

\subsection{Local Reasoning about a Copying Garbage Collector}

Birkedal, Torp-Smith, and Reynolds\cite{Birkedal04} take a very
different approach to Myreen, they take an existing garbage collector
(Cheney's copying collector) and prove it correct, using a variant of
separation logic.

Similarly to Myreen, they define correctness in terms of a heap
isomorphism modulo garbage cells, and in particular, that the
resultant heap has no garbage cells whatsoever.

The motivation of this paper is to be a stepping stone towards
programming logics which incorporate garbage collection, and so being
able to prove the correctness of the combination of a program and a
language runtime, which has thus far not been feasible.

This paper contributes to the field an extension of separation logic
with semantics for more types of assertion over finite sets, which is
used to express the isomorphism property of the algorithm. Needless to
say, this is an essential property of any garbage collector which
moves data around in the heap.

The major difference between these two verified garbage collectors is
that Myreen used extraction whereas Birkedal \textit{et al.} used
proof. Thus, this proof cannot really be generalised to other
collectors (although the definition of correctness), whereas Myreen's
proofs are much more applicable to copying collectors in general.

\chapter{Problem Analysis}
\label{sec:analysis}

\section{Aims Revisited}
\label{sec:analysis-aims}

Reviewing the literature reveals that much of the recent work in
garbage collection development has been focused on concurrency, and
obtaining real-time guarantees. The topic of verified garbage
collection has received comparatively little interest, and is perhaps
viewed as something which would be nice in an ideal world, but not as
important as making garbage collectors as fast as possible. This is a
shame, as garbage collection bugs can be a very difficult problem to
detect through testing alone.

There have been a few verified garbage collectors, however these tend
to be proofs for specific extant algorithms, or extraction using very
strict definitions of garbage collection. The former typically cannot,
unfortunately, generalise to other collectors, and the latter makes
generalising to types of collectors other than what the author had in
mind very difficult.

This project explores the route of using generic and flexible
formalisms, not tied to any particular class of collector, and which
can be used for the proof or extraction methods. The aim, through
algorithm/formalism co-design, is to produce:

\begin{itemize}
  \item a generic formalism for garbage collection partial
    correctness, in terms of garbage collection as a function applied
    to the program state;

  \item an illustration of the utility of the formalism by proofs of
    correctness of two simple mark-sweep collectors: one specialised
    for immutable languages, and one more general;

  \item design, proof, and implementation of a copying collector.
\end{itemize}

\section{Development Methodology}
\label{sec:analysis-development}

\todo{How I'm going to develop the algorithm and prove it}

\section{Evaluation Methodology}
\label{sec:analysis-evaluation}

The formalism and algorithm shall be evaluated separately.

\subsection{Evaluating the Formalism}
\label{sec:analysis-evaluation-formalism}

The formalism shall be evaluated by considering how well it applies to
a variety of collectors. In this dissertation, only mark-sweep and
copying will be proven, but the proof obligations for other types of
collectors will be examined in order to determine if the formalism is
capable of satisfying them.

Furthermore, what the formalism doesn't state will be discussed, as in
order to make it widely applicable it must abstract over the details
which make each class of collector unique.

Finally, it shall be compared to other formalisms, and the advantages
and disadvantages of it over the others will be discussed.

\subsection{Evaluating the Algorithm}
\label{sec:analysis-evaluation-algorithm}

The algorithm shall be compared to current garbage collection
algorithms, in particular stop-the-world collectors, and the relative
performance and ease of implementation discussed.

\subsection{Qualiative Evaluation}
\label{sec:analysis-evaluation-qualitative}

Finally, how ``nice'' the formalism and algorithm are to use will be
considered. The aim of the project is primarily to produce a flexible
and generic formalism, but also to produce a formalism which can be
used to ease the difficulty of proving correctness. Even if the
formalism has everything that could be desired, if it is difficult to
work with, there is scope for improvement.

Similarly, the algorithm should be simplistic and primarily be used to
illustrate the use of the formalism, but it should also be a
realistically viable garbage collection algorithm. Whilst designing
and verifying a concurrent real-time collector is out of scope for
this project, it should still offer something new.

\chapter{Formalising Garbage Collection}

\section{What Partial Correctness Means}

To precisely formalise garbage collection, we must first informally
determine what the whole process is about. Myreen\cite{Myreen10}
starts by defining the heap as a finite partial mapping from addresses
to a list of pointers and some data. He then proceeds to formalise
garbage collection as a state isomorphism with no garbage cells left
over, however this is a rather strict definition.

\begin{definition}[Weak Garbage Collection (informal)]
  A garbage collector satisfies the weak garbage collection property
  if it is a state isomorphism, where the number of garbage cells in
  the resultant heap is strictly less than in the original heap.
\end{definition}

This may not appear to be terribly useful, but in some cases it is the
strongest statement we can make. For example, in the case of
generational garbage collection, the nepotism effect (tenured garbage
cells pointing to cells in the younger generation) may cause some
cells which are, strictly speaking, garbage to survive a minor
collection. These cells are then collected whenever there is a major
collection, but to say that a generational collects only ``collects''
when the entire heap is examined would be a very strange abuse of
terminology.

\begin{definition}[Strong Garbage Collection (informal)]
  A garbage collector satisfies the strong garbage collection property
  if it is a state isomorphism, where there are no garbage cells in
  the resultant heap.
\end{definition}

This is an English-language restatement of Myreen's formal notion of
garbage collection correctness, which is expressed more precisely in
terms of a filtering function.

\subsection{Logic}

\todo{Introduction of the logic being used in the project. Restatement
of the English formalisms in the logic.}

\section{Correctness of a Mark-Sweep Collector for Immutable
  Languages}

\todo{Proof of correctness of the Erlang collector}

\section{Correctness of a Mark-Sweep Collector for Mutable Languages}

\todo{Proof of correctness of the Runciman collector}
\chapter{Algorithm Design \& Implementation}

\section{The Basic Algorithm}

\todo{The algorithm in pseudocode, justified}

\section{Formalising the Algorithm}

\todo{The algorithm with appropriate logical annotations and refinements}

\subsection{Proof of Partial Correctness}

\todo{Proof of correctness}

\subsection{Proof of Total Correctness}

\todo{Proof of halting of a gc cycle}

\section{Implementation of the Algorithm}

\todo{Implementation of the algorithm in a language runtime: choice of
language, runtime system, justifications.}

\subsection{What Makes the Implementation Correct}

\todo{Argument that the implementation is a correct translation of the
algorithm, pointers to how this could be formally verified.}
\chapter{Results \& Evaluation}

\section{Theoretical Results}

\todo{Theoretical contributions of the project}

\section{Implementation}

\todo{Benchmarks(?) of the implemented algorithm, and notes about how
  it is similar/different to typical non-verified algorithms in use.}

\section{Evaluation}

\todo{Things I did badly, or which were really contrived, such as not
  being incremental/concurrent/real-time/whatever.}
\chapter{Conclusion}

\todo{Summary of contribution of project}

\section{Further Work}

\subsection{Specific Improvements}

\todo{Named, specific, things which could be made better}

\subsection{Research}

\todo{Possible research directions suggested by the project}

\section{Final Thoughts}

\todo{How I feel, briefly, about the whole thing. Try to avoid
  references to mounting insanity.}

\begin{appendices}
  % \addcontentsline{toc}{subsection}{Index}
  % \printindex

  \addcontentsline{toc}{subsection}{Glossary}
  \printglossaries

  \addcontentsline{toc}{subsection}{Bibliography}
  \bibliographystyle{plain}
  \bibliography{references}
\end{appendices}

\end{document}
