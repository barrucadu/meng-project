\documentclass[11pt,a4paper,twoside,openany]{report}

% Nicer fonts
\usepackage[T1]{fontenc}
\usepackage{lmodern}

% Graphics support
\usepackage{graphicx}

% Hyperlinks, URLs etc.
\usepackage{hyperref}
\usepackage{url}
\hypersetup{
    colorlinks=true,
    citecolor=black,
    urlcolor=black,
    linkcolor=black,
    pagecolor=black,
    anchorcolor=black
}

% Stuff at the end
\usepackage[toc,page]{appendix}
\usepackage[xindy]{glossaries}
\usepackage[xindy]{imakeidx}
\loadglsentries{glossary.glo}
\renewcommand*{\glossaryentrynumbers}[1]{}
\makeglossaries
\makeindex

% Footnotes
\renewcommand{\thefootnote}{\fnsymbol{footnote}}
\usepackage{perpage}
\MakePerPage{footnote}

% Advanced maths features
\usepackage{amsmath}
\usepackage{amssymb}
\usepackage{amsfonts}
% Conditional macros
\usepackage{xifthen}

% Definitions and theorems
\usepackage{amsthm}
\usepackage{etoolbox}
\newtheorem{definition}{Definition}[section]
\newtheorem{lemma}{Lemma}[section]
\newtheorem{theorem}{Theorem}[section]

\makeatletter
\newtheoremstyle{example}% name
{10pt}% space before
{10pt}% space after
{\addtolength{\@totalleftmargin}{2.5em}
\addtolength{\linewidth}{-2.5em}
\parshape 1 2.5em \linewidth}% body font
{}% indent
{}% header font
{}% header punctuation
{.5em}% gap after header
{\textbf{\thmname{#1}\thmnumber{ #2}:}\textit{\thmnote{ #3}}\\}% header
\makeatother

\AtEndEnvironment{example}{\hfill\ensuremath{\square}}

\theoremstyle{example}
\newtheorem{example}{Example}[section]

% Drawings
\usepackage[usenames,dvipsnames]{color}
\usepackage{tikz}
\usetikzlibrary{trees}

% Page layout jiggery pokery
\usepackage{pdflscape}
\usepackage{multicol}

\newenvironment{wide}{%
  \begin{list}{}{%
      \setlength{\topsep}{0pt}%
      \addtolength{\leftmargin}{-1.5cm}%
      \addtolength{\rightmargin}{-1.5cm}%
      \setlength{\listparindent}{\parindent}%
      \setlength{\itemindent}{\parindent}%
      \setlength{\parsep}{\parskip}}%
  \item[]%
}{%
  \end{list}%
}

\usepackage[inner=2cm,outer=4cm]{geometry}

\newenvironment{lscape}{\begin{landscape}\begin{wide}}{\end{wide}\end{landscape}}

% Logic & Maths
\usepackage{bussproofs}

\newcommand{\htriple}[3]{\left\{#1\right\}\ \mbox{#2}\ \left\{#3\right\}}
\newcommand{\len}{\mathrm{len}~}
\newcommand{\true}{\mathbf{true}}
\newcommand{\false}{\mathbf{false}}
\renewcommand{\iff}{\Leftrightarrow}
\renewcommand{\implies}{\Rightarrow}

\DeclareMathOperator{\sepc}{\ast}
\DeclareMathOperator{\sepi}{---\!\ast}

\newcommand{\encarmap}{\rightharpoonup}
\newcommand{\pointsto}{\hookrightarrow}
\newcommand{\restrict}{\downharpoonright}
\newcommand{\addr}{\mathrm{Addr}}
\newcommand{\data}{\mathrm{Data}}
\newcommand{\nullp}{\mathit{null}}
\newcommand{\listof}[1]{#1\ \mathbf{list}}
\newcommand{\x}{\times}
\newcommand{\of}{\circ}
\newcommand{\reach}[2]{\mathrm{reach}\left(#1, #2\right)}
\newcommand{\garbage}[2]{\mathrm{garbage}\left(#1, #2\right)}
\newcommand{\keep}[2]{\mathrm{keep}\left(#1, #2\right)}
\newcommand{\markset}[2]{\mathrm{mark}\left(#1, #2\right)}
\newcommand{\map}[2]{\mathrm{map}\left(#1, #2\right)}
\newcommand{\rename}[2]{\mathrm{rename}\left(#1, #2\right)}
\newcommand{\filter}[2]{\mathrm{filter}\left(#1, #2\right)}
\newcommand{\gc}{\xrightarrow{gc}}
\newcommand{\translate}{\xrightarrow{translate}}
\newcommand{\id}{\mathrm{id}}
\newcommand{\dom}{\mathrm{dom}}
\newcommand{\gchead}[1]{\mathrm{gchead}\left(#1\right)}
\newcommand{\free}[1]{\mathrm{free}\left(#1\right)}
\newcommand{\marked}[2]{\mathrm{marked}\left(#1, #2\right)}

\newcommand{\ltrue}{\mathcal T}
\newcommand{\lfalse}{\mathcal F}

% Listings
\usepackage{chngcntr} % Renumbering done further down because the
                      % counter is only available after the document
                      % begins, for some reason.
\usepackage{listings}

\definecolor{LightGray}{rgb}{0.9,0.9,0.9}

\lstset{basicstyle=\small\ttfamily}
\lstset{showstringspaces=false}
\lstset{numbers=left, numberstyle=\tiny, stepnumber=1, numbersep=5pt}
\lstset{keywordstyle=\color{MidnightBlue}\bfseries}
\lstset{commentstyle=\color{JungleGreen}}
\lstset{identifierstyle=\color{OliveGreen}}
\lstset{stringstyle=\color{Red}}
\lstset{backgroundcolor=\color{LightGray}}
\lstset{breaklines=true}

\lstdefinestyle{makefile}
{
    numberblanklines=false,
    language=make,
    tabsize=4,
    keywordstyle=\color{red},
    identifierstyle= %plain identifiers for make
}

\title{%
{\normalsize\scshape University of York}\\
\vspace{2em}
\hrule
\vspace{1em}
{\huge\bfseries Memory Management}\\
\vspace{1em}
{\bfseries Verified Garbage Collection}\\
\vspace{1em}
\hrule
\vspace{1em}
{\small Submitted for part completion of MEng Computer Systems and Software Engineering}
\vfill%
{\normalsize Number of words: TODO as counted by ``detex | wc -w'',\\
excluding code listings and appendices.}
}

\author{%
Author: Michael Walker\\
Supervisor: Colin Runciman
}

\date{}

\begin{document}

\counterwithin{lstlisting}{section}

\pagestyle{empty}

\maketitle

\cleardoublepage
\begin{abstract}
Programming languages with automatic garbage collection free a
programmer from worrying about how exactly to manage memory and
prevent leaks and accidental frees. However, there is little formal
verification done currently.

In this project, I provide proofs of correctness for two extant
garbage collectors, implement the collectors with key assertions
checked as they execute, and then extract a more general formalism
from these. Finally, I discuss how this formalism could be applied to
collectors other than those that I considered, and give some
indication of the further work remaining to be done to formalise
garbage collection.

\vfill

\paragraph{Statement of Ethics}

This project concerns only the development and implementation of
algorithms. No human participation is involved at any point and as
such there are no specific ethical implications.
\end{abstract}


% Table of contents, roman numeral page numbers
\cleardoublepage
\pagestyle{plain}
\pagenumbering{roman}
\setcounter{page}{1}
\tableofcontents

%\listoffigures
%\listoftables
\lstlistoflistings

\glsaddall
\printglossaries

% Reset page numbering for content
\newpage
\cleardoublepage
\pagenumbering{arabic}

\chapter{Introduction}
\label{sec:intro}

As programs become more complicated, we have a greater need for
high-level languages which abstract away from much of the underlying
machine. A key part of such a language is a run-time memory management
system, and in particular the automatic disposal of unneeded allocated
memory: that is, garbage collection.

\section{Motivation}
\label{sec:intro-motivation}

Garbage collection is an important part of many modern programming
languages, however there is an implicit assumption that it never goes
wrong. It would not be feasible for the programmer to account for all
the ways in which the run-time support system may fail, and doing so
would remove the benefit of using such a language in the first
place. But these systems are not perfect: they are written and tested
by humans, with formal verification being very uncommon.

An arguably better approach is to design a garbage collector which is
then formally verified, in order to be able to definitively eliminate
all bugs from the collector, and so release the programmer from the
possible worry of how it may fail. For total certainty, any
implementation must be proven to be a faithful translation of the
algorithm to machine code, however merely having a verified algorithm
is a good stepping-stone towards this.

\section{Aims}
\label{sec:intro-goals}



\section{Current Work}
\label{sec:intro-current}



\section{Outline}
\label{sec:intro-outline}

Chapter \ref{sec:lit} summarises the historical context and recent
developments in the field of the project are presented, and is built
upon in Chapter 3 where the literature-motivated aims of the project
are presented.

Chapters \ref{sec:marksweep} to \ref{sec:gc} proceed to formalise
partial correctness for garbage collection, by considering
independently mark-sweep and copying collectors, and then abstracting
from the specifics of both to produce a general formalism.

Finally, Chapters \ref{sec:results} and \ref{sec:conclusion} summarise
the contributions of the project and what remains to be done, as well
as discussing limitations of the work.

\chapter{Literature Review}

This chapter reviews the background of the project: the traditional
methods of garbage collection, more complicated collectors, methods of
algorithm and software verification, and current work in the
production of verified garbage collectors.

\section{Garbage Collection}

Garbage collection is the process of automatically reclaiming unneeded
memory from a program. In languages with manual memory management,
such as C, a programmer can allocate new memory with \texttt{malloc},
but then must remember to release it with \texttt{free}. If this is
not done, a memory leak occurs, and the program may gradually consume
more and more memory.

In a garbage collected language, however, the programmer can allocate
new memory, but typically cannot (and does not need to) explicitly
deallocate it. Instead, some runtime system determines when blocks of
memory are no longer needed, typically by determining if they can be
reached by following pointers from the ``roots''---the set of
variables in scope---or not.

\subsection{The Traditional Algorithms}

We can divide most garbage collection algorithms into four camps:
reference counting, mark-sweep, copying, and mark-compact. Each is
suited to different use-cases, and the choice of which to use is often
determined by the behaviour of the particular system in which they
will be used.

\subsubsection{Reference Counting}

Reference counting is a very simple scheme where each block contains a
\textit{reference count}, some integer field which keeps track of the
number of references to the block. In the most basic form of the
algorithm, the compiler maintains a write barrier, which increments
and decrements the count immediately as references are created or
destroyed. Furthermore, as soon as the count reaches zero, the block
is deallocated\cite{Collins60}.

This strategy seems good, as it distributes the load of memory
management throughout the computation, however it results in the
problem of unbounded time taken to free a block, as freeing it may
cause other things to be freed, and so on\cite{GarbageCollection}.
This problem can be offset by pushing to a free list, rather than
doing the recursive free, or by deferring the reference counting.

The Deutsch--Bobrow algorithm is an example of this kind, where
reference count adjustments are stored in a transaction file, which is
routinely processed to adjust the state of the entire
system\cite{Deutsch76}.

A fatal flaw, alas, with reference counting is that it cannot reclaim
self-referential structures, even if they are
garbage\cite{McBeth63}. Some systems make use of ``weak references'',
references which do not cause the count to be modified, to offset
this. However, this places a burden on the programmer and so loses
some of the convenience of the system.

\subsubsection{Mark-Sweep}

\todo{What it is, when it was introduced, flaws, benefits}

\subsubsection{Copying}

\todo{What it is, when it was introduced, flaws, benefits}

\subsubsection{Mark-Compact}

\todo{What it is, when it was introduced, flaws, benefits}

\subsection{Hybrid Collectors}

\todo{Why they are sometimes good}

\subsubsection{Generational Collectors}

\todo{What it is, when it was introduced, flaws, benefits,
  generational hypotheses}

\section{Algorithm Verification}

\todo{What is verification}

\subsection{Verification by Proof}

\todo{How to take some code and a specification and prove the two equivalent}

\subsection{Verification by Extraction from Specification}

\todo{How to refine a specification until we can pluck out code}

\section{Verified Garbage Collection}

\todo{Work on verifying garbage collectors, using either method}

\chapter{Problem Analysis}
\label{sec:analysis}

In this chapter I shall discuss how I am going to approach the
project. In section \ref{sec:analysis-aims} I will discuss the aims,
with some reflection on the literature; in section
\ref{sec:analysis-development} I will discuss how I am going to
develop the formalism, the algorithm, and prove its correctness with
respect to the formalism; and in section \ref{sec:analysis-evaluation}
I will discuss how I am going to evaluate the success of my project at
the end.

\section{Aims Revisited}
\label{sec:analysis-aims}

Reviewing the literature reveals that much of the recent work in
garbage collection development has been focused on concurrency, and
obtaining real-time guarantees. The topic of verified garbage
collection has received comparatively little interest, and is perhaps
viewed as something which would be nice in an ideal world, but not as
important as making garbage collectors as fast as possible. This is a
shame, as garbage collection bugs can be a very difficult problem to
detect through testing alone.

There have been a few verified garbage collectors, however these tend
to be proofs for specific extant algorithms, or extraction using very
strict definitions of garbage collection. The former typically cannot,
unfortunately, generalise to other collectors, and the latter makes
generalising to types of collectors other than what the author had in
mind very difficult.

This project explores the route of using generic and flexible
formalisms, not tied to any particular class of collector, and which
can be used for the proof or extraction methods. The aim, through
algorithm/formalism co-design, is to produce:

\begin{itemize}
  \item a generic formalism for garbage collection partial
    correctness, in terms of garbage collection as a function applied
    to the program state;

  \item an illustration of the utility of the formalism by proofs of
    correctness of two simple mark-sweep collectors: one specialised
    for immutable languages, and one more general;

  \item design, proof, and implementation of a copying collector.
\end{itemize}

\section{Development Methodology}
\label{sec:analysis-development}

\todo{How I'm going to develop the algorithm and prove it}

\section{Evaluation Methodology}
\label{sec:analysis-evaluation}

The formalism and algorithm shall be evaluated separately.

\subsection{Evaluating the Formalism}
\label{sec:analysis-evaluation-formalism}

The formalism shall be evaluated by considering how well it applies to
a variety of collectors. In this dissertation, only mark-sweep and
copying will be proven, but the proof obligations for other types of
collectors will be examined in order to determine if the formalism is
capable of satisfying them.

Furthermore, what the formalism doesn't state will be discussed, as in
order to make it widely applicable it must abstract over the details
which make each class of collector unique.

Finally, it shall be compared to other formalisms, and the advantages
and disadvantages of it over the others will be discussed.

\subsection{Evaluating the Algorithm}
\label{sec:analysis-evaluation-algorithm}

The algorithm shall be compared to current garbage collection
algorithms, in particular stop-the-world collectors, and the relative
performance and ease of implementation discussed. Whilst performance
is not a primary concern in the project, it would be nice to produce
something of competitive speed.

\subsection{Qualiative Evaluation}
\label{sec:analysis-evaluation-qualitative}

Finally, how ``nice'' the formalism and algorithm are to use will be
considered. The aim of the project is primarily to produce a flexible
and generic formalism, but also to produce a formalism which can be
used to ease the difficulty of proving correctness. Even if the
formalism has everything that could be desired, if it is difficult to
work with, there is scope for improvement.

Similarly, the algorithm should be simplistic and primarily be used to
illustrate the use of the formalism, but it should also be a
realistically viable garbage collection algorithm. Whilst designing
and verifying a concurrent real-time collector is out of scope for
this project, it should still offer something new.

\subsection{Summary}
\label{sec:analysis-summary}

\todo{Summary, once I have written the rest}

\chapter{The Heap as an Array}
\label{sec:heap}

In this chapter I shall explain and justify how the heap can be
modelled as an array, and introduce a specific formalism for reasoning
about array-based programs, along with some examples of its use. This
formalism will be used in later chapters.

In a physical computer, the heap lives in virtual memory, where
virtual memory consists of an integer-indexed collection of machine
words. In the case of x86 computers, these words are bytes, and the
range of addresses allows for 4GiB of virtual memory per
process. Contiguous addresses refer to contiguous heap locations, and
the addressing starts at zero (although this is usually mapped to an
invalid memory location to catch errors).

This sounds rather like a zero-indexed array, and in fact we can
consider the heap to be such an array to allow reasoning. More
abstract formalisms (such as considering the heap to be a digraph) are
possible, and one such alternative (separation logic) shall be
reviewed, however I argue that the array model is the simplest way to
capture the semantics of memory as seen by programs.

In the array formalism, pointers simply become indexes into this
array. We need to be careful about illegal addresses, such as zero,
but by imposing a validity condition upon pointers, such as requiring
they all point to the start of a cell, we can side-step this issue.

\section{The Alternative: Separation Logic}
\label{sec:heap-separation}

Separation logic, introduced by Reynolds in 2002\cite{Reynolds02}, is
an extension of Hoare logic for reasoning about states consisting of a
stack and heap, where the heap is a partial function from addresses to
values. In particular, the logic provides the following extra
predicates to reason about the heap,

\begin{itemize}
  \item $\mathbf{emp}$, the heap is empty;

  \item $e \mapsto e'$, the heap is the singleton heap, where the
    address $e$ maps to the value $e'$;

  \item $P \sepc Q$, the heap can be split into two disjoint partitions
    where $P$ holds in one and $Q$ in the other;

  \item $P \sepi Q$, were the heap to be extended by a disjoint part
    which satisfies $P$, the entire heap would satisfy $Q$.
\end{itemize}

In addition to these operators, there is an additional deduction rule,
allowing modular reasoning about program components,

\begin{prooftree}
  \AxiomC{$\htriple{P}{C}{Q}$}
  \UnaryInfC{$\htriple{P \sepc R}{C}{Q \sepc R}$}
\end{prooftree}

This is known as the frame rule, and is valid if no free variables in
$R$ are modified in $C$. This states that you can take a proof of a
program in a small context, and reuse it in a larger context.

In this formalism pointers are parameters to the partial function
which is the heap, where the heap can be anything. For example, you
could reason about a heap of cells, where pointers are all multiples
of 2. This formalism also allows pointer arithmetic (as does the array
formalism), but reasoning about properties such as reachability is
more difficult in this case.

The main reason I do not use separation logic in this project is
because I will necessarily be reasoning about the entire state, and
there would be very little scope for separation to be of use. Thus,
using separation logic could result in additional complication for
little gain. However, in the case of non-stop-the-world collectors,
the tools of separation logic (or something similar) would undoubtedly
be useful.

\section{Reasoning with Arrays}
\label{sec:heap-arrays}

Abstract program logics such as Hoare logic typically do not worry
about data types beyond simple primitives such as integers and
booleans, however this is not enough to model all aspects of
programming. Gries introduces a model of arrays\cite{Gries87}, where
an array of type $[T]$ is treated as a function from some finite
contiguous subset of the natural numbers (including zero) to $T$, and
array assignment is modelled as functional override.

Henceforth, I shall be using $A$ as an example array.

We denote an array as a tuple, where the $n$th element of the tuple is
the value of $A[n]$. The bounding indices of the array are not
constrained in general, and so these can be accessed by the values
$A.lower$ and $A.upper$, leading to a definition of the domain of an
array $\dom~A = \{i~|~A.lower \leq i \leq A.upper\}$.

\begin{definition}[Array Access Triple]
  Assigning an array member to a variable is exactly the same as
  normal assignment, with the additional precondition that the array
  subscript be in range.

  \begin{align*}
    \htriple{P\left[A[n]/x\right] \land n \in \dom~A}{x := A[n]}{P}
  \end{align*}
\end{definition}

\begin{example}[Summing a list]
  \label{exmpl:heap-sum}
  
  Let's denote by $\mathrm{sum}~A$ (where $A$ is an array) the sum of
  all the elements in it, and now we want to prove the following
  program does satisfy this postcondition.

\begin{verbatim}
i   := A.lower;
sum := 0;
while i <= A.upper do
    sum := sum + A[i];
    i := i + 1
\end{verbatim}

  Specifically, we want to prove $\mathtt{sum} = \mathrm{sum}~A$. We
  shall do so by using this loop invariant, $\mathtt{sum} =
  \mathrm{sum}~A[A.lower : \mathtt{i} - 1]$, where $A[i:j]$ is the
  array consisting of the elements of $A$ between $i$ and $j$, and if
  $i = j + 1$ is the empty array.

  \begin{prooftree}
    \AxiomC{$\ltrue \implies I''$}

    \AxiomC{$\htriple{I''}{L1}{I'}$}
    \AxiomC{$\htriple{I'}{L2}{I}$}
    \BinaryInfC{$\htriple{I''}{L1; L2}{I}$}

    \AxiomC{$\htriple{I \land \mathtt{i} \leq A.upper}{L4}{J}$}
    \AxiomC{$\htriple{J}{L5}{I}$}
    \BinaryInfC{$\htriple{I \land \mathtt{i} \leq A.upper}{L4;
        L5}{I}$}
    \UnaryInfC{$\htriple{I}{while i < A.upper do (L4; L5)}{I \land
        \mathtt{i} > A.upper}$}

    \BinaryInfC{$\htriple{I''}{\ldots}{\mathtt{sum} =
        \mathrm{sum}~A}$}

    \BinaryInfC{$\htriple{\ltrue}{\ldots}{\mathtt{sum} =
        \mathrm{sum}~A}$}
  \end{prooftree}

  Where the following definitions hold,

  \begin{align*}
    J &\iff \mathtt{sum} = \mathrm{sum}~A[A.lower : \mathtt{i} + 1]\\
    I' &\iff I[0/\mathtt{sum}]\\
    &\iff 0 = \mathrm{sum}~A[A.lower : \mathtt{i} - 1]\\
    I'' &\iff I'[A.lower/\mathtt{i}]\\
    &\iff 0 = \mathrm{sum}~A[A.lower : A.lower - 1]
  \end{align*}
\end{example}

\begin{definition}[Array Assignment Triple]
  Assignment to an array is modelled as functional override,

  \begin{align*}
    \htriple{P[(A; i : e)/A] \land i \in \dom~A}{A[i] = e}{P}
  \end{align*}

  Where $(A; i : e)[j]$ evaluates to $e$ if $i = j$, and $A[j]$
  otherwise.
\end{definition}

\begin{example}[Doubling a list]
  \label{exmpl:heap-double}

  We want to prove that the following program doubles the value of
  every element of the array $A$, storing the result in $B$,

\begin{verbatim}
i := A.lower;
B := A;
while i <= A.upper do
    B[i] = B[i] * 2;
    i := i + 1
\end{verbatim}

  And we shall use the invariant $\forall A.lower \leq j < i,\ B[j]
  = 2 A[j]$ to do so.

  \begin{prooftree}
    \AxiomC{$\ltrue \implies I''$}

    \AxiomC{$\htriple{I''}{L1}{I'}$}
    \AxiomC{$\htriple{I'}{L2}{I}$}
    \BinaryInfC{$\htriple{I''}{L1; L2}{I}$}

    \AxiomC{$\htriple{I \land \mathtt{i} \leq A.upper}{L4}{J}$}
    \AxiomC{$\htriple{J}{L5}{I}$}
    \BinaryInfC{$\htriple{I \land \mathtt{i} \leq A.upper}{L4;
        L5}{I}$}
    \UnaryInfC{$\htriple{I}{\ldots}{I \land \mathtt{i} > A.upper}$}

    \BinaryInfC{$\htriple{I''}{\ldots}{D}$}

    \BinaryInfC{$\htriple{\ltrue}{\ldots}{D}$}
  \end{prooftree}

  Where the following definitions hold,

  \begin{align*}
    D &\iff \forall A.lower \leq i < A.upper,\ B[i] = 2 A[i]\\
    J &\iff \forall A.lower \leq j < i + 1,\ B[j] = 2 A[j]\\
    I' &\iff I[B/A]\\
    &\iff \forall A.lower \leq j < i,\ A[j] = 2 A[j]\\
    I'' &\iff I'[A.lower/\mathtt{i}]\\
    &\iff \forall A.lower \leq j < A.lower,\ A[j] = 2 A[j]
  \end{align*}
\end{example}

Thus, we can see that array reasoning is a very simple and intuitive
tool, and as the axioms are very simple, they are amenable to machine
verification or automation (although I will not make use of this).

\subsection{The Heap as an Array for Garbage Collection}
\label{sec:heap-arrays-gc}

\todo{Why this makes thinking about gc easy}

\section{Summary}
\label{sec:heap-summary}

\todo{Summary of chapter: key ideas behind Gries, what will need to be
carried forward}

\include{05-formalism-marksweep}
\include{06-formalism-copying}
\include{07-formalism-general}
\chapter{Results \& Evaluation}
\label{sec:results}

Previously, I have attempted to provide proofs of correctness for two
extant garbage collection algorithms, and from this extract a more
generalised notion of correctness. In this chapter, I shall discuss
and evaluate the results of the two facets of my work, and how
successful I have been in achieving the goal of this project. I shall
also discuss an experimental validation of my proofs involving an
implementation of the two collectors discussed.

\section{Proofs of Collectors}
\label{sec:results-collectors}

In the process of producing a generalised notion of correctness for
garbage collectors, I first provided proofs of the Armstrong/Virding
and Fenichel/Yochelson collectors: two very different approaches, but
both for systems based on fixed-size cons cells. The proofs for
neither collector I considered could be machine verified in their
current form, however I believe that they remain convincing despite
this.

\section{Collector Implementation}
\label{sec:results-impl}

As evidence of the correctness of the proofs, I
implemented\footnote{All of the code produced can be found in Appendix
  \ref{sec:gc-impl}, and online at
  \url{https://github.com/barrucadu/meng-project}.} both of the
collectors in C, and wrote a small test program with a very
constrained heap which constructs a variety of linked lists, and
results in over 200 garbage collections occurring. I then attached
assertions to points in the programs to verify the preconditions,
postconditions, and invariants of the collectors.

The implementations are written in C99, and have been tested with both
clang 3.4 and gcc 4.8.2. They should work under any
standards-compliant compiler, but I have not verified this.

The implementation does have some limitations imposed upon it for
simplicity: there is a fixed-size array of roots, which the programmer
must ensure contains all roots, and the requirement of immutable cells
in the Armstrong/Virding collector is not implemented. Resolving both
of these would have been significantly more work, and are not
necessary in order to demonstrate that the garbage collection works.

Both collectors manage a heap of \texttt{NUM\_CELLS} cells (in the
case of Fenichel/Yochelson, both semispaces are this big), in order to
have identical characteristics. This then allows identical assertions
to be used in the test program. As neither collector comes with an
allocator, I have had to produce allocators, \texttt{alloc()}, which
work with them. The way I have done so is not necessarily the only, or
best, way, it is just what occurred to me and appears to work. In both
cases, the collector is invoked by the allocator upon not having
enough memory left for an allocation. The preconditions are checked
before calling the \texttt{gc()} function, and the postconditions
checked after. In each of the garbage collector loops, the invariants
are checked after each iteration.

Due to using a low level language, the implementations of the
collectors look somewhat more complex than what is specified in their
respective papers, but is as close as I could reasonably obtain.

The assertions, and their implementation are discussed below.

\subsection{Armstrong/Virding}
\label{sec:results-impl-ms}

\subsubsection{Preconditions}
\label{sec:results-impl-ms-pre}

The Armstrong/Virding collector, as I discussed, has a precondition
that the $first$ cell is always reachable, as otherwise it would be
garbage, and due to it never being freed, garbage would survive a
collection. Furthermore, it is assumed that when the collector is
called, all cells pointed to by roots have been marked.

\begin{lstlisting}[language=C,caption={Armstrong/Virding Preconditions}]
assert(reachable(gchead(first)));

for(unsigned int i = 0; i < NUM_ROOTS; i++)
  if(roots[i] != NULL)
    gchead(roots[i])->mark = MARKED;
\end{lstlisting}

\subsubsection{Postconditions}
\label{sec:results-impl-ms-post}

The postcondition express simply the correct mark-sweep postcondition:
after collection, all cells which are allocated (not on the free list)
are reachable from the roots. Furthermore, the live cell invariant is
checked here as a postcondition, although arguably it should be
checked with the rest of the invariants, in the garbage collection
loop.

\begin{lstlisting}[language=C,caption={Armstrong/Virding Postconditions}]
for(unsigned int i = 0; i < NUM_CELLS; i++)
  if(reachable(&heap[i]))
    assert(!on_free_list(&heap[i]));

for(unsigned int i = 0; i < NUM_CELLS; i++)
  if(!on_free_list(&heap[i]))
    assert(reachable(&heap[i]));
\end{lstlisting}

\subsubsection{Invariants}
\label{sec:results-impl-ms-invariants}

The invariants appear somewhat more complex. The first loop expresses
$\forall c \in h,\ \alloc{c} \land \id~c > \id~\mathtt{SCAV} \implies
(\forall c \pointsto x, \id~x < \id~\mathtt{SCAV} \implies
\marked{h}{c})$: all cells outside of the region that has been
collected, but which are pointed to by a cell in that region, have
been marked. The latter two express the simpler properties that all
allocated cells in the collected region are live, and all allocated
cells in the collected region are unmarked.

\begin{lstlisting}[language=C,caption={Armstrong/Virding Invariants}]
for(unsigned int i = 0; i < NUM_CELLS; i++)
  if(reachable(&heap[i]) && cell_id(&heap[i]) > cell_id(gchead(SCAV)))
    {
      if(heap[i].cell.car.tag == REFERENCE &&
         heap[i].cell.car.val.ptr != NULL &&
         cell_id(gchead(heap[i].cell.car.val.ptr)) < cell_id(gchead(SCAV)))
        assert(gchead(heap[i].cell.car.val.ptr)->mark == MARKED);

      if(heap[i].cell.cdr.tag == REFERENCE &&
         heap[i].cell.cdr.val.ptr != NULL &&
         cell_id(gchead(heap[i].cell.cdr.val.ptr)) < cell_id(gchead(SCAV)))
        assert(gchead(heap[i].cell.cdr.val.ptr)->mark == MARKED);
    }

for(unsigned int i = 0; i < NUM_CELLS; i++)
  if(reachable(&heap[i]) && cell_id(&heap[i]) >= cell_id(gchead(SCAV)))
    assert(reachable(&heap[i]));

for(unsigned int i = 0; i < NUM_CELLS; i++)
  if(reachable(&heap[i]) && cell_id(&heap[i]) > cell_id(gchead(SCAV)))
    assert(heap[i].mark == UNMARKED);
\end{lstlisting}

\subsection{Fenichel/Yochelson}
\label{sec:results-impl-c}

\subsubsection{Preconditions}
\label{sec:results-impl-c-pre}

An invariant which I did not mention whilst discussing my proof, but
which the garbage collector, mutator, and allocator must obey, is that
there are no inter-semispace pointers. Due to a bug in my linked list
code (which has since been remedied), a garbage collection occuring
whilst appending a value to a list would result in this invariant
being broken, and resulted in my being stuck for some time before I
realised what the issue was. I then decided to enforce this as a
precondition of the collector.

\begin{lstlisting}[language=C,caption={Fenichel/Yochelson Preconditions}]
cell *sspace = (cons_space == FIRST) ? semispace1 : semispace2;

for(unsigned int i = 0; i < NUM_CELLS; i++)
  if(reachable(&sspace[i]))
    {
      if(sspace[i].car.tag == REFERENCE &&
         sspace[i].car.val.ptr != NULL)
        assert(sspace[i].car.val.ptr >= sspace &&
               sspace[i].car.val.ptr <= &sspace[NUM_CELLS]);

      if(sspace[i].cdr.tag == REFERENCE &&
         sspace[i].cdr.val.ptr != NULL)
        assert(sspace[i].cdr.val.ptr >= sspace &&
               sspace[i].cdr.val.ptr <= &sspace[NUM_CELLS]);
    }
\end{lstlisting}

\subsubsection{Postconditions}
\label{sec:results-impl-c-post}

As a postcondition I again checked the live cell invariant, and also
asserted that all allocated cells are reachable.

\begin{lstlisting}[language=C,caption={Fenichel/Yochelson Postconditions}]
cell* sspace = (cons_space == FIRST) ? semispace1 : semispace2;

for(unsigned int i = 0; i < NUM_CELLS; i++)
  if(reachable(&sspace[i]))
    assert(!on_free_list(&sspace[i]));

for(unsigned int i = 0; i < NUM_CELLS; i++)
  if(!on_free_list(&sspace[i]))
    assert(reachable(&sspace[i]));
\end{lstlisting}

\subsubsection{Invariants}
\label{sec:results-impl-c-invariants}

The invariant I decided to check is that, after collecting every root,
all pointers in cells reachable from that root have been
updated. Actually, I went for the weaker property that pointers
pointed into the correct semispace. Unfortunately, proving the
stronger property would require having access to the pre-state of the
heap, which would require copying it before initiating a garbage
collection.

\begin{lstlisting}[language=C,caption={Fenichel/Yochelson GC Loop}]
for(unsigned int i = 0; i < NUM_ROOTS; i ++)
  if(roots[i] != NULL)
    {
      roots[i] = collect(roots[i]);
      check_pointers_updated(roots[i], NULL);
    }
\end{lstlisting}

\begin{lstlisting}[language=C,caption={Fenichel/Yochelson Pointer Checking}]
static void check_pointers_updated(const cell* root, const cell* cur)
{
  if(root == cur)
    return;

  if(cur == NULL)
    cur = root;

  cell *sspace = (cons_space == FIRST) ? semispace1 : semispace2;

  if(cur->car.tag == REFERENCE &&
     cur->car.val.ptr != ALREADYCOPIED &&
     cur->car.val.ptr != NULL)
    {
      assert(cur->car.val.ptr >= sspace &&
             cur->car.val.ptr <= &sspace[NUM_CELLS]);
      check_pointers_updated(root, cur->car.val.ptr);
    }

  if(cur->cdr.tag == REFERENCE &&
     cur->cdr.val.ptr != NULL)
    {
      assert(cur->cdr.val.ptr >= sspace &&
             cur->cdr.val.ptr <= &sspace[NUM_CELLS]);
      check_pointers_updated(root, cur->cdr.val.ptr);
    }
}
\end{lstlisting}

\section{Generalised Garbage Collection Correctness}
\label{sec:results-correctness}

\todo{Evaluation 2}

\section{Summary}
\label{sec:results-summary}

\todo{Key strengths and weaknesses of work}
\chapter{Conclusion}
\label{sec:conclusion}

\todo{Summary of contribution of project}

\section{Further Work}
\label{sec:conclusion-further}

\subsection{Specific Improvements}
\label{sec:conclusion-further-specific}

\todo{Named, specific, things which could be made better}

\subsection{Research}
\label{sec:conclusion-further-research}

\todo{Possible research directions suggested by the project}

\section{Final Thoughts}
\label{sec:conclusion-thoughts}

\todo{How I feel, briefly, about the whole thing. Try to avoid
  references to mounting insanity.}

\begin{appendices}
\chapter{Checked Garbage Collector Implementations}
\label{sec:gc-impl}

\lstset{language=C}

\begin{wide}
  \section{Armstrong/Virding}
  \label{sec:gc-impl-armstrong-virding}

  \lstset{caption={Implementation of Armstrong/Virding in C}}
  \lstset{label=lst:armstrong-virding-c}
  \lstinputlisting{../code/armstrong-virding.c}

  \lstset{caption={Header file for the Armstrong/Virding C implementation}}
  \lstset{label=lst:armstrong-virding-h}
  \lstinputlisting{../code/armstrong-virding.h}

  \section{Fenichel/Yochelson}
  \label{sec:gc-impl-fenichel-yochelson}

  \lstset{caption={Implementation of Fenichel/Yochelson in C}}
  \lstset{label=lst:fenichel-yochelson-c}
  \lstinputlisting{../code/fenichel-yochelson.c}

  \lstset{caption={Header file for the Fenichel/Yochelson C implementation}}
  \lstset{label=lst:fenichel-yochelson-h}
  \lstinputlisting{../code/fenichel-yochelson.h}

   \section{Test Program}
   \label{sec:gc-impl-test-program}

   \lstset{caption={Linked list ``library'' with swappable GC}}
   \lstset{label=lst:lists-c}
   \lstinputlisting{../code/lists.c}

   \lstset{caption={Header file for the linked list ``library''}}
   \lstset{label=lst:lists-h}
   \lstinputlisting{../code/lists.h}

%   \lstset{caption={Test program}}
%   \lstset{label=lst:main-c}
%   \lstinputlisting{../code/main.c}

%   \lstset{caption={Makefile}}
%   \lstset{label=lst:makefile}
%   \lstset{style=makefile}
%   \lstinputlisting{../code/Makefile}

  \section{Shared Code}
  \label{sec:gc-impl-shared}

  \lstset{caption={Shared root-management code}}
  \lstset{label=lst:shared-c}
  \lstinputlisting{../code/shared.c}

  \lstset{caption={Definition of cells and shared prototypes}}
  \lstset{label=lst:shared-h}
  \lstinputlisting{../code/shared.h}
\end{wide}

\end{appendices}

\bibliographystyle{plain}
\bibliography{references}

\end{document}
