\documentclass[11pt,a4paper,twoside,openany]{report}

% Nicer fonts
\usepackage[T1]{fontenc}
\usepackage{lmodern}

% Graphics support
\usepackage{graphicx}

% Hyperlinks, URLs etc.
\usepackage{hyperref}
\usepackage{url}
\hypersetup{
    colorlinks=true,
    citecolor=black,
    urlcolor=black,
    linkcolor=black,
    pagecolor=black,
    anchorcolor=black
}

% Stuff at the end
\usepackage[toc,page]{appendix}
\usepackage[xindy]{glossaries}
\usepackage[xindy]{imakeidx}
\loadglsentries{glossary.glo}
\renewcommand*{\glossaryentrynumbers}[1]{}
\makeglossaries
\makeindex

% Footnotes
\renewcommand{\thefootnote}{\fnsymbol{footnote}}
\usepackage{perpage}
\MakePerPage{footnote}

% Advanced maths features
\usepackage{amsmath}
\usepackage{amssymb}
\usepackage{amsfonts}
% Conditional macros
\usepackage{xifthen}

% Definitions and theorems
\usepackage{amsthm}
\usepackage{etoolbox}
\newtheorem{definition}{Definition}[section]
\newtheorem{lemma}{Lemma}[section]
\newtheorem{theorem}{Theorem}[section]

\makeatletter
\newtheoremstyle{example}% name
{10pt}% space before
{10pt}% space after
{}% body font
{}% indent
{}% header font
{}% header punctuation
{.5em}% gap after header
{\textbf{\thmname{#1}\thmnumber{ #2}:}\textit{\thmnote{ #3}}\\}% header
\makeatother

\AtEndEnvironment{example}{\hfill\ensuremath{\square}}

\theoremstyle{example}
\newtheorem{example}{Example}[section]

% Drawings
\usepackage[usenames,dvipsnames]{color}
\usepackage{tikz}
\usetikzlibrary{trees}

% Page layout jiggery pokery
\usepackage{pdflscape}
\usepackage{multicol}

\usepackage[inner=2cm,outer=4cm]{geometry}

\newenvironment{wide}{%
  \begin{list}{}{%
      \setlength{\topsep}{0pt}%
      \addtolength{\leftmargin}{-1.5cm}%
      \addtolength{\rightmargin}{-1.5cm}%
      \setlength{\listparindent}{\parindent}%
      \setlength{\itemindent}{\parindent}%
      \setlength{\parsep}{\parskip}}%
  \item[]%
}{%
  \end{list}%
}

\newenvironment{lscape}{\begin{landscape}\begin{wide}}{\end{wide}\end{landscape}}

\newenvironment{lscape2col}{%
  \begin{landscape}%
    \setlength{\topsep}{0pt}%
    \addtolength{\leftmargin}{-1cm}%
    \addtolength{\rightmargin}{-1cm}%
    \setlength{\listparindent}{\parindent}%
    \setlength{\itemindent}{\parindent}%
    \setlength{\parsep}{\parskip}%
    \setlength{\columnsep}{1.5cm}%
    \begin{multicols}{2}%
}{%
    \end{multicols}%
  \end{landscape}%
}

% Logic & Maths
\usepackage{bussproofs}

\newcommand{\htriple}[3]{\left\{#1\right\}\ \mbox{#2}\ \left\{#3\right\}}
\newcommand{\len}{\mathrm{len}~}
\newcommand{\true}{\mathbf{true}}
\newcommand{\false}{\mathbf{false}}
\renewcommand{\iff}{\Leftrightarrow}
\renewcommand{\implies}{\Rightarrow}

\DeclareMathOperator{\sepc}{\ast}
\DeclareMathOperator{\sepi}{---\!\ast}

\newcommand{\encarmap}{\rightharpoonup}
\newcommand{\pointsto}{\hookrightarrow}
\newcommand{\restrict}{\downharpoonright}
\newcommand{\addr}{\mathrm{Addr}}
\newcommand{\data}{\mathrm{Data}}
\newcommand{\nullp}{\mathit{null}}
\newcommand{\listof}[1]{#1\ \mathbf{list}}
\newcommand{\x}{\times}
\newcommand{\of}{\circ}
\newcommand{\reach}[2]{\mathrm{reach}\left(#1, #2\right)}
\newcommand{\garbage}[2]{\mathrm{garbage}\left(#1, #2\right)}
\newcommand{\keep}[2]{\mathrm{keep}\left(#1, #2\right)}
\newcommand{\markset}[2]{\mathrm{mark}\left(#1, #2\right)}
\newcommand{\map}[2]{\mathrm{map}\left(#1, #2\right)}
\newcommand{\rename}[2]{\mathrm{rename}\left(#1, #2\right)}
\newcommand{\filter}[2]{\mathrm{filter}\left(#1, #2\right)}
\newcommand{\gc}{\xrightarrow{gc}}
\newcommand{\translate}{\xrightarrow{translate}}
\newcommand{\id}{\mathrm{id}}
\newcommand{\dom}{\mathrm{dom}}
\newcommand{\gchead}[1]{\mathrm{gchead}\left(#1\right)}
\newcommand{\free}[1]{\mathrm{free}\left(#1\right)}
\newcommand{\marked}[2]{\mathrm{marked}\left(#1, #2\right)}

\newcommand{\ltrue}{\mathcal T}
\newcommand{\lfalse}{\mathcal F}

% Listings
\usepackage{chngcntr} % Renumbering done further down because the
                      % counter is only available after the document
                      % begins, for some reason.
\usepackage{listings}

\definecolor{LightGray}{rgb}{0.9,0.9,0.9}

\lstset{basicstyle=\footnotesize\ttfamily}
\lstset{showstringspaces=false}
\lstset{numbers=left, numberstyle=\tiny, stepnumber=1, numbersep=5pt}
\lstset{keywordstyle=\color{MidnightBlue}\bfseries}
\lstset{commentstyle=\color{JungleGreen}}
\lstset{identifierstyle=\color{OliveGreen}}
\lstset{stringstyle=\color{Red}}
\lstset{backgroundcolor=\color{LightGray}}
\lstset{breaklines=true}

\lstdefinestyle{makefile}
{
    numberblanklines=false,
    language=make,
    tabsize=4,
    keywordstyle=\color{red},
    identifierstyle= %plain identifiers for make
}

\title{%
{\normalsize\scshape University of York}\\
\vspace{2em}
\hrule
\vspace{1em}
{\huge\bfseries Memory Management}\\
\vspace{1em}
{\bfseries Verified Garbage Collection}\\
\vspace{1em}
\hrule
\vspace{1em}
{\small Submitted for part completion of MEng Computer Systems and Software Engineering}
\vfill%
{\normalsize Number of words: TODO as counted by ``detex | wc -w'',\\
excluding code listings and appendices.}
}

\author{%
Author: Michael Walker\\
Supervisor: Colin Runciman
}

\date{}

\begin{document}

\counterwithin{lstlisting}{section}

\pagestyle{empty}

\maketitle

\cleardoublepage
\begin{abstract}
Programming languages with automatic garbage collection free a
programmer from worrying about how exactly to manage memory and
prevent leaks and accidental frees. However, there is little formal
verification done currently.

In this project, I provide proofs of correctness for two extant
garbage collectors, implement the collectors with key assertions
checked as they execute, and then extract a more general formalism
from these. Finally, I discuss how this formalism could be applied to
collectors other than those that I considered, and give some
indication of the further work remaining to be done to formalise
garbage collection.

\vfill

\paragraph{Statement of Ethics}

This project concerns only the development and implementation of
algorithms. No human participation is involved at any point and as
such there are no specific ethical implications.
\end{abstract}


% Table of contents, roman numeral page numbers
\cleardoublepage
\pagestyle{plain}
\pagenumbering{roman}
\setcounter{page}{1}
\tableofcontents

%\listoffigures
%\listoftables
\lstlistoflistings

\glsaddall
\printglossaries

% Reset page numbering for content
\newpage
\cleardoublepage
\pagenumbering{arabic}

\chapter{Introduction}

As systems become more complicated, we have a greater need for
higher-level languages which abstract away from much of the underlying
machine. A key part of such a language is a run-time \gls{memory
  management} system, and in particular the automatic disposal of
unneeded allocated memory: that is, \gls{garbage collection}.

\section{Motivation}

Garbage collection is an important part of many modern programming
languages, however there is an implicit assumption that it never goes
wrong. It would not be feasible for the programmer to account for all
the ways in which the run-time support system may fail, and doing so
would remove the benefit of using a managed language in the first
place. But these systems are not perfect: they are written and tested
by humans, with formal verification being very uncommon.

An arguably better approach is to design a \gls{garbage collector}
which is then formally verified, in order to be able to definitively
eliminate all bugs from the \gls{collector}, and so release the
programmer from the possible worry of how it may fail. For total
certainty, any implementation must be proven to be a faithful
translation of the algorithm to machine code, however merely having a
verified algorithm is a good stepping-stone towards this.

\section{Outline}

Chapter 2 summarises the historical context and recent developments in
the field of the project are presented, and is built upon in Chapter 3
where the full aims and motivations of the project are presented.

Chapter 4 proceeds to formalise partial correctness for
garbage collection, and justifies the formalism chosen by proving
correct two different collectors.

Chapter 5 develops a garbage collection algorithm, proves it
correct using the formalism, and then shows an implementation of the
algorithm in a language runtime. The relation between algorithm and
code is then discussed.

Finally, Chapters 6 and 7 summarise the contributions of the project
and what remains to be done, as well as discussing limitations of the
work.

\chapter{Literature Review}

\todo{This is all very general: perhaps make more specific as more
  details of the project get worked out}

This chapter reviews the background of the project: the traditional
methods of garbage collection, more complicated collectors, methods of
algorithm and software verification, and current work in the
production of verified garbage collectors.

\section{Garbage Collection}

Garbage collection is the process of automatically reclaiming unneeded
memory from a program. In languages with manual memory management,
such as C, a programmer can allocate new memory with \texttt{malloc},
but then must remember to release it with \texttt{free}. If this is
not done, a memory leak occurs, and the program may gradually consume
more and more memory.

In a garbage collected language, however, the programmer can allocate
new memory, but typically cannot (and does not need to) explicitly
deallocate it. Instead, some runtime system determines when blocks of
memory are no longer needed, typically by determining if they can be
reached by following pointers from the ``roots''---the set of
variables in scope---or not.

\subsection{The Traditional Algorithms}

We can divide most garbage collection algorithms into four camps:
reference counting, mark-sweep, copying, and mark-compact. Each is
suited to different use-cases, and the choice of which to use is often
determined by the behaviour of the particular system in which they
will be used.

\subsubsection{Reference Counting}

Reference counting is a very simple scheme where each block contains a
\textit{reference count}, some integer field which keeps track of the
number of references to the block. In the most basic form of the
algorithm, the compiler maintains a write barrier, which increments
and decrements the count immediately as references are created or
destroyed. Furthermore, as soon as the count reaches zero, the block
is deallocated\cite{Collins60}.

This strategy seems good, as it distributes the load of memory
management throughout the computation, however it results in the
problem of unbounded time taken to free a block, as freeing it may
cause other things to be freed, and so on\cite{GarbageCollection}.
This problem can be offset by pushing to a free list, rather than
doing the recursive free, or by deferring the reference counting.

The Deutsch--Bobrow algorithm is an example of this kind, where
reference count adjustments are stored in a transaction file, which is
routinely processed to adjust the state of the entire
system\cite{Deutsch76}.

A fatal flaw, alas, with reference counting is that it cannot reclaim
self-referential structures, even if they are
garbage\cite{McBeth63}. Some systems make use of ``weak references'',
references which do not cause the count to be modified, to offset
this. However, this places a burden on the programmer and so loses
some of the convenience of the system.

\subsubsection{Mark-Sweep}

The mark-sweep method was invented for the early Lisp
systems\cite{McCarthy60}, and it functions by giving every block a
\textit{mark bit}, which is initially unset. Upon running out of
memory, all cells reachable from the roots are marked (this can be
done recursively quite simply), and then all blocks in the heap are
examined: those which are unmarked are freed, and those which are
marked are unmarked, ready for the next collection.

Compared to reference counting, mark-sweep is a very different beast:
it handles cycles easily and there is no overhead on pointer
operations. It does, however, ``stop the world'' in order to perform a
collection\cite{GarbageCollection}. Furthermore, garbage collection
becomes more frequent as heap residency increases.

These problems can be offset somewhat by interleaving the sweeping
with the mutator (the user program, so called because it mutates the
heap). Hughes's lazy sweep algorithm\cite{Hughes82} does this by
performing a fixed amount of sweeping at each allocation, and so the
only long garbage collection pause happens when the heap needs to be
marked.

\subsubsection{Copying}

Copying collectors divide the heap into two \textit{semispaces}, where
allocation is only done in one of them at a time. In garbage
collection, all live block are copied to the other semispace, and the
roles of the semispaces are swapped\cite{Fenichel69}. Unlike
mark-sweep collectors, the pause time of a copying collector is
proportional only to the number of live blocks in the heap, rather
than the size of the entire heap.

Copying collectors have become fairly popular due to the low cost of
allocation and reduction of fragmentation, however it has a cost of a
half of the heap space\cite{GarbageCollection}.

\subsubsection{Mark-Compact}

Heap fragmentation can be a great problem: there may be more than
enough space to allocate something, but not enough contiguous space to
do so. Furthermore, high fragmentation can increase the rate of page
faults and cache misses, reducing the performance of the mutator.

Thus, we come to the mark-compact collectors. These first mark the
live portion of the heap, and then copy marked blocks over garbage
ones, moving everything towards one end of the heap in order to remove
fragmentation. There are three ways of performing this compacting:
blocks can be moved without regard for their original order; blocks
which point to each other can be placed next to each other; and blocks
can simply ``slide'' towards one end of the heap.

Unfortunately, these collectors require multiple traversals over the
heap, and so are potentially even slower than mark-sweep
collectors\cite{GarbageCollection}.

\subsection{Hybrid Collectors}

Sometimes any one of the traditional algorithms is not quite good
enough for a particular problem, and a better collector can be formed
by combining them. For example, reference counting could be combined
with periodic copying, in order to reclaim cyclic structures and
reduce fragmentation. For another example, Immix\cite{Blackburn08} is
a hybrid mark-region (memory is divided into large regions, inside
which allocation occurs) and copying collector.

\subsubsection{Generational Collectors}

Generational garbage collection is a type of hybrid collector: the
heap is divided into a number of \textit{generations}, where each
generation holds progressively older blocks. This is based on
empirical observations of block lifetimes resulting in the
generational hypothesis: young objects die quickly, whereas old
objects stick around\cite{Ungar84}.

Allocation happens in the youngest generation, and when that fills up
a \textit{minor collection} is started. The generation is garbage
collected, and old enough blocks get copied to the next generation. In
the simplest case, all live blocks at the time of a minor collection
get promoted. If the entire heap is full, a \textit{major collection}
is started, which uses some other garbage collection
algorithm\cite{GarbageCollection}.

Generational garbage collection tends to perform well for languages
where old-to-young pointers are rare, such as most functional
languages, and so has become fairly popular amongst implementers of
such languages.

\section{Algorithm Verification}

\todo{What is verification}

\subsection{Verification by Proof}

\todo{How to take some code and a specification and prove the two equivalent}

\subsection{Verification by Extraction}

\todo{How to refine a specification until we can pluck out code}

\subsection{Proof Assistants}

\todo{If I decide to use one, brief summary of what they are, and then
  a slightly more detailed subsubsection about the one I will use}

\section{Verified Garbage Collection}

Finally, we look at two examples of garbage collectors which have been
verified in different ways.

\subsection{Reusable verification of a copying collector}

Myreen\cite{Myreen10} has constructed a verified garbage collector
using the extraction method: correctness was formalised using the HOL4
theorem prover, was refined to produce an algorithm for correct
copying collectors, and was then refined further to the level of
verified machine code.

The collectors produced have been named L1, which is a specification
of garbage collection; L2, which is an implementation of L1 as the
transitive closure of a step relation; L3, which is a deterministic
implementation of L2 with more realistic memory semantics; L4, which
uses actual implementation types; and L5, which is verified machine
code. L5 is constructed by earlier work of Myreen, on constructing a
certifying compiler.

Other than the obvious, the main contribution of Myreen is that his
proof is quite generic, as only the lowest level of refinement (L5)
makes use of a specific programming logic, and could be applied to many
different copying collectors, just by changing the specific refinement
steps taken.

It is possible that his L1 collector could be applied to non-copying
collectors, providing an already formalised notion of correctness
``for free''.

\subsection{Local Reasoning about a Copying Garbage Collector}

Birkedal, Torp-Smith, and Reynolds\cite{Birkedal04} take a very
different approach to Myreen, they take an existing garbage collector
(Cheney's copying collector) and prove it correct, using a variant of
separation logic.

Similarly to Myreen, they define correctness in terms of a heap
isomorphism modulo garbage cells, and in particular, that the
resultant heap has no garbage cells whatsoever.

The motivation of this paper is to be a stepping stone towards
programming logics which incorporate garbage collection, and so being
able to prove the correctness of the combination of a program and a
language runtime, which has thus far not been feasible.

This paper contributes to the field an extension of separation logic
with semantics for more types of assertion over finite sets, which is
used to express the isomorphism property of the algorithm. Needless to
say, this is an essential property of any garbage collector which
moves data around in the heap.

The major difference between these two verified garbage collectors is
that Myreen used extraction whereas Birkedal \textit{et al.} used
proof. Thus, this proof cannot really be generalised to other
collectors (although the definition of correctness), whereas Myreen's
proofs are much more applicable to copying collectors in general.

\chapter{The Heap as an Array}
\label{sec:heap}

In this chapter I shall explain and justify how the heap can be
modelled as an array, and introduce a specific formalism for reasoning
about array-based programs, along with some examples of its use. This
formalism will be used in later chapters.

In a physical computer, the heap lives in virtual memory, where
virtual memory consists of an integer-indexed collection of machine
words. In the case of x86 computers, these words are bytes, and the
range of addresses allows for 4GiB of virtual memory per
process. Contiguous addresses refer to contiguous heap locations, and
the addressing starts at zero (although this is usually mapped to an
invalid memory location to catch errors).

This sounds rather like a zero-indexed array, and in fact we can
consider the heap to be such an array to allow reasoning. More
abstract formalisms (such as considering the heap to be a digraph) are
possible, and one such alternative (separation logic) shall be
reviewed, however I argue that the array model is the simplest way to
capture the semantics of memory as seen by programs.

In the array formalism, pointers simply become indexes into this
array. We need to be careful about illegal addresses, such as zero,
but by imposing a validity condition upon pointers, such as requiring
they all point to the start of a cell, we can side-step this issue.

\section{The Alternative: Separation Logic}
\label{sec:heap-separation}

Separation logic, introduced by Reynolds in 2002\cite{Reynolds02}, is
an extension of Hoare logic for reasoning about states consisting of a
stack and heap, where the heap is a partial function from addresses to
values. In particular, the logic provides the following extra
predicates to reason about the heap,

\begin{itemize}
  \item $\mathbf{emp}$, the heap is empty;

  \item $e \mapsto e'$, the heap is the singleton heap, where the
    address $e$ maps to the value $e'$;

  \item $P \sepc Q$, the heap can be split into two disjoint partitions
    where $P$ holds in one and $Q$ in the other;

  \item $P \sepi Q$, were the heap to be extended by a disjoint part
    which satisfies $P$, the entire heap would satisfy $Q$.
\end{itemize}

In addition to these operators, there is an additional deduction rule,
allowing modular reasoning about program components,

\begin{prooftree}
  \AxiomC{$\htriple{P}{C}{Q}$}
  \UnaryInfC{$\htriple{P \sepc R}{C}{Q \sepc R}$}
\end{prooftree}

This is known as the frame rule, and is valid if no free variables in
$R$ are modified in $C$. This states that you can take a proof of a
program in a small context, and reuse it in a larger context.

In this formalism pointers are parameters to the partial function
which is the heap, where the heap can be anything. For example, you
could reason about a heap of cells, where pointers are all multiples
of 2. This formalism also allows pointer arithmetic (as does the array
formalism), but reasoning about properties such as reachability is
more difficult in this case.

The main reason I do not use separation logic in this project is
because I will necessarily be reasoning about the entire state, and
there would be very little scope for separation to be of use. Thus,
using separation logic could result in additional complication for
little gain. However, in the case of non-stop-the-world collectors,
the tools of separation logic (or something similar) would undoubtedly
be useful.

\section{Reasoning with Arrays}
\label{sec:heap-arrays}

Abstract program logics such as Hoare logic typically do not worry
about data types beyond simple primitives such as integers and
booleans, however this is not enough to model all aspects of
programming. Gries introduces a model of arrays\cite{Gries87}, where
an array of type $[T]$ is treated as a function from some finite
contiguous subset of the natural numbers (including zero) to $T$, and
array assignment is modelled as functional override.

Henceforth, I shall be using $A$ as an example array.

We denote an array as a tuple, where the $n$th element of the tuple is
the value of $A[n]$. The bounding indices of the array are not
constrained in general, and so these can be accessed by the values
$A.lower$ and $A.upper$, leading to a definition of the domain of an
array $\dom~A = \{i~|~A.lower \leq i \leq A.upper\}$.

\begin{definition}[Array Access Triple]
  Assigning an array member to a variable is exactly the same as
  normal assignment, with the additional precondition that the array
  subscript be in range.

  \begin{align*}
    \htriple{P\left[A[n]/x\right] \land n \in \dom~A}{x := A[n]}{P}
  \end{align*}
\end{definition}

\begin{example}[Summing a list]
  \label{exmpl:heap-sum}
  
  Let's denote by $\mathrm{sum}~A$ (where $A$ is an array) the sum of
  all the elements in it, and now we want to prove the following
  program does satisfy this postcondition.

\begin{verbatim}
i   := A.lower;
sum := 0;
while i <= A.upper do
    sum := sum + A[i];
    i := i + 1
\end{verbatim}

  Specifically, we want to prove $\mathtt{sum} = \mathrm{sum}~A$. We
  shall do so by using this loop invariant, $\mathtt{sum} =
  \mathrm{sum}~A[A.lower : \mathtt{i} - 1]$, where $A[i:j]$ is the
  array consisting of the elements of $A$ between $i$ and $j$, and if
  $i = j + 1$ is the empty array.

  \begin{prooftree}
    \AxiomC{$\ltrue \implies I''$}

    \AxiomC{$\htriple{I''}{L1}{I'}$}
    \AxiomC{$\htriple{I'}{L2}{I}$}
    \BinaryInfC{$\htriple{I''}{L1; L2}{I}$}

    \AxiomC{$\htriple{I \land \mathtt{i} \leq A.upper}{L4}{J}$}
    \AxiomC{$\htriple{J}{L5}{I}$}
    \BinaryInfC{$\htriple{I \land \mathtt{i} \leq A.upper}{L4;
        L5}{I}$}
    \UnaryInfC{$\htriple{I}{while i < A.upper do (L4; L5)}{I \land
        \mathtt{i} > A.upper}$}

    \BinaryInfC{$\htriple{I''}{\ldots}{\mathtt{sum} =
        \mathrm{sum}~A}$}

    \BinaryInfC{$\htriple{\ltrue}{\ldots}{\mathtt{sum} =
        \mathrm{sum}~A}$}
  \end{prooftree}

  Where the following definitions hold,

  \begin{align*}
    J &\equiv \mathtt{sum} = \mathrm{sum}~A[A.lower : \mathtt{i} + 1]\\
    I' &\equiv I[0/\mathtt{sum}]\\
    &\equiv 0 = \mathrm{sum}~A[A.lower : \mathtt{i} - 1]\\
    I'' &\equiv I'[A.lower/\mathtt{i}]\\
    &\equiv 0 = \mathrm{sum}~A[A.lower : A.lower - 1]
  \end{align*}
\end{example}

\begin{definition}[Array Assignment Triple]
  Assignment to an array is modelled as functional override,

  \begin{align*}
    \htriple{P[(A; i : e)/A] \land i \in \dom~A}{A[i] = e}{P}
  \end{align*}

  Where $(A; i : e)[j]$ evaluates to $e$ if $i = j$, and $A[j]$
  otherwise.
\end{definition}

\begin{example}[Doubling a list]
  \label{exmpl:heap-double}

  We want to prove that the following program doubles the value of
  every element of the array $A$, storing the result in $B$,

\begin{verbatim}
i := A.lower;
B := A;
while i <= A.upper do
    B[i] = B[i] * 2;
    i := i + 1
\end{verbatim}

  And we shall use the invariant $\forall A.lower \leq j < i,\ B[j]
  = 2 A[j]$ to do so.

  \begin{prooftree}
    \AxiomC{$\ltrue \implies I''$}

    \AxiomC{$\htriple{I''}{L1}{I'}$}
    \AxiomC{$\htriple{I'}{L2}{I}$}
    \BinaryInfC{$\htriple{I''}{L1; L2}{I}$}

    \AxiomC{$\htriple{I \land \mathtt{i} \leq A.upper}{L4}{J}$}
    \AxiomC{$\htriple{J}{L5}{I}$}
    \BinaryInfC{$\htriple{I \land \mathtt{i} \leq A.upper}{L4;
        L5}{I}$}
    \UnaryInfC{$\htriple{I}{\ldots}{I \land \mathtt{i} > A.upper}$}

    \BinaryInfC{$\htriple{I''}{\ldots}{D}$}

    \BinaryInfC{$\htriple{\ltrue}{\ldots}{D}$}
  \end{prooftree}

  Where the following definitions hold,

  \begin{align*}
    D &\equiv \forall A.lower \leq i < A.upper,\ B[i] = 2 A[i]\\
    J &\equiv \forall A.lower \leq j < i + 1,\ B[j] = 2 A[j]\\
    I' &\equiv I[B/A]\\
    &\equiv \forall A.lower \leq j < i,\ A[j] = 2 A[j]\\
    I'' &\equiv I'[A.lower/\mathtt{i}]\\
    &\equiv \forall A.lower \leq j < A.lower,\ A[j] = 2 A[j]
  \end{align*}
\end{example}

Thus, we can see that array reasoning is a very simple and intuitive
tool, and as the axioms are very simple, they are amenable to machine
verification or automation (although I will not make use of this).

\section{Summary}
\label{sec:heap-summary}

In this chapter I introduced a simple formalism for reasoning about
arrays, and showed two small examples of its use. I argued that
modelling the heap as an array is, whilst a very simplistic approach,
entirely appropriate for when the entirety of the heap is being
reasoned about, as I intend to do so in this dissertation.

In future chapters the heap will be treated as an array, and so the
reader should keep in mind the access and assignment rules. However, I
shall be defining pointers in such a way that they are always valid,
and so the reader does not need to concern themselves with checking if
heap access is within bounds or not, and I shall drop the bounds
checking for simplicity.

\chapter{Correctness of Mark-Sweep Collectors}
\label{sec:marksweep}

This chapter talks about the necessary formalisms and proof
obligations for mark-sweep collectors. Firstly, we define the basic
language necessary for talking about garbage collection in Section
\ref{sec:marksweep-defns}. The problem of correctness is decomposed
into marking and sweeping in Sections \ref{sec:marksweep-marking} and
\ref{sec:marksweep-sweeping}, and finally we build up to a proof of
the Armstrong/Virding\cite{Armstrong95} collector in Section
\ref{sec:marksweep-example}.

\section{The Language of Garbage Collection}
\label{sec:marksweep-defns}

\paragraph{Preliminary Definitions}
The following terms are used throughout the dissertation: $h$, a heap;
$h'$ the state of a heap $h$ before garbage collection; $r$, the root
set; $a \pointsto b$, the cell $a$ contains a pointer to the cell $b$;
$\free{c}$, true iff the cell $c$ is on the free list; and
$\reach{h}{r}$, the set of cells reachable from $r$ in
$h$. Furthermore, quantifiers are used as follows: $\forall c \in h$
ranges over pointers to cells in the heap $h$; and $\forall w \in c$
ranges over addresses within a cell $c$. The definitions of $\free{c}$
and $\reach{h}{r}$ are expressed more formally later on.

\begin{definition}[Heap]
  \label{def:ms-heap}
  The heap is an array of words in the form (type, data).
\end{definition}

Firstly, we need a heap. The convention that the heap is a flat array
has been used, and so pointers are just indices into this array. This
allows the use of Gries' array formalism\cite{Gries87}. Furthermore,
type data is needed for each word, as there must be some way of
distinguishing pointer data from non-pointer data. It may seem that
needing to tag every single word with a type is wasteful, but in the
case where the collector is able to determine this automatically, the
type can be stored, as it were, in a ghost variable.

\begin{definition}[Cell]
  \label{def:ms-cell}
  A cell is a contiguous sequence of one or more words in the heap,
  starting with a sequence of one or more metadata words used by the
  garbage collector, known as the header of the cell.
\end{definition}

Next we move on to cells, at least an entire word has been reserved
for the use of the garbage collector, but this is quite generous, and
in practice would be reduced as much as possible. Reserving space
simplifies proof, and can be worked around in practice by transforming
a more efficient garbage collector into a less efficient one, and then
performing the proof: if the transformation is correct, the proof will
hold for the original, albeit with slightly different space
characteristics.

\begin{definition}[Pointer]
  \label{def:ms-pointer}
  A pointer is an index into the heap array, pointing to the start of
  the data section of a cell.
\end{definition}

Pointers have been restricted to only pointing at the start of the
data portion of a cell,\footnote{It would seem, at first glance, that
  this restriction in conjunction with the definition of words in the
  heap does not allow null pointers, and this would be the case if we
  interpreted the definition of the heap to mean ``all words have a
  static type'', but if we allow the type of a word to be determined
  at runtime, then this is not the case. The type rule for pointers
  could, in practice, be something like ``resides in the nth word of a
  cell and is the address of the start of the data portion of another
  cell'', in which case invalid pointers (such as null pointers) would
  be considered to be data.} as then going from an internal pointer to
a cell header does not need to be considered at all, furthermore, in
practice the mutator does not need to know about the metadata, and
should not know: it should be for purely internal use. If the size and
layout of the header is known in advance, any word in it can be
accessed by pointer arithmetic. Now we'll define a helper function
which will be of use later on: given a pointer, $p$, let $\gchead{p}$
return the heap index for the start of the garbage collection
metadata.

\begin{definition}[Word Preservation]
  \label{def:ms-word-preservation}
  After garbage collection, no allocated cells have been mutated.
   \[\forall c \in h,\ \forall w \in c,\ \lnot\free{c} \implies h[w] =
   h'[w]\]
\end{definition}

A short word needs to be said on what it means for a cell to be
allocated. Conceptually, cells are either allocated, in which case
they may or may not be garbage, or deallocated, in which case the
allocator can hand them out to the mutator. However, this is a
temporal property, and difficult to define as a predicate. Thus,
rather than talk about a cell being allocated or not, we shall talk
about it being on the free list or not: this is what the predicate
$\free{c}$ expresses.

Word preservation is a necessary property of an entire mark-sweep
collector, but may be broken by the marker or the sweeper, as long as
it is restored at the end. As such, it has not included it in the
definitions of correctness in the following two sections, but it is
still essential.

Finally, in order to ensure that pointers only point to sane
locations, we must show that the live cells and the free cells are
disjoint,

\begin{definition}[Live Cell Invariant]
  \label{def:ms-live-cell-invariant}
  No live cells contain a pointer to a cell on the free list.
  \[\forall c \in \reach{h}{r},\ \lnot\free{c}\]
\end{definition}

\section{Marking}
\label{sec:marksweep-marking}

This section shall attack the problem of correct marking, firstly as a
problem in graph theory, and then as a more concrete problem when
dealing with an array of memory with pointers.

Let's start out by saying exactly what goes in that garbage collection
metadata. The first word is a mark flag, equal to 0 if the cell has
not been marked, and any other value if it has.

\begin{definition}[Reachable]
  \label{def:ms-reachable}
  A cell is reachable if it is pointed to by a root, or pointed to by
  a cell which is reachable.

  \begin{minipage}{.5\textwidth}
    \begin{prooftree}
      \AxiomC{$r \in roots$}
      \AxiomC{$r \pointsto a$}
      \BinaryInfC{$a \in \reach{h}{roots}$}
    \end{prooftree}
  \end{minipage}
  \begin{minipage}{.5\textwidth}
    \begin{prooftree}
      \AxiomC{$a \in \reach{h}{roots}$}
      \AxiomC{$a \pointsto b$}
      \BinaryInfC{$b \in \reach{h}{roots}$}
    \end{prooftree}
  \end{minipage}
\end{definition}

If we abandon our array-based interpretation of memory for the moment,
and instead consider the heap to be a digraph of cells connected by
pointers, then reachabilty can be defined inductively over the nodes
of the graph very simply, and this lends itself to implementation with
a recursive function. If we denote the set of all marked, and
allocated, cells by $\markset{h}{roots}$, then what we want after
marking is $\markset{h}{roots} = \reach{h}{roots}$.

Returning to the array formalism, what we thus want from a marker is
the following property:

\begin{definition}[Correct Marking]
  \label{def:ms-correct-marking}
  After marking, for all allocated cells, the mark flag is unset if
  and only if the cell is not reachable.

  \[\forall c \in h,\ \lnot\free{c} \implies
  \left(h\left[\gchead{c}\right] = 0 \iff c \notin
    \reach{h}{roots}\right)\]
\end{definition}

Note that this does not say anything about unallocated cells. An
unallocated cell can contain any garbage, but then it is the job of
the allocator to clean it up before handing it out to the mutator. An
alternative formulation would be $\lnot\free{c} \iff
\left(\ldots\right)$, but this would then require the garbage
collector to unset the mark flag of any cell it frees, and as cells
need to be modified before being given to the mutator anyway,
confining this to the allocator seems reasonable.

As we'll be referring to the property of a cell being marked multiple
times, define a shorthand, $\marked{h}{c} \equiv
h\left[\gchead{c}\right] \neq 0$.

\section{Sweeping}
\label{sec:marksweep-sweeping}

This section shall consider sweeping, with an assumption that correct
marking has already been performed. Then, we shall briefly consider
interleaved marking and sweeping, such as happens in some incremental
and other special-case collectors.

What is meant when it is said that a sweeper is correct? After a
little thought, we will realise that we mean it doesn't deallocate any
cells which were marked, and it unmarks all of the cells. In the case
of a correct marker, this is equivalent to preserving and unmarking
exactly the reachable cells, which (with the additional constraint of
word preservation) is exactly what we want our collector to do.

\begin{definition}[Correct Sweeping]
  \label{def:ms-correct-sweeping}
  All cells which are marked are preserved and unmarked, no garbage
  cells are preserved, and nothing other than the garbage collector
  metadata gets mutated.
  \[\forall c \in h',\ \left(\lnot\free{c} \iff c \in
    \markset{h'}{roots}\right) \land \left(\lnot\free{c} \implies
    \lnot\marked{h}{c}\right)\]
\end{definition}

The essential component here is the $\iff$, the effect of which is to
identify the set of allocated cells with the set of cells which were
marked.

\subsection{Interleaved Marking and Sweeping}
\label{sec:marksweep-sweeping-interleaved}

Hitherto it has been assumed that marking happens in its entirety, and
then sweeping happens in its entirety. However, sometimes marking and
sweeping are interleaved, such that in each iteration of the collector
a little more of each is done.

In order for correctness to hold here, we can conceptually separate it
out into a marker and a sweeper, but rather than talking about the
entire heap, we talk about the subheap considered so far. Thus, the
correctness of each remains the same, but we have an additional proof
obligation.

\begin{definition}[Interleaved Marking/Sweeping Correctness]
  \label{def:ms-interleaved}
  When the sweeper considers a cell, the marker has already marked it
  and everything it points to (if it was reachable).
\end{definition}

Thus, the marker always remains ahead of the sweeper, and so no
problems arise. In fact, we could consider normal mark-sweep
collectors a special case of this type, where this property trivially
holds.

\section{Case Study: A Garbage Collector for Erlang}
\label{sec:marksweep-example}

Armstrong and Virding\cite{Armstrong95} introduced a simple mark-sweep
collector for Erlang, making use of the immutability of the language
in order to combine the mark and sweep stages. The collector operates
on a heap of cons cells, and uses a pointer named \texttt{SCAV} to
keep track of the current position of the collector in the heap. The
algorithm is presented in Figure \ref{fig:marksweep-example-algo}.

\begin{figure}[t]
  \centering
  \begin{lstlisting}
last = current
SCAV = hist(last)
while (SCAV != first) {
    if (marked(SCAV)) {
        possibly_mark(car(SCAV));
        possibly_mark(cdr(SCAV));
        unmark(SCAV);
        last = SCAV;
        SCAV = hist(last);
    } else {
        tmp = SCAV;
        SCAV = hist(SCAV);
        set_history(last, SCAV);
        free_cons(tmp);
    }
}
  \end{lstlisting}
  \captionsetup{format=default}
  \caption{The Armstrong/Virding garbage collector.}
  \label{fig:marksweep-example-algo}
\end{figure}

The \texttt{possibly\_mark}\footnote{Notational aside: variable and
  function names lifted directly from the algorithms are typeset in
  \texttt{typewriter font}, even in proofs, to distinguish them.}
function follows and marks its argument if it is a pointer;
\texttt{first} and \texttt{current} point to the first allocated cell
and the head of the free list, respectively; and the \texttt{hist}
function returns a pointer to the cell allocated before its
argument. It is assumed that all cells pointed to by roots have been
marked before calling the collector.

This collector works because the history fields form a linked list of
allocated cells, going back to the beginning of time. As the language
is immutable, pointers must always point back in time, and so if a
cell is unmarked by the time it is reached by following the history
list, then it must be garbage. We can express this property as a loop
invariant: the portion of the heap which has been considered consists
only of unmodified, reachable cells, and all cells which are directly
reachable from a cell in the considered region are marked.

\subsubsection{Partial Correctness}
\label{sec:marksweep-example-partial}

Let $\id~x$ be the \textit{allocation ID} of a cell $x$. The first
cell has an ID of zero, and the ID of every other cell is one greater
than the ID of the previously allocated cell: we can then express the
loop invariant as in Figure
\ref{fig:marksweep-example-partial-invariant}.

\begin{figure}[t]
  \centering
  \begin{align*}
    \forall c \in h',\ \forall w \in c,\ &first \in \reach{h'}{roots}\\
%
    &\land \id~c > \id~\mathtt{SCAV} \implies (\free{c}
      \iff c \notin \reach{h'}{roots})\\
%
    &\land \id~c > \id~\mathtt{SCAV} \land \lnot\free{c}
      \implies h[w] = h'[w] \land \lnot \marked{h}{c}\\
%
    &\land \id~c > \id~\mathtt{SCAV} \land \lnot\free{c}
      \implies (\forall c \pointsto x,\ \id~x < \id~\mathtt{SCAV}
      \implies \marked{h}{x})\\
%
    &\land \left(\id~c < \id~\mathtt{SCAV} \land \nexists x \pointsto
      c,\ \id~x > \id~\mathtt{SCAV}\right) \implies \lnot\marked{h}{c}
  \end{align*}  
  \caption{Loop invariant for the Armstrong/Virding collector}
  \label{fig:marksweep-example-partial-invariant}
\end{figure}

As the collector never touches the \texttt{first} cell, in order for
there to be no garbage left over at the end, then it must necessarily
be reachable. This is somewhat unusual, and arises purely because this
collector arguably has a bug, in that this one cell of garbage can
survive collection.

Let's begin by rewriting the algorithm in terms of arrays, and expand
out all of the functions within it. Furthermore, let's say that a cell
in this scheme consists of two words of metadata and two of data,
arranged in the order $\langle mark,\ history,\ car,\ cdr
\rangle$. The rewritten algorithm is displayed in Figure
\ref{fig:marksweep-example-partial-arrays}.

\begin{figure}[t]
  \centering
  \begin{lstlisting}
last = current
SCAV = h[gchead(last) + 1]
while (SCAV != first) {
    if (marked(h, SCAV)) {
        car = h[SCAV];
        cdr = h[SCAV + 1];

        if(type(car) == pointer) {
            h[gchead(car)] = 1
        } else {
            skip
        };

        if(type(cdr) == pointer) {
            h[gchead(cdr)] = 1
        } else {
            skip
        };

        h[gchead(SCAV)] = 0;

        last = SCAV;
        SCAV = h[gchead(last) + 1];
    } else {
        tmp = SCAV;
        SCAV = h[gchead(SCAV) + 1];
        h[gchead(last) + 1] = SCAV;
        free_cons(tmp);
    }
}
  \end{lstlisting}
  \caption{Array-based version of the Armstrong/Virding collector}
  \label{fig:marksweep-example-partial-arrays}
\end{figure}

Combining the correct marking, correct sweeping, and word preservation
requirements, we arrive at the following postcondition for a correct
garbage collector: \[\forall c \in h',\ \forall w \in c,\
(\free{c} \iff c \notin \reach{h'}{roots}) \land (\lnot\free{c} \implies
h[w] = h'[w] \land \lnot\marked{h}{c})\]

\begin{lemma}[Loop Invariant]
  \begin{figure}[t]
    \centering
    \begin{align*}
      \forall c \in h',\ \forall w \in c,\ &\mathtt{SCAV} = first \land
        first \in \reach{h'}{roots}\\
%
      &\land \id~c > \id~\mathtt{SCAV} \implies (\free{c}
        \iff c \notin \reach{h'}{roots})\\
%
      &\land \id~c > \id~\mathtt{SCAV} \land \lnot\free{c}
        \implies h[w] = h'[w] \land \lnot \marked{h}{c}\\
%
      &\land \id~c > \id~\mathtt{SCAV} \land \lnot\free{c}
        \implies (\forall c \pointsto x,\ \id~x < \id~\mathtt{SCAV}
        \implies \marked{h}{x})\\
%
      &\land \left(\id~c < \id~\mathtt{SCAV} \land \nexists x \pointsto
        c,\ \id~x > \id~\mathtt{SCAV}\right) \implies
        \lnot\marked{h}{c}\\\\
%
%
      \forall c \in h',\ \forall w \in c,\ & \id~c > \id~first \implies
        (\free{c} \iff c \notin \reach{h'}{roots})\\
%
      &\land \id~c > \id~first \land \lnot\free{c}
        \implies h[w] = h'[w] \land \lnot \marked{h}{c}\\
%
      &\land \id~c > \id~first \land \lnot\free{c}
        \implies (\forall c \pointsto x,\ \id~x < \id~first
        \implies \marked{h}{x})\\
%
      &\land \left(\id~c < \id~first \land \nexists x \pointsto
        c,\ \id~x > \id~first\right) \implies
        \lnot\marked{h}{c}\\\\
%
%
      \forall c \in h',\ \forall w \in c,\ & \free{c} \iff c \notin
        \reach{h'}{roots}\\
%
      &\land \lnot\free{c} \implies h[w] = h'[w] \land \lnot \marked{h}{c}\\
%
      &\land \lnot\free{c} \implies (\forall c \pointsto x,\
        \lfalse \implies \marked{h}{x})\\
%
      &\land \left(\lfalse \land \nexists x \pointsto
        c,\ \id~x > \id~first\right) \implies
        \lnot\marked{h}{c}\\\\
%
%
      \forall c \in h',\ \forall w \in c,\ & (\free{c} \iff c \notin
        \reach{h'}{roots}) \land (\lnot\free{c} \implies h[w] = h'[w] \land
        \lnot \marked{h}{c})
    \end{align*}
    \caption{Sufficiency of the loop invariant for the Armstrong/Virding collector}
    \label{fig:marksweep-example-partial-sufficient}
  \end{figure}

  The sufficiency of the loop invariant is shown in Figure
  \ref{fig:marksweep-example-partial-sufficient}, and now it remains
  to be shown that the invariant holds,

  \begin{prooftree}
    \AxiomC{$\htriple{I \land C \land M}{\ldots}{I}$}
    \AxiomC{$\htriple{I \land C \land \lnot M}{\ldots}{I}$}
    \BinaryInfC{$\htriple{I \land C}{if(marked(h, SCAV)) \ldots}{I}$}
    \UnaryInfC{$\htriple{I}{while \ldots}{I \land \lnot C}$}
  \end{prooftree}
  where $C = \mathtt{SCAV} \neq first$ and $M =
  \marked{h}{\mathtt{SCAV}}$.

  The proof naturally separates into two cases, one for each branch of
  the if statement. These are proven separately in Lemmas
  \ref{lem:lia} and \ref{lem:lib}.
  \label{lem:li}
\end{lemma}

\begin{lemma}[Loop Invariant (first branch)]
  Knowing that nothing in this branch deallocates cells, the proof
  goals can immediately be simplified to these three conditions:
  \begin{enumerate}
    \item $\mathtt{SCAV} \in \reach{h'}{roots}$
    \item $\forall w \in \mathtt{SCAV},\ h[w] = h'[w] \land
      \lnot\marked{h}{\mathtt{SCAV}}$
    \item $\forall \mathtt{SCAV} \pointsto x,\ \marked{h}{x}$
  \end{enumerate}
  Firstly, let's look at the two expansions of
  \texttt{possibly\_mark}, as they are almost identical. Let's refer
  to the variable holding the possible pointer as $x$, and let $X
  \equiv type(x) = pointer$.
  \begin{prooftree}
    \AxiomC{$\htriple{X \land P}{h[gchead(x)] = 1}{Q}$}

    \AxiomC{$\lnot X \land P \implies Q$}
    \UnaryInfC{$\htriple{\lnot X \land
        P}{skip}{Q}$}
    \BinaryInfC{$\htriple{P}{if(type(x) == pointer) \{ \ldots \} else \{
          skip \}}{Q}$}
  \end{prooftree}
  The desired postcondition, of course, is \[(type(x) = pointer \land
  \id~x < \id~SCAV) \implies \marked{h}{x}\]

  Fortunately, we know that if $x$ is a pointer, then the allocation
  ID of its cell must be lower than that of the current cell because
  of the constraint over the entire system: data is immutable. Thus,
  only backward pointers are possible, and so we can simplify the
  postcondition to just \[type(x) = pointer \implies \marked{h}{x}\]

  If $x$ is a pointer, then the cell it points to gets marked. If the
  gaps in the proof tree are filled in as follows, then this can be
  achieved:
  \begin{align*}
    P &\equiv type(x) = pointer \implies \marked{(h; \gchead{x}:1)}{x}\\
    Q &\equiv type(x) = pointer \implies \marked{h}{x}
  \end{align*}
  Now it must just be shown that the assignment works:
  \begin{align*}
    (X \land P)[h/(h; \gchead{x}:1)] &\iff (type(x) = pointer\\
    &\quad\land (type(x) = pointer \implies \marked{(h;
      \gchead{x}:1)}{x}))\\
    &\quad[h/(h; \gchead{x}:1)]\\
%
    &\iff type(x) = pointer \land (type(x) = pointer \implies
    \marked{h}{x})\\
%
    &\implies Q
  \end{align*}
  It can now be seen that by simply combining lines 5 to 18, we end up
  in the state, \[(type(car) = pointer \implies \marked{h}{car}) \land
  (type(cdr) = pointer \implies \marked{h}{cdr})\] which is exactly
  what is wanted. Part (3) of the invariant has now been handled. This
  result can now be inductively applied, combined with the
  precondition of this entire block, $\marked{h}{\mathtt{SCAV}}$ to
  trivially see that part (1) holds. Finally, it can be seen that we
  never assign to anything other than the mark flags, and the mark
  flag of \texttt{SCAV} is unset on line 20, and so part (2) holds.
  \label{lem:lia}
\end{lemma}

\begin{lemma}[Loop Invariant (second branch)]
  Informally, as we have assumed correct marking, we know that this
  cell is not reachable from the roots. The invariant is broken on
  line 26, as an unreachable cell is left in the list ahead of
  \texttt{SCAV}, but then it is restored in lines 27 and 28, as the
  cell is removed and deallocated. Thus, the invariant still holds.
  \label{lem:lib}
\end{lemma}

\begin{proof}
  Knowing that the loop invariant holds and is sufficient for
  correctness, as established by Lemma \ref{lem:li}, it only remains
  to be shown that it holds at the start of the loop.

  \begin{prooftree}
    \AxiomC{$\htriple{I''}{last = current}{I'}$}
    \AxiomC{$\htriple{I'}{\texttt{SCAV} = h[gchead(last) + 1]}{I}$}
    \BinaryInfC{$\htriple{I''}{last = current; \texttt{SCAV} = h[gchead(last) + 1]}{I}$}
  \end{prooftree}

  Furthermore, as $h$ refers to the current heap and $h'$ to the
  original heap (being a ghost variable), and as the heap has not been
  mutated yet, $h = h'$, which allows further simplifications to be
  made.


  \begin{align*}
    I' &= I[h[\gchead{last} + 1] / \mathtt{SCAV}]\\
    &=\forall c \in h,\ \forall w \in c,\\
    &\quad\quad first \in \reach{h}{roots}\\
    &\quad\quad \land \id~c > \id~h[\gchead{last} + 1] \implies
    (\free{c} \iff c \notin \reach{h}{roots})\\
    &\quad\quad \land \id~c > \id~h[\gchead{last} + 1] \land
    \lnot\free{c} \implies h[w] = h[w] \land \lnot \marked{h}{c}\\
    &\quad\quad \land \id~c > \id~h[\gchead{last} + 1] \land
    \lnot\free{c} \implies (\forall c \pointsto x,\ \id~x <
    \id~h[\gchead{last} + 1]\\&\quad\quad\quad\quad\implies \marked{h}{x})\\
    &\quad\quad \land \left(\id~c < \id~h[\gchead{last} + 1] \land
    \nexists x \pointsto c,\ \id~x > \id~h[\gchead{last} + 1]\right)
    \implies \lnot\marked{h}{c}\\\\
%
    I'' &= I'[current / last]\\
    &=\forall c \in h,\ \forall w \in c,\\
    &\quad\quad first \in \reach{h}{roots}\\
    &\quad\quad \land \id~c > \id~h[\gchead{current} + 1] \implies
    (\free{c} \iff c \notin \reach{h}{roots})\\
    &\quad\quad \land \id~c > \id~h[\gchead{current} + 1] \land
    \lnot\free{c} \implies h[w] = h[w] \land \lnot \marked{h}{c}\\
    &\quad\quad \land \id~c > \id~h[\gchead{current} + 1] \land
    \lnot\free{c} \implies (\forall c \pointsto x,\ \id~x <
    \id~h[\gchead{current} + 1]\\&\quad\quad\quad\quad\implies \marked{h}{x})\\
    &\quad\quad \land \left(\id~c < \id~h[\gchead{current} + 1] \land
    \nexists x \pointsto c,\ \id~x > \id~h[\gchead{current} + 1]\right)
    \\&\quad\quad\quad\quad\implies \lnot\marked{h}{c}\\
%
    &=\forall c \in h,\ \forall w \in c,\\
    &\quad\quad first \in \reach{h}{roots}\\
    &\quad\quad \land \lfalse \implies (\free{c} \iff c \notin
    \reach{h}{roots})\\
    &\quad\quad \land \lfalse \land \lnot\free{c} \implies
    \ltrue \land \lnot \marked{h}{c}\\
    &\quad\quad \land \lfalse \land \lnot\free{c} \implies
    (\forall c \pointsto x,\ \id~x < \id~h[\gchead{current} + 1]
    \implies \marked{h}{x})\\
    &\quad\quad \land \left(\ltrue \land \nexists x \pointsto
    c,\ \lfalse\right) \implies \lnot\marked{h}{c}\\
%
    &= \forall c \in h,\ first \in \reach{h}{r} \land (\lnot\free{c}
    \implies \lnot \marked{h}{c})
  \end{align*}

  Thus, we can see that the algorithm is correct if, when it starts,
  no cells are marked. As no cells are marked after it completes, this
  becomes a constraint on the allocator, to unmark recycled cells
  before giving them to the mutator. It can also be seen from the
  precondition that the collector has an overhead of one cell: the
  \texttt{first} cell is never collected, and so correctness does not
  strictly hold if \texttt{first} is not reachable, but the collector
  is close enough such that it doesn't really matter.
\end{proof}

\subsubsection{Total Correctness}
\label{sec:marksweep-example-total}

We can use allocation IDs to express that the history list forms an
unbroken chain of cells, all the way back to the first cell, as
follows:

\[\forall x,\ \left(x = \mathrm{first} \implies x =
  \mathrm{hist}(x)\right) \land \left(x \neq \mathrm{first} \implies
  \exists n \in \mathbb N_{1},\ \id~x = n +
  \id~\mathrm{hist}(x)\right)\]

The loop condition can be restated as $\id~\mathtt{SCAV} \neq 0$. As
there are no IDs below 0 ($\mathrm{hist}(\mathrm{first}) =
\mathrm{first}$), that can be further rewritten to $\id~\mathtt{SCAV}
> 0$. \texttt{SCAV} starts out as the last-allocated cell, and so has
an ID $\geq 0$. Now, in order to prove termination, we simply need to
show that $\id~\mathtt{SCAV}$ decreases at every iteration of the
loop.

To simplify, let us throw away everything which does not relate to
mutating \texttt{SCAV}, giving the following loop:

\begin{lstlisting}
while (SCAV != first) {
    if (marked(SCAV)) {
        last = SCAV;
        SCAV = hist(last);
    } else {
        SCAV = hist(SCAV);
    }
}
\end{lstlisting}

Substituting variables, we get

\begin{lstlisting}
while (SCAV != first) {
    if (marked(SCAV)) {
        SCAV = hist(SCAV);
    } else {
        SCAV = hist(SCAV);
    }
}
\end{lstlisting}

And then we can eliminate the if statement:

\begin{lstlisting}
while (SCAV != first) {
    SCAV = hist(SCAV);
}
\end{lstlisting}

Trivially, we can now see that there are two cases:

\begin{description}
  \item[Case 1, $\mathtt{SCAV} = \mathrm{first}$] In this case,
    $\mathrm{hist}(\mathtt{SCAV}) = \mathtt{SCAV}$, and so
    $\id~\mathtt{SCAV}$ does not decrease, but this situation is
    precisely the loop termination condition.

  \item[Case 2, $\mathtt{SCAV} \neq \mathrm{first}$] In this case, we
    know by the history list assumption that
    $\id~\mathrm{hist}(\mathtt{SCAV}) < \id~\mathtt{SCAV}$, and so
    $\id~\mathtt{SCAV}$ does decrease. Furthermore, we know that the
    ID cannot decrease below zero, and so we get a strictly decreasing
    sequence of positive natural numbers. Clearly, this sequence is
    not infinite, and so the loop terminates.
\end{description}

\section{Summary}
\label{sec:marksweep-summary}

This chapter has defined the heap, cells, pointers, word preservation,
and reachability (defns.  \ref{def:ms-heap}, \ref{def:ms-cell},
\ref{def:ms-pointer}, \ref{def:ms-word-preservation}, and
\ref{def:ms-reachable}), and used these to then define correct marking
and sweeping (defns.  \ref{def:ms-correct-marking} and
\ref{def:ms-correct-sweeping}), giving some thought to the correctness
of interleaved marking and sweeping (defn. \ref{def:ms-interleaved}).

It has been argued that correctness for a mark-sweep collector comes
down to marking all (and only) of the reachable cells, freeing all
others, and then unmarking. And, furthermore, that no words should be
modified in non-garbage cells after garbage collection.

Then the Armstrong/Virding\cite{Armstrong95} collector was used as an
example of the developed formalism, with proofs of its partial and
total correctness provided. These proofs shall not be referenced in
future chapters, and merely serve as an illustration of the utility
of the definition of correctness presented here. Definitions, however,
will be reused.

\chapter{Correctness of Copying Collectors}
\label{sec:copying}

We shall use the same definitions of the \gls{heap}
(defn. \ref{def:ms-heap}), \gls{cell} (defn. \ref{def:ms-cell}),
\gls{pointer} (defn. \ref{def:ms-pointer}), and reachability
(defn. \ref{def:ms-reachable}) as in the prior chapter. Unfortunately,
the definition of word preservation
(defn. \ref{def:ms-word-preservation}) will have to change, because a
\gls{copying} \gls{collector} renames addresses.

\section{Copying}
\label{sec:copying-copying}

Before presenting a definition of correct copying, I have to highlight
the inadequacies of the \gls{mark-sweep} word preservation. Firstly,
it does not allow for mutation of \glspl{pointer}, which is essential
for a \gls{copying} \gls{collector}, but more subtly, it does not
allow for \glspl{cell} to be relocated. This arises in the $h[w] =
h'[w]$ part of the formalism. If \glspl{cell} can be relocated, then
this makes no sense, as $w$ is a \gls{pointer}, and so $w$ in the new
\gls{heap} is not necessarily the same as $w$ in the old
\gls{heap}. Thus, we also need to be able to find old $w$ values.

\begin{definition}[Correct Copying]
  \label{def:c-correct-copying}
  After garbage collection, no allocated cells have been mutated,
  except in the pointer fields, in which case we apply an address
  translation function $f$.

  \begin{align*}
    \forall c \in h,\ \forall w \in c,\ \alloc{c} & \implies
    \mathrm{type}(w) = \mathrm{pointer} \iff h[w] = f(h'[f^{-1}(w)])\\
    &\quad\land \mathrm{type}(w) \neq \mathrm{pointer} \iff h[w] =
    h'[f^{-1}(w)]
  \end{align*}
\end{definition}

I have decided to define correct copying as a word preservation-like
property, rather than define it separately, as what is ``correct
copying'' but a way of correctly transforming the contents and
locations of cells, according to some rule?

\section{Address Translation}
\label{sec:copying-address}

\todo{Proof obligations for address translation}

\section{Case Study: A Garbage Collector for Lisp}
\label{sec:copying-example}

\todo{Slightly informal proof for Fenichel/Yochelson 
collector. Informality is ok, because this is only to show the ideas
are in the right direction.}

\chapter{Partial Correctness for Garbage Collection}

\todo{Outline of partial correctness}

\section{Comparison of Mark-Sweep and Copying Correctness}

\todo{Comparison of approaches}

\subsection{Stages of Garbage Collection}

\todo{Stages all garbage collectors do (perhaps implicitly)}

\section{A Framework for Proof}

\todo{Somewhat more concrete proof obligations which, together, imply
  partial correctness.}

\subsection{Sufficiency of the Framework}

\todo{Proof that the framework works}

\section{Less Strict Collectors}

\todo{Discussion of assumptions (stop-the-world)}

\subsection{Incremental Garbage Collection}

\todo{Transformation to stop-the-world and proof. Remaining out of
  scope proof obligation (all valid interleavings work)}

\subsection{Generational Garbage Collection}

\todo{Generations are subheaps for minor-collection
  correctness. Normal proof for major-collection correctness.}

\chapter{Results \& Evaluation}
\label{sec:results}

Previously, I have attempted to provide proofs of correctness for two
extant garbage collection algorithms, and from this extract a more
generalised notion of correctness. In this chapter, I shall discuss
and evaluate the results of the two facets of my work, and how
successful I have been in achieving the goal of this project. I shall
also discuss an experimental validation of my proofs involving an
implementation of the two collectors discussed.

\section{Proofs of Collectors}
\label{sec:results-collectors}

In the process of producing a generalised notion of correctness for
garbage collectors, I first provided proofs of the Armstrong/Virding
and Fenichel/Yochelson collectors: two very different approaches, but
both for systems based on fixed-size cons cells. The proofs for
neither collector I considered could be machine verified in their
current form, however I believe that they remain convincing despite
this.

I hold this to be the case because the collectors were chosen for
their simplicity: they are both old, established, collectors which
have stood the test of time, and they are sufficiently simple that one
can `see' their correctness at a glance. They do not attempt to
perform any nontrivial cleverness to improve the performance of the
mutator or memory system, as more modern hybrid collectors are wont to
do, and they fall strictly within the bounds of mark-sweep and copying
collectors, without borrowing anything from other areas. Thus, the
proofs simply make explicit the intuition we feel when we read them,
and the implementation of the collectors provides an experimental
validation of this intuition.

The proofs may appear quite different on the surface, however many of
the definitions used in the mark-sweep proof were reused in the
copying proof, and those that were changed were made more
general. The proofs may not be so different after all,
then. Naturally, there are some parts which are collector-specific
(such as correct marking, sweeping, and copying), but these arose from
trying to convert the very abstract postcondition into something more
amenable to proof.

\section{Collector Implementation}
\label{sec:results-impl}

As evidence of the correctness of the proofs, I
implemented\footnote{All of the code produced can be found in Appendix
  \ref{sec:gc-impl}, and online at
  \url{https://github.com/barrucadu/meng-project}.} both of the
collectors in C, and wrote a small test program with a very
constrained heap which constructs a variety of linked lists, and
results in over 200 garbage collections occurring. I then attached
assertions to points in the programs to verify the preconditions,
postconditions, and invariants of the collectors.

The implementations are written in C99, and have been tested with both
clang 3.4 and gcc 4.8.2. They should work under any
standards-compliant compiler, but I have not verified this.

The implementation does have some limitations imposed upon it for
simplicity: there is a fixed-size array of roots, which the programmer
must ensure contains all roots, and the requirement of immutable cells
in the Armstrong/Virding collector is not implemented. Resolving both
of these would have been significantly more work, and are not
necessary in order to demonstrate that the garbage collection works.

Both collectors manage a heap of \texttt{NUM\_CELLS} cells (in the
case of Fenichel/Yochelson, both semispaces are this big), in order to
have identical characteristics. This then allows identical assertions
to be used in the test program. As neither collector comes with an
allocator, I have had to produce allocators, \texttt{alloc()}, which
work with them. The way I have done so is not necessarily the only, or
best, way, it is just what occurred to me and appears to work. In both
cases, the collector is invoked by the allocator upon not having
enough memory left for an allocation. The preconditions are checked
before calling the \texttt{gc()} function, and the postconditions
checked after. In each of the garbage collector loops, the invariants
are checked after each iteration.

Due to using a low level language, the implementations of the
collectors look somewhat more complex than what is specified in their
respective papers, but is as close as I could reasonably obtain.

The assertions, and their implementation are discussed below.

\subsection{Armstrong/Virding}
\label{sec:results-impl-ms}

\subsubsection{Preconditions}
\label{sec:results-impl-ms-pre}

The Armstrong/Virding collector, as I discussed, has a precondition
that the $first$ cell is always reachable, as otherwise it would be
garbage, and due to it never being freed, garbage would survive a
collection. Furthermore, it is assumed that when the collector is
called, all cells pointed to by roots have been marked.

\begin{lstlisting}[language=C,caption={Armstrong/Virding Preconditions}]
assert(reachable(gchead(first)));

for(unsigned int i = 0; i < NUM_ROOTS; i++)
  if(roots[i] != NULL)
    gchead(roots[i])->mark = MARKED;
\end{lstlisting}

\subsubsection{Postconditions}
\label{sec:results-impl-ms-post}

The postcondition express simply the correct mark-sweep postcondition:
after collection, all cells which are allocated (not on the free list)
are reachable from the roots. Furthermore, the live cell invariant is
checked here as a postcondition, although arguably it should be
checked with the rest of the invariants, in the garbage collection
loop.

\begin{lstlisting}[language=C,caption={Armstrong/Virding Postconditions}]
for(unsigned int i = 0; i < NUM_CELLS; i++)
  if(reachable(&heap[i]))
    assert(!on_free_list(&heap[i]));

for(unsigned int i = 0; i < NUM_CELLS; i++)
  if(!on_free_list(&heap[i]))
    assert(reachable(&heap[i]));
\end{lstlisting}

\subsubsection{Invariants}
\label{sec:results-impl-ms-invariants}

The invariants appear somewhat more complex. The first loop expresses
$\forall c \in h,\ \alloc{c} \land \id~c > \id~\mathtt{SCAV} \implies
(\forall c \pointsto x, \id~x < \id~\mathtt{SCAV} \implies
\marked{h}{c})$: all cells outside of the region that has been
collected, but which are pointed to by a cell in that region, have
been marked. The latter two express the simpler properties that all
allocated cells in the collected region are live, and all allocated
cells in the collected region are unmarked.

\begin{lstlisting}[language=C,caption={Armstrong/Virding Invariants}]
for(unsigned int i = 0; i < NUM_CELLS; i++)
  if(reachable(&heap[i]) && cell_id(&heap[i]) > cell_id(gchead(SCAV)))
    {
      if(heap[i].cell.car.tag == REFERENCE &&
         heap[i].cell.car.val.ptr != NULL &&
         cell_id(gchead(heap[i].cell.car.val.ptr)) < cell_id(gchead(SCAV)))
        assert(gchead(heap[i].cell.car.val.ptr)->mark == MARKED);

      if(heap[i].cell.cdr.tag == REFERENCE &&
         heap[i].cell.cdr.val.ptr != NULL &&
         cell_id(gchead(heap[i].cell.cdr.val.ptr)) < cell_id(gchead(SCAV)))
        assert(gchead(heap[i].cell.cdr.val.ptr)->mark == MARKED);
    }

for(unsigned int i = 0; i < NUM_CELLS; i++)
  if(reachable(&heap[i]) && cell_id(&heap[i]) >= cell_id(gchead(SCAV)))
    assert(reachable(&heap[i]));

for(unsigned int i = 0; i < NUM_CELLS; i++)
  if(reachable(&heap[i]) && cell_id(&heap[i]) > cell_id(gchead(SCAV)))
    assert(heap[i].mark == UNMARKED);
\end{lstlisting}

\subsection{Fenichel/Yochelson}
\label{sec:results-impl-c}

\subsubsection{Preconditions}
\label{sec:results-impl-c-pre}

An invariant which I did not mention whilst discussing my proof, but
which the garbage collector, mutator, and allocator must obey, is that
there are no inter-semispace pointers. Due to a bug in my linked list
code (which has since been remedied), a garbage collection occuring
whilst appending a value to a list would result in this invariant
being broken, and resulted in my being stuck for some time before I
realised what the issue was. I then decided to enforce this as a
precondition of the collector.

\begin{lstlisting}[language=C,caption={Fenichel/Yochelson Preconditions}]
cell *sspace = (cons_space == FIRST) ? semispace1 : semispace2;

for(unsigned int i = 0; i < NUM_CELLS; i++)
  if(reachable(&sspace[i]))
    {
      if(sspace[i].car.tag == REFERENCE &&
         sspace[i].car.val.ptr != NULL)
        assert(sspace[i].car.val.ptr >= sspace &&
               sspace[i].car.val.ptr <= &sspace[NUM_CELLS]);

      if(sspace[i].cdr.tag == REFERENCE &&
         sspace[i].cdr.val.ptr != NULL)
        assert(sspace[i].cdr.val.ptr >= sspace &&
               sspace[i].cdr.val.ptr <= &sspace[NUM_CELLS]);
    }
\end{lstlisting}

\subsubsection{Postconditions}
\label{sec:results-impl-c-post}

As a postcondition I again checked the live cell invariant, and also
asserted that all allocated cells are reachable.

\begin{lstlisting}[language=C,caption={Fenichel/Yochelson Postconditions}]
cell* sspace = (cons_space == FIRST) ? semispace1 : semispace2;

for(unsigned int i = 0; i < NUM_CELLS; i++)
  if(reachable(&sspace[i]))
    assert(!on_free_list(&sspace[i]));

for(unsigned int i = 0; i < NUM_CELLS; i++)
  if(!on_free_list(&sspace[i]))
    assert(reachable(&sspace[i]));
\end{lstlisting}

\subsubsection{Invariants}
\label{sec:results-impl-c-invariants}

The invariant I decided to check is that, after collecting every root,
all pointers in cells reachable from that root have been
updated. Actually, I went for the weaker property that pointers
pointed into the correct semispace. Unfortunately, proving the
stronger property would require having access to the pre-state of the
heap, which would require copying it before initiating a garbage
collection.

\begin{lstlisting}[language=C,caption={Fenichel/Yochelson GC Loop}]
for(unsigned int i = 0; i < NUM_ROOTS; i ++)
  if(roots[i] != NULL)
    {
      roots[i] = collect(roots[i]);
      check_pointers_updated(roots[i], NULL);
    }
\end{lstlisting}

\begin{lstlisting}[language=C,caption={Fenichel/Yochelson Pointer Checking}]
static void check_pointers_updated(const cell* root, const cell* cur)
{
  if(root == cur)
    return;

  if(cur == NULL)
    cur = root;

  cell *sspace = (cons_space == FIRST) ? semispace1 : semispace2;

  if(cur->car.tag == REFERENCE &&
     cur->car.val.ptr != ALREADYCOPIED &&
     cur->car.val.ptr != NULL)
    {
      assert(cur->car.val.ptr >= sspace &&
             cur->car.val.ptr <= &sspace[NUM_CELLS]);
      check_pointers_updated(root, cur->car.val.ptr);
    }

  if(cur->cdr.tag == REFERENCE &&
     cur->cdr.val.ptr != NULL)
    {
      assert(cur->cdr.val.ptr >= sspace &&
             cur->cdr.val.ptr <= &sspace[NUM_CELLS]);
      check_pointers_updated(root, cur->cdr.val.ptr);
    }
}
\end{lstlisting}

\section{Generalised Garbage Collection Correctness}
\label{sec:results-correctness}

\todo{Evaluation 2}

\section{Summary}
\label{sec:results-summary}

\todo{Key strengths and weaknesses of work}
\chapter{Conclusions, Related and Future Work}
\label{sec:conclusion}

This chapter discusses what has been achieved in the project in
Section \ref{sec:conclusion-review}, reviews the related work of
McCreight, Shao, Lin and Li\cite{McCreight07} in Section
\ref{sec:conclusion-related}, and then introduces some further work
suggested to me whilst undertaking this project in Section
\ref{sec:conclusion-further}. Finally, we close with some thoughts on
the project as a whole in Section \ref{sec:conclusion-thoughts}.

\section{A Summary of What Has Been Achieved}
\label{sec:conclusion-review}

Chapter \ref{sec:marksweep} produced precise formal definitions of
many common structures used by garbage collectors from first
principles, allowing reasoning about garbage collection at all to be
possible, then used these to formalise mark-sweep correctness and
prove correct the Armstrong/Virding collector. As this is quite a
simple collector, there are undoubtedly other proofs of its
correctness somewhere, however a non-verified proof is only as
reliable as the collection of people who have read it, and so each new
proof adds further evidence that it is correct.

Chapter \ref{sec:copying} extended the definitions produced in Chapter
\ref{sec:marksweep} to allow reasoning about copying collection, and
then verified the Fenichel/Yochelson collector. The formalism was then
compared with that of Myreen\cite{Myreen10} and shown to be very
similar, providing confidence that Myreen's formalism is a reasonable
one for verifying copying collectors, even though it was developed in
a very different way.

Chapter \ref{sec:gc} further extended the results of Chapter
\ref{sec:copying}, resulting in a natural progression from the more
specific to the more general, and gives a new formalism which allows
reasoning about different types of collectors, as shown in Section
\ref{sec:results-correctness}.

To conclude, the project contributes an independently-produced
standard for garbage collection correctness, which has been used,
indirectly, to verify two extant garbage collectors: the
Armstrong/Virding mark-sweep collector for languages with immutable
state, and the Fenichel/Yochelson copying collector for LISP systems.

\section{Related Work}
\label{sec:conclusion-related}

In 2007, McCreight, Shao, Lin, and Li produced their paper, \textit{A
  General Framework for Certifying Garbage Collectors and Their
  Mutators}\cite{McCreight07}, which models garbage collection
correctness by defining an interface between the mutator and the
collector, where the mutator accesses an abstract heap, and the
collector a concrete heap. The two are related by an invariant, and
the garbage collector is defined as a step relation, and by
parameterising the mutator proof by these two predicates, a mutator
may be verified independently of a collector. This is machine
verified and proofs for three separate collectors have been
produced.

My work was produced independently of this, and by the time I found
the paper the majority of the work in Chapters \ref{sec:marksweep},
\ref{sec:copying}, and \ref{sec:gc} was done. The key difference is in
separating the heaps as seen by the mutator and collector, allowing
independent verification. My approach does not do this, and so
verification of a mutator necessarily is collector-specific, as
different collectors will have different invariants and preconditions.

\section{Further Work}
\label{sec:conclusion-further}

\subsection{Machine Verification}
\label{sec:conclusion-further-machineverification}

Whilst it would have been desirable to produce machine-verified
proofs, this was not done for two reasons: firstly, learning to use a
proof assistant would have been a significant amount of work to add to
the project, and secondly I believe the proofs are good enough in
their current form. Parts of the proofs may be amenable to automation,
however, which would have simplified the work whilst simultaneously
making it more rigorous.

Given the two proofs in the current forms, however, it would not be
infeasible to translate them into a form suitable for machine
verification, and to fill in the blanks, resulting in completely
machine-verified proofs. This endeavour would also highlight any
assumptions unknowingly made in this project, and bring to light any
further obligations, particularly those on the mutator or allocator.

\subsection{Run-time Invariant Checking}
\label{sec:conclusion-further-invariants}

The garbage collector is only a part of a larger system, consisting of
the allocator, mutator, and a possible intermediary memory-management
subsystem. Proving the garbage collector correct is all well and good,
but if the other components break the invariants or preconditions upon
which the garbage collector relies, it may fail, no matter how
rigorous the proof of correctness.

Unfortunately, the task of verifying a nontrivial system is very
difficult, and this is the reason why formal verification is so rare
in practice. Thus, it may be worthwhile to implement run-time checking
of invariants into the system, and to either halt or emit an error
when such an invariant is broken.

If this could be implemented efficiently, it would only be a small
overhead for increased reliability, as garbage collection bugs can not
arise in a verified collector where all of the expectations
hold. Furthermore, if this could be implemented such that invariants
are checked after every operation which could break them, it would
produce a powerful debugging tool, as a programmer would know exactly
where errors were introduced into the system.

\subsection{Incremental Garbage Collectors}
\label{sec:conclusion-further-incremental}

It has been assumed that all garbage collectors are of the
stop-the-world type, but this is often not the case in
practice. Incremental garbage collectors are very common, especially
in systems with need for rapid response to input, and my formalism
cannot reason about them at all.

It would be desirable to extend the formalism to be able to cope with
incremental collectors, perhaps by modelling the system as a
mutate/collect loop, and determining what invariants are needed for
the collector to not interfere with the mutator and vice versa, whilst
still removing all garbage from the system.

Furthermore, as an incremental collector does not remove all garbage
from the system every time it completes a cycle (as the mutator may
have produced some more in the intervening time), a more flexible
notion of removing all garbage would be required. Possibly something
around the idea that, were the mutator to cease producing any garbage
for some period of time, the collector would remove it all. If this
time could be bounded, then this would also allow more accurate
reasoning about the space characteristics of programs with incremental
garbage collectors.

\subsection{Concurrent Garbage Collectors}
\label{sec:conclusion-further-concurrent}

It is increasingly the case that the garbage collector and mutator run
in separate threads, giving very low delays on removing garbage from
memory. It may even be the case that the collector and mutator run on
separate processor cores continuously, which results in this not just
being a special case of incremental collection.

As it was assumed that collectors are of the stop-the-world type,
there is no scope for reasoning about this in the current
formalism. In order to show correctness, it would be necessary to show
that, for all possible interleavings, the collector and mutator could
not interfere with each other. Whilst with incremental collection we
simply need to show this holds for a fixed mutate/collect cycle, here
it is much more difficult, as mutation can occur simultaneously with
garbage collection.

However, this would be a very difficult problem to approach in
general, and could possibly be attacked by making the problem more
similar to incremental collection by constraining where collection and
mutation occurs, allowing a large class of dangerous states to be
avoided at the cost of some concurrency.

\section{Final Thoughts}
\label{sec:conclusion-thoughts}

This project has achieved its goal, and whilst the motivation (that
there be very little work on generic verification of garbage
collectors) proved to be untrue, the approach developed here is
different from the one discovered and is, I believe, more amenable to
hand-verification of collectors.


\begin{appendices}
\begin{lscape2col}
  \chapter{Checked Garbage Collector Implementations}
\label{sec:gc-impl}

\lstset{language=C}

\begin{wide}
  \section{Armstrong/Virding}
  \label{sec:gc-impl-armstrong-virding}

  \lstset{caption={Implementation of Armstrong/Virding in C}}
  \lstset{label=lst:armstrong-virding-c}
  \lstinputlisting{../code/armstrong-virding.c}

  \lstset{caption={Header file for the Armstrong/Virding C implementation}}
  \lstset{label=lst:armstrong-virding-h}
  \lstinputlisting{../code/armstrong-virding.h}

  \section{Fenichel/Yochelson}
  \label{sec:gc-impl-fenichel-yochelson}

  \lstset{caption={Implementation of Fenichel/Yochelson in C}}
  \lstset{label=lst:fenichel-yochelson-c}
  \lstinputlisting{../code/fenichel-yochelson.c}

  \lstset{caption={Header file for the Fenichel/Yochelson C implementation}}
  \lstset{label=lst:fenichel-yochelson-h}
  \lstinputlisting{../code/fenichel-yochelson.h}

   \section{Test Program}
   \label{sec:gc-impl-test-program}

   \lstset{caption={Linked list ``library'' with swappable GC}}
   \lstset{label=lst:lists-c}
   \lstinputlisting{../code/lists.c}

   \lstset{caption={Header file for the linked list ``library''}}
   \lstset{label=lst:lists-h}
   \lstinputlisting{../code/lists.h}

   \lstset{caption={Test program}}
   \lstset{label=lst:main-c}
   \lstinputlisting{../code/main.c}

   \lstset{caption={Makefile}}
   \lstset{label=lst:makefile}
   \lstinputlisting[style=makefile]{../code/Makefile}

  \section{Shared Code}
  \label{sec:gc-impl-shared}

  \lstset{caption={Shared root-management code}}
  \lstset{label=lst:shared-c}
  \lstinputlisting{../code/shared.c}

  \lstset{caption={Definition of cells and shared prototypes}}
  \lstset{label=lst:shared-h}
  \lstinputlisting{../code/shared.h}
\end{wide}

\end{lscape2col}
\end{appendices}

\bibliographystyle{plain}
\bibliography{references}

\end{document}
