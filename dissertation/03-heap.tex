\chapter{The Heap as an Array}
\label{sec:heap}

This chapter explains and justifies how the heap can be modelled as an
array, and introduces a specific formalism for reasoning about
array-based programs, along with some examples of its use. This
formalism will be used in later chapters.

Section \ref{sec:heap-why} justifies the use of modelling the heap as
a simple array, with Section \ref{sec:heap-separation} presenting an
alternative model. Section \ref{sec:heap-arrays} then introduces Gries
array formalism\cite{Gries87}.

\section{Why Arrays?}
\label{sec:heap-why}

In a physical computer, the heap lives in virtual memory, where
virtual memory consists of an integer-indexed collection of machine
words. In the case of x86 computers, these words are bytes, and the
range of addresses allows for 4GiB of virtual memory per
process. Contiguous addresses refer to contiguous heap locations, and
the addressing starts at zero, although this is usually mapped to an
invalid memory location to catch errors.

This sounds rather like a zero-indexed array, and in fact we can
consider the heap to be such an array to allow simple reasoning. More
abstract formalisms, such as considering the heap to be a digraph, are
possible, and one such alternative shall be reviewed, however I argue
that the array model is the simplest way to capture the semantics of
memory as seen by programs.

In the array formalism, pointers simply become indices into this
array. We need to be careful about illegal addresses, such as zero,
but by imposing a validity condition upon pointers, such as requiring
they all point to the start of a cell, we can side-step this issue.

\section{The Alternative: Separation Logic}
\label{sec:heap-separation}

Separation logic, introduced by Reynolds in 2002\cite{Reynolds02}, is
an extension of Hoare logic for reasoning about states consisting of a
stack and heap, where the heap is a partial function from addresses to
values. In particular, the logic provides the following extra
predicates to reason about the heap,

\begin{itemize}
  \item $\mathbf{emp}$, the heap is empty;

  \item $e \mapsto e'$, the heap is the singleton heap, where the
    address $e$ maps to the value $e'$;

  \item $P \sepc Q$, the heap can be split into two disjoint partitions
    where $P$ holds in one and $Q$ in the other;

  \item $P \sepi Q$, were the heap to be extended by a disjoint part
    which satisfies $P$, the entire heap would satisfy $Q$.
\end{itemize}

In addition to these operators, there is an additional deduction rule,
the frame rule, allowing modular reasoning about program components,
\begin{prooftree}
  \AxiomC{$\htriple{P}{C}{Q}$}
  \UnaryInfC{$\htriple{P \sepc R}{C}{Q \sepc R}$}
\end{prooftree}
which is valid if no free variables in $R$ are modified in $C$, it
states that a proof of a program in a small context can be reused in a
larger context.

In this formalism pointers are parameters to the partial function
which is the heap, where the heap can consist of anything. For
example, you could reason about a heap of cells, where pointers are
all multiples of 2. Considering addresses to be numbers allows pointer
arithmetic, as does the array formalism, but reasoning about
properties such as reachability is more difficult in this case.

The main reason separation logic is not used in this project is
because it will be necessary to reason about the entire state, and
there would be very little scope for separation to be of use. Thus,
using separation logic could result in additional complication for
little gain. However, in the case of non-stop-the-world collectors,
the tools of something like separation logic would undoubtedly be
useful.

\section{Reasoning with Arrays}
\label{sec:heap-arrays}

Abstract program logics such as Hoare logic typically do not worry
about data types beyond simple primitives such as integers and
booleans, however this is not enough to model all aspects of
programming. Gries introduces a model of arrays\cite{Gries87}, where
an array of type $[T]$ is treated as a function from some finite
contiguous subset of the natural numbers (including zero) to $T$, and
array assignment is modelled as functional override.

Henceforth, $A$ shall be used as an example array. We denote an array
as a tuple, where the $n$th element of the tuple is the value of
$A[n]$. The bounding indices of the array are not constrained in
general, and so these can be accessed by the values $A.lower$ and
$A.upper$, leading to a definition of the domain of an array $\dom~A =
\{i~|~A.lower \leq i \leq A.upper\}$.

\begin{definition}[Array Access Triple]
  Assigning an array member to a variable is exactly the same as
  normal assignment, with the additional precondition that the array
  subscript be in range.
  \[\htriplem{P\left[A[n]/x\right] \land n \in \dom~A}{x := A[n]}{P}\]
\end{definition}

\begin{example}[Summing a list]
  \label{exmpl:heap-sum}
  
  Let's denote by $\mathrm{sum}~A$ (where $A$ is an array) the sum of
  all the elements in it, and now we want to prove the following
  program correctly computes this value.

\begin{lstlisting}
i   := A.lower;
sum := 0;
while i <= A.upper do
    sum := sum + A[i];
    i := i + 1
\end{lstlisting}

  \begin{figure}[t]
    \centering
      \begin{prooftree}
        \AxiomC{$\ltrue \implies I''$}

        \AxiomC{$\htriple{I''}{L1}{I'}$}
        \AxiomC{$\htriple{I'}{L2}{I}$}
        \BinaryInfC{$\htriple{I''}{L1; L2}{I}$}

        \AxiomC{$\htriple{I \land \mathtt{i} \leq A.upper}{L4}{J}$}
        \AxiomC{$\htriple{J}{L5}{I}$}
        \BinaryInfC{$\htriple{I \land \mathtt{i} \leq A.upper}{L4;
            L5}{I}$}
        \UnaryInfC{$\htriple{I}{while i < A.upper do (L4; L5)}{I \land
            \mathtt{i} > A.upper}$}

        \BinaryInfC{$\htriple{I''}{\ldots}{\mathtt{sum} =
            \mathrm{sum}~A}$}

        \BinaryInfC{$\htriple{\ltrue}{\ldots}{\mathtt{sum} =
            \mathrm{sum}~A}$}
      \end{prooftree}
    \caption{Proof tree for the list sum example.}
    \label{fig:exmpl:heap-sum-tree}
  \end{figure}

  Specifically, we want to prove $\mathtt{sum} = \mathrm{sum}~A$. We
  shall do so by using this loop invariant, $\mathtt{sum} =
  \mathrm{sum}~A[A.lower : \mathtt{i} - 1]$, where $A[i:j]$ is the
  array consisting of the elements of $A$ between $i$ and $j$, and if
  $i = j + 1$ is the empty array. A proof tree is displayed in Figure
  \ref{fig:exmpl:heap-sum-tree}, where the following definitions hold,
  \begin{align*}
    J &\equiv \mathtt{sum} = \mathrm{sum}~A[A.lower : \mathtt{i} + 1]\\
    I' &\equiv I[0/\mathtt{sum}]\\
    &\equiv 0 = \mathrm{sum}~A[A.lower : \mathtt{i} - 1]\\
    I'' &\equiv I'[A.lower/\mathtt{i}]\\
    &\equiv 0 = \mathrm{sum}~A[A.lower : A.lower - 1]
  \end{align*}
\end{example}

\begin{definition}[Array Assignment Triple]
  Assignment to an array is modelled as functional override,

  \begin{align*}
    \htriplem{P[(A; i : e)/A] \land i \in \dom~A}{A[i] = e}{P}
  \end{align*}

  Where $(A; i : e)[j]$ evaluates to $e$ if $i = j$, and $A[j]$
  otherwise.
\end{definition}

\begin{example}[Doubling a list]
  \label{exmpl:heap-double}

  We want to prove that the following program doubles the value of
  every element of the array $A$, storing the result in $B$,
\pagebreak
\begin{lstlisting}
i := A.lower;
B := A;
while i <= A.upper do
    B[i] = B[i] * 2;
    i := i + 1
\end{lstlisting}

  \begin{figure}[t]
    \centering
    \begin{prooftree}
      \AxiomC{$\ltrue \implies I''$}

      \AxiomC{$\htriple{I''}{L1}{I'}$}
      \AxiomC{$\htriple{I'}{L2}{I}$}
      \BinaryInfC{$\htriple{I''}{L1; L2}{I}$}

      \AxiomC{$\htriple{I \land \mathtt{i} \leq A.upper}{L4}{J}$}
      \AxiomC{$\htriple{J}{L5}{I}$}
      \BinaryInfC{$\htriple{I \land \mathtt{i} \leq A.upper}{L4;
          L5}{I}$}
      \UnaryInfC{$\htriple{I}{\ldots}{I \land \mathtt{i} > A.upper}$}

      \BinaryInfC{$\htriple{I''}{\ldots}{D}$}

      \BinaryInfC{$\htriple{\ltrue}{\ldots}{D}$}
    \end{prooftree}
    \caption{Proof tree for the list doubling example.}
    \label{fig:exmpl:heap-double-tree}
  \end{figure}

  And we shall use the invariant $\forall A.lower \leq j < i,\ B[j]
  = 2 A[j]$ to do so. A proof tree is displayed in Figure
  \ref{fig:exmpl:heap-double-tree}, where the following definitions hold,
  \begin{align*}
    D &\equiv \forall A.lower \leq i < A.upper,\ B[i] = 2 A[i]\\
    J &\equiv \forall A.lower \leq j < i + 1,\ B[j] = 2 A[j]\\
    I' &\equiv I[B/A]\\
    &\equiv \forall A.lower \leq j < i,\ A[j] = 2 A[j]\\
    I'' &\equiv I'[A.lower/\mathtt{i}]\\
    &\equiv \forall A.lower \leq j < A.lower,\ A[j] = 2 A[j]
  \end{align*}
\end{example}

Thus, we can see that array reasoning is a very simple and intuitive
tool, and as the axioms are very simple, they are amenable to machine
verification or automation, although this will not be used in the project.

\section{Summary}
\label{sec:heap-summary}

This chapter has introduced a simple formalism for reasoning about
arrays, and showed two small examples of its use. It was argued that
modelling the heap as an array is, whilst a very simplistic approach,
entirely appropriate for when the entirety of the heap is being
reasoned about, as is intended to be done in this dissertation.

In future chapters the heap will be treated as an array, and so the
reader should keep in mind the access and assignment rules. However,
pointers shall be defined in such a way that they are always valid,
and so the reader does not need to concern themselves with checking if
heap access is within bounds or not, and bounds checking will be
dropped for simplicity.
