\chapter{Formalising Garbage Collection}

\section{What Partial Correctness Means}

Myreen starts out by formalising the \gls{heap} as a finite partial
mapping from addresses to \glspl{cell} of \glspl{pointer} and
\gls{data}. He then goes on to define garbage collection as removing
all \gls{garbage} \glspl{cell} from the \gls{heap} and then performing
a consistent renaming of addresses\cite{Myreen10}. This may appear to
be quite strict, as not all \glspl{garbage collector} examine the
entire \gls{heap} in every collection, but it shall be shown how this
can be generalised to other types of \gls{collector}.

\begin{definition}[Heap]
  The heap is a finite partial mapping, $h$, from addresses (and null) to data:

  \[h : \addr \finparmap \listof{\left(\addr + \nullp\right)} \x \data\]

  We allow null fields in order to store data inside pointers.
\end{definition}

We assume that there is a set $roots$ which contains the \glspl{root}
to be used during \gls{garbage collection}. In practice, this set
would consist of all variables in scope.

\begin{definition}[Reachable]
  All roots are reachable; all cells which can be reached by following
  pointers from the roots are reachable.

  \begin{minipage}{.5\textwidth}
    \begin{prooftree}
      \AxiomC{$r \in roots$}
      \UnaryInfC{$r \in \reach{h}{roots}$}
    \end{prooftree}
  \end{minipage}
  \begin{minipage}{.5\textwidth}
    \begin{prooftree}
      \AxiomC{$r \in \reach{h}{roots}$}
      \AxiomC{$r \pointsto x$}
      \BinaryInfC{$x \in \reach{h}{roots}$}
    \end{prooftree}
  \end{minipage}

  The notation $r \pointsto x$ means that there is a pointer in the
  cell $r$ to the cell $x$.
\end{definition}

\begin{definition}[Garbage]
  A cell is garbage if it is not reachable.

  \begin{prooftree}
    \AxiomC{$x \in \dom~h$}
    \AxiomC{$x \notin \reach{h}{roots}$}
    \BinaryInfC{$x \in \garbage{h}{roots}$}
  \end{prooftree}
\end{definition}

Now we can formalise what we mean by ``\gls{garbage collection}''. It
is the process of removing from the heap all \glspl{cell} which are
not reachable. We can express this with a domain
restriction. In order to reason about \glspl{collector} which update
addresses (to borrow another idea from Myreen), we also need to allow
addresses to be renamed.

\begin{definition}[Translation]
  Given an injective function $f : \addr \to \addr$, let $\map{f}{x}$
  apply $f$ to every non-null pointer in a cell $x$, and
  $\rename{f}{h}$ update all addresses in a heap $h$ by $f$.

  We now define a relation which performs consistent renaming over the
  entire heap:

  \[(h, roots) \translate \left(\rename{f}{h}, \map{f}{roots}\right)\]
\end{definition}

Myreen requires $f$ to be an involution, as he is concerned with
correctness for copying collectors, and so considers only global swap
functions. In general, however, if we merely wish to preserve the
property of not ``clobbering'' the heap (merging addresses or throwing
addresses away), an injective function is all that is necessary. Thus,
we can see that even at his highest-level formalism, Myreen already
specialises for the case of copying collectors, and so it is not
applicable outside of that domain.

\begin{definition}[Garbage Collection]
  Garbage collection is the process of removing all garbage cells from
  the heap, and performing a consistent renaming of addresses.

  \[(h, r) \gc (h', r') = (h \restrict \reach{h}{roots}, r) \translate
  (h', r')\]
\end{definition}

\subsection{Less Strict Collectors}

The requirement of removing all \gls{garbage} may appear to be too
strict for some types of collectors, such as generational (which only
look at a part of the heap at a time) or incremental (which only
perform a fixed amount of work each collection), however the formalism
can also be applied to these types of collectors.

\subsubsection{Generational Collectors}

The problem with generational collectors is that we only look at one
generation at a time, and so the heap may contain garbage in other
generations which is not touched. Even worse, there may be
\gls{garbage} in the collected generation, but which is pointed to by
something from an older generation, and so is treated as live! This
latter problem is known as \gls{nepotism}, and is a problem that must
be overcome with \glspl{generational garbage collection}.

We can resolve this problem with two tactics.

\begin{enumerate}
  \item For the correctness of minor collection, regard each
    generation (along with all younger ones) as its own
    \gls{heap}. The \glspl{root} are then the \glspl{root} of the
    computation, plus any inbound \glspl{old-to-young pointer}.

  \item For the correctness of major collection, consider the
    \gls{heap} as a whole.
\end{enumerate}

Generational correctness, then, consists of showing that both the
minor and major collections are correct. Furthermore, if the collector
records young-to-old pointers as well as old-to-young, each generation
can be regarded as its own heap, simplifying the collection correctness.

\subsubsection{Incremental Collectors}

\Glspl{incremental collector}, similarly to generational collectors, only do
a fixed amount of work each time. However, rather than dividing the
heap up into subheaps, they pick up where they left off at the start
of each collection. When the entire heap has been collected, they
start again at the beginning.

We can envision the collector as being somewhat like this:

\begin{verbatim}
set up the initial collector state

while there is work to be done:
    do a little collection
    do a little mutation

do any finalisation work
\end{verbatim}

Hopefully it can be seen that there is a trivial transformation from
any \gls{incremental collector} to a stop-the-world \gls{collector}:
we simply remove the ``do a little mutation'' step from the
collector. The collector can still be invoked under the same
circumstances as it would be when it was incremental, but now we pause
the \gls{mutator} until all of the collection is done, and then return
control to it.

\subsection{Specialisation}

This is intended to be a fairly general formalism that can be applied
to a wide variety of garbage collectors, however this also means that
it can be specialised in some cases.

In the case of non-moving collectors, such as mark-sweep, the
translation relation ($\translate$) is the identity, which is
equivalent to using $f = \id$, and so we must only consider the domain
restriction.

Similarly, in copying collectors, where data is moved from one space
to another, the translation function ($f$) can be modelled as a swap
function such that $f \of f = \id$. This makes $f$ an involution,
which is a stricter requirement than being an injective function.

\section{A Framework for Proof}

In this section, a framework for proving correctness is proposed, as a
series of specific lemmata, which should hopefully be easier to work
with than the abstract formalism of correctness.

Additionally, having such a framework allows us to reason about small
parts of a \gls{garbage collector}, rather the entire thing. It may be
unfeasible to prove correctness for a large and complicated
\gls{collector}, but proving that (for example) \glspl{pointer} are
modified correctly may be enough.

\subsection{Garbage Collection: The Commonalities}

Firstly, we must establish what it is that is common to all
\glspl{garbage collector}. Somewhat trivially, we can see that
\glspl{garbage collector}:

\begin{enumerate}
  \item Partition the \glspl{cell} into cells to keep and cells to
    discard.
  \item Free the cells to be discarded.
  \item Move the cells to be kept.
\end{enumerate}

In the case of non-moving collectors, such as mark-sweep, the last
point is not performed. In the case of moving collectors, such as
mark-compact, the second is not explicitly performed. We can, however,
intuitively see that these three steps are necessary for a garbage
collector to be correct.

Points 1 and 2 are performed by the \gls{heap} domain restriction, and
point 3 is performed by the address translation function.

\subsection{Heap Domain Restriction}

In order to show that the \gls{collector} performs a correct
\gls{heap} domain restriction, we must show that nothing necessary is
freed and no garbage is preserved, and that data fields are
in the same state at the end of collection as they were at the start.

\subsubsection{Cells to be Preserved}

We define a set, $\keep{h}{roots}$ consisting of the set of
\glspl{cell} preserved by the collector.

\begin{lemma}[Nothing live discarded]
  \[\reach{h}{roots} \subseteq \keep{h}{roots}\]
\end{lemma}

\begin{lemma}[No garbage preserved]
  \[\garbage{h}{roots} \cap \keep{h}{roots} = \emptyset\]

  It may be easier to prove the equivalent, reformulated, version:

  \[\keep{h}{roots} \subseteq \reach{h}{roots}\]
\end{lemma}

It should be very easy to see that proving these two lemmata shows
that $\keep{h}{roots} = \reach{h}{roots}$.

\subsubsection{Invariance of Data}

The only operations performed by the abstract collector are removing
garbage cells and renaming \glspl{pointer}; there is no room for
modifying data, or modifying \glspl{pointer} outside of the
translation function (but that will be covered in the next
subsection).

Thus, we also need to show:

\begin{lemma}[Data is unmodified]
\end{lemma}

\begin{lemma}[Null pointers remain null]
\end{lemma}

\subsection{Address Translation}

The second operation which the abstract \gls{collector} performs is a
consistent renaming of \glspl{pointer} across the entire
\gls{heap}. In order to reason about this, we must extract the
\gls{pointer} renaming portion of the \gls{garbage collector} under
consideration.

We shall call this translation function $f$.

\subsubsection{Consistency of Translation Function}

Fundamentally, we need the translation function to be consistent. It
cannot merge two addresses. It cannot throw any addresses
away. Furthermore, it must (trivially) be from the set of addresses to
the set of addresses.

We can express all of these with two requirements,

\begin{lemma}[$f$ must be a function]
  If $f$ is a function, then every address maps to one other address.

  \[\forall a : \addr,\ \exists ! b : \addr,\ (a, b) \in f\]
\end{lemma}

\begin{lemma}[$f$ must be injective]
  If $f$ is injective, then every pair of unique addresses map to
  unique addresses.

  \[\forall a, b \in \dom~f,\ f(a) = f(b) \iff a = b\]
\end{lemma}

\subsubsection{Consistency of Function Application}

Finally, we need to show that $f$ is applied consistently across the
entire \gls{heap}. We can separate this into thinking about the
\gls{root} set and the live \gls{heap}

\begin{lemma}[Roots are remapped]
  Specifically, the roots are remapped, no new roots are introduced,
  and no roots are thrown away:

  \[roots' = \map{f}{roots}\]
\end{lemma}

\begin{lemma}[Heap is remapped]
  As address remapping is done after the domain restriction, we only
  need to concern ourselves with live cells here:

  \[h' \restrict \keep{h'}{roots'} = \rename{f}{h \restrict \keep{h}{roots}}\]
\end{lemma}

\subsection{Sufficiency of the Framework}

We now show that the lemmata, together, show that a relation is a
garbage collection relation.

\begin{theorem}[Heap Domain Restriction]
  Firstly, we must show that only those cells which are live are
  preserved by the domain restriction.

  \begin{align*}
    \keep{h}{roots} &\subseteq \reach{h}{roots} & \mbox{lemma 1}\\
    \reach{h}{roots} &\subseteq \keep{h}{roots} & \mbox{lemma 2}\\
    \therefore \keep{h}{roots} &= \reach{h}{roots}
  \end{align*}

  As lemmata 3 and 4 show that data is not modified, we can simply state,

  \[h' = h \restrict \reach{h}{roots}\]
\end{theorem}

\begin{theorem}[Address Translation]
  Given lemmata 5 and 6, $f$ is a consistent renaming function, and so
  we simply need to show that we can deduce a translation relation. We
  can see from lemma 8 that there is no garbage in the resultant heap:

  \begin{align*}
    h' \restrict \keep{h'}{roots'} &= \rename{f}{h \restrict
      \keep{h}{roots}} &\mbox {lemma 8}\\
    &= h' & \mbox{rename}
  \end{align*}

  We can now consider the moving/renaming part of the garbage
  collector as a whole:

  \begin{align*}
    (h, r) \to (h', r') &= (h, r) \to \left(\rename{f}{h \restrict
        \keep{h}{roots}}, \map{f}{r}\right) & \mbox{lemma 7}\\
    &= (h \restrict \keep{h}{roots}, r) \translate (h', r') &\mbox{translate}
  \end{align*}
\end{theorem}

Now these two results can be put together to show correct garbage
collection.

\begin{theorem}[Sufficiency of Framework]
  \begin{align*}
    (h, r) \to (h', r') &= (h \restrict \keep{h}{roots}, r) \translate
    (h', r') & \mbox{thm. 2}\\
    &= (h \restrict \reach{h}{roots}, r) \translate (h', r') &
    \mbox{thm. 1}\\
    &= (h, r) \gc (h', r') &\mbox{gc}
  \end{align*}
\end{theorem}

\section{Correctness of Mark-Sweep Collectors}

\Gls{mark-sweep} \glspl{collector} perform no moving or address
translation, and preserve all \glspl{cell} which get marked. The proof
obligations are thus as follows:

\begin{prooftree}
  \AxiomC{$\markset{h}{roots} = \keep{h}{roots}$}
  \AxiomC{$\keep{h}{roots} = \reach{h}{roots}$}
  \BinaryInfC{$(h, roots) \gc (h', roots')$}
\end{prooftree}

In addition, for the lack of data mutation to hold, all mark flags
must be reset at the end of collection.

In this section, we shall consider two simple mark-sweep collectors,
specialised for different types of language, and show how the proofs
for each differ.

\subsection{For Immutable Languages}

Armstrong and Virding\cite{Armstrong95} introduce a simple
\gls{mark-sweep} \gls{collector} for Erlang, making use of the
immutability of the language in order to combine the mark and sweep
stages. The \gls{collector} operates on a \gls{heap} of cons cells,
and uses a \gls{pointer} \texttt{SCAV} to keep track of the current
position of the \gls{collector} in the heap.

The algorithm is as follows:

\begin{lstlisting}
last = current
SCAV = hist(last)
while (SCAV != first) {
    if (marked(SCAV)) {
        possibly_mark(car(SCAV));
        possibly_mark(cdr(SCAV));
        unmark(SCAV);
        last = SCAV;
        SCAV = hist(last);
    } else {
        tmp = SCAV;
        SCAV = hist(SCAV);
        set_history(last, SCAV);
        free_cons(tmp);
    }
}
\end{lstlisting}

The \texttt{possibly\_mark} function follows and marks its argument if
it is a pointer; \texttt{first} and \texttt{current} point to the
first and last allocated \glspl{cell}; the \texttt{hist} function
returns a \gls{pointer} to the \gls{cell} allocated before its
argument. It is assumed that all roots (except \texttt{current}) have
been marked before calling the collector.

This \gls{collector} works because the history fields form a linked
list of allocated cells, going back to the beginning of time. As the
language is immutable, pointers must always point back in time, and so
if a cell is unmarked by the time it is reached (by following the
history list), then it must be garbage.

We can express this as a loop invariant: the portion of the heap which
has been considered consists only of unmarked, reachable cells.

\subsubsection{Partial Correctness}

Let $\id~x$ be the ``allocation ID'' of a cell $x$. The first cell has
an ID of 0, and the ID of every other cell is 1 + the ID of the
previously allocated cell.

We can then express the loop invariant as follows:

\[\forall x,\ \id~x > \id~\mathtt{SCAV} \implies
\lnot \mathrm{marked}(x) \land x \in \reach{h}{roots}\]

The astute reader will notice that this says nothing about the first
cell. This is because the collector contains no machinery to alter
which cell is considered the ``first'' cell, and so cannot free it. In
order for the collector to be correct, we'll have to abuse terminology
a bit and just define the first cell to be a root (and so not
garbage).

\todo{Actually prove these, an argument is a starting point, not a
  proof. Read the Gries book to see if I could apply his array
  reasoning.}

\begin{lemma}[Lines 5 to 9 preserve the invariant]
  \label{lem:erlang1}
  \[\htriple{\mathrm{marked}(\mathrm{SCAV}) \land \mathrm{SCAV} \neq
    \mathrm{first} \land I}{L5--9}{I}\]

  Lines 5 and 6 aren't relevant here. Line 7 unmarks
  \texttt{SCAV}. Then we have,

  \begin{prooftree}
    \AxiomC{$\htriple{I''}{L8}{I'}$}
    \AxiomC{$\htriple{I'}{L9}{I}$}
    \BinaryInfC{$\htriple{I''}{L8--9}{I}$}
  \end{prooftree}

  Where

  \begin{align*}
    I'' &\iff I'[\mathrm{hist}(\mathrm{last})/\mathrm{SCAV}]\\
    &\iff
    I[\mathrm{SCAV}/\mathrm{last}][\mathrm{hist}(\mathrm{last})/\mathrm{SCAV}]\\
    &\iff \forall x,\ \id~x > \id~\mathrm{hist}(\mathrm{SCAV}) \implies
    \lnot \mathrm{marked}(x) \land x \in \reach{h}{roots}\\
  \end{align*}

  We know that $I$ holds (by precondition), that \texttt{SCAV} is
  reachable (by if rule), and that \texttt{SCAV} is unmarked (by line
  7). Therefore, we know that $I''$ holds.

  Thus, the lemma is true.
\end{lemma}

\begin{lemma}[Lines 11 to 14 preserve the invariant]
  \label{lem:erlang2}
  \[\htriple{\lnot\mathrm{marked}(\mathrm{SCAV}) \land \mathrm{SCAV}
    \neq \mathrm{first} \land I}{L11--14}{I}\]

  We know that \texttt{SCAV} is garbage, and that $I$ holds. Thus, if
  we remove \texttt{SCAV}, $I$ will continue to hold as we
  progress. This is what lines 13 and 14 do.
\end{lemma}

\begin{proof}[Garbage collection correctness]
  Having proven these lemmata, we can now go ahead to show the
  correctness of the collector as a whole,

  \begin{prooftree}
    \AxiomC{$\htriple{I''}{L1}{I'}$}
    \AxiomC{$\htriple{I'}{L2}{I}$}
    \BinaryInfC{$\htriple{I''}{L1--2}{I}$}

    \AxiomC{lem. \ref{lem:erlang1}}
    \AxiomC{lem. \ref{lem:erlang2}}
    \BinaryInfC{$\htriple{I \land \mathrm{SCAV} \neq
        \mathrm{first}}{L4--15}{I}$}

    \UnaryInfC{$\htriple{I}{L3--16}{I \land \mathrm{SCAV} =
        \mathrm{first}}$}

    \BinaryInfC{$\htriple{I''}{\ldots}{I \land \mathrm{SCAV} =
        \mathrm{first}}$}
  \end{prooftree}

  Where the following hold:

  \begin{align*}
    I &\iff \forall x,\ \id~x > \id~\mathtt{SCAV} \implies \lnot
       \mathrm{marked}(x) \land x \in \reach{h}{roots}\\
    I' &\iff I[\mathrm{hist}(\mathrm{last})/\mathrm{SCAV}]\\
    &\iff \forall x,\ \id~x > \id~\mathrm{hist}(\mathrm{last})
      \implies \lnot \mathrm{marked}(x) \land x \in \reach{h}{roots}\\
    I'' &\iff I'[\mathrm{current}/\mathrm{last}]\\
    &\iff \forall x,\ \id~x > \id~\mathtt{\mathrm{hist}(\mathrm{current})} \implies \lnot
       \mathrm{marked}(x) \land x \in \reach{h}{roots}\\
    &\iff \lnot\mathrm{marked}(\mathrm{current}) \land \mathrm{current}
       \in \reach{h}{roots}
  \end{align*}
\end{proof}

This proof, however, is flawed. It does not show that all reachable
cells will be marked before they are collected. Thus, a separate proof
of this is provided:

\todo{Prove properly.}

\begin{proof}[All reachable cells are marked before being collected]
  We know that pointers can only point back in time, as the language
  is immutable. This means that $a \pointsto b \implies \id~a >
  \id~b$. By $I$ in Total Correctness, we know that the history list
  is ordered by ID. By the code, we know that we traverse the history
  list in order. Thus, if something is reachable, it will be marked
  before it is visited.
\end{proof}

\subsubsection{Total Correctness}

We can use allocation IDs to express that the history list forms an
unbroken chain of cells, all the way back to the first cell, as
follows:

\[\forall x,\ \left(x = \mathrm{first} \implies x =
  \mathrm{hist}(x)\right) \land \left(x \neq \mathrm{first} \implies
  \exists n \in \mathbb N_{1},\ \id~x = n + \id~\mathrm{hist}(x)\right)\]

The loop condition can be restated as $\id~\mathtt{SCAV} \neq 0$. As
there are no IDs below 0 ($\mathrm{hist}(\mathrm{first}) =
\mathrm{first}$), that can be further rewritten to $\id~\mathtt{SCAV}
> 0$. \texttt{SCAV} starts out as the last-allocated cell, and so has
an ID $\geq 0$. Now, in order to prove termination, we simply need to
show that $\id~\mathtt{SCAV}$ decreases at every iteration of the
loop.

To simplify, let us throw away everything which does not relate to
mutating \texttt{SCAV}, giving the following loop:

\begin{lstlisting}
while (SCAV != first) {
    if (marked(SCAV)) {
        last = SCAV;
        SCAV = hist(last);
    } else {
        SCAV = hist(SCAV);
    }
}
\end{lstlisting}

Substituting variables, we get

\begin{lstlisting}
while (SCAV != first) {
    if (marked(SCAV)) {
        SCAV = hist(SCAV);
    } else {
        SCAV = hist(SCAV);
    }
}
\end{lstlisting}

And then we can eliminate the if statement:

\begin{lstlisting}
while (SCAV != first) {
    SCAV = hist(SCAV);
}
\end{lstlisting}

Trivially, we can now see that there are two cases:

\begin{description}
  \item[Case 1, $\mathtt{SCAV} = \mathrm{first}$] In this case,
    $\mathrm{hist}(\mathtt{SCAV}) = \mathtt{SCAV}$, and so
    $\id~\mathtt{SCAV}$ does not decrease, but this situation is
    precisely the loop termination condition.

  \item[Case 2, $\mathtt{SCAV} \neq \mathrm{first}$] In this case, we
    know by the history list assumption that
    $\id~\mathrm{hist}(\mathtt{SCAV}) < \id~\mathtt{SCAV}$, and so
    $\id~\mathtt{SCAV}$ does decrease. Furthermore, we know that the
    ID cannot decrease below zero, and so we get a strictly decreasing
    sequence of positive natural numbers. Clearly, this sequence is
    not infinite, and so the loop terminates.
\end{description}

\subsubsection{Assumptions}

I have had to assume that all cells pointed to by roots are marked
prior to collection, and furthermore have been able to establish
details of how the allocator works. The allocator is not detailed in
the Armstrong/Virding paper, but the precondition that
\texttt{current} is never marked, and furthermore the observation that
pointers in it are never followed, suggests details of how the
allocator functions:

\begin{lstlisting}
tmp = current

current = get_new_cell()
unmark(current)
set_hist(current, tmp)

car(tmp) = null
cdr(tmp) = null

return tmp
\end{lstlisting}

This ensures that \texttt{current} is always the cell at the end of
the history list, and is the next cell to be allocated whenever one is
requested. This does impose an overhead of one cell, but it simplifies
the collector. We can also see that the free list is managed by
\texttt{get\_new\_cell()} and \texttt{free\_cons()}.

Finally, I have had to assume that \texttt{first} is never
garbage. The collector has no machinery to remove it, and so if it
does become garbage, the collector is no longer correct. In order to
resolve this, we can just define the \texttt{first} cell to be always
reachable. This imposes another one cell overhead.

\subsection{For Mutable Languages}

\todo{Proof of correctness of the Runciman collector}