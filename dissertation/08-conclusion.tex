\chapter{Conclusion}
\label{sec:conclusion}

This chapter shall discuss further work suggested to me by undertaking
this project, briefly review the prior chapters, and close with some
final thoughts on the matter.

\section{Related Work}
\label{sec:conclusion-related}

In 2007, McCreight, Shao, Lin, and Li produced their paper, \textit{A
  General Framework for Certifying Garbage Collectors and Their
  Mutators}\cite{McCreight07}, which models garbage collection
correctness by defining an interface between the mutator and the
collector, where the mutator accesses an abstract heap, and the
collector a concrete heap. The two are related by an invariant, and
the garbage collector is defined as a step relation, and by
parameterising the mutator proof by these two predicates, a mutator
may be verified independently of a collector. This is machine
verified and proofs for three separate collectors have been
produced.

My work was produced independently of this, and by the time I found
the paper the majority of the work in Chapters \ref{sec:marksweep},
\ref{sec:copying}, and \ref{sec:gc} was done. The key difference is in
separating the heaps as seen by the mutator and collector, allowing
independent verification. My approach does not do this, and so
verification of a mutator necessarily is collector-specific, as
different collectors will have different invariants and preconditions.

\section{Review}
\label{sec:conclusion-review}

Chapter \ref{sec:marksweep} produced precise formal definitions of
many common structures used by garbage collectors from first
principles, allowing reasoning about garbage collection at all to be
possible, then used these to formalise mark-sweep correctness and
prove correct the Armstrong/Virding collector. As this is quite a
simple collector, there are undoubtedly other proofs of its
correctness somewhere, however a non-verified proof is only as
reliable as the collection of people who have read it, and so each new
proof adds further evidence that it is correct.

Chapter \ref{sec:copying} extended the definitions produced in Chapter
\ref{sec:marksweep} to allow reasoning about copying collection, and
then verified the Fenichel/Yochelson collector. The formalism was then
compared with that of Myreen\cite{Myreen10} and shown to be very
similar, providing confidence that Myreen's formalism is a reasonable
one for verifying copying collectors, even though it was developed in
a very different way.

Chapter \ref{sec:gc} further extended the results of Chapter
\ref{sec:copying}, resulting in a natural progression from the more
specific to the more general, and gives a new formalism which allows
reasoning about different types of collectors, as shown in Section
\ref{sec:results-correctness}.

To conclude, the project contributes an independently-produced
standard for garbage collection correctness, which has been used,
indirectly, to verify two extant garbage collectors: the
Armstrong/Virding mark-sweep collector for languages with immutable
state, and the Fenichel/Yochelson copying collector for LISP systems.

\section{Further Work}
\label{sec:conclusion-further}

\subsection{Machine Verification}
\label{sec:conclusion-further-machineverification}

Whilst it would have been desirable to produce machine-verified
proofs, this was not done for two reasons: firstly, learning to use a
proof assistant would have been a significant amount of work to add to
the project, and secondly I believe the proofs are good enough in
their current form. Parts of the proofs may be amenable to automation,
however, which would have simplified the work whilst simultaneously
making it more rigorous.

Given the two proofs in the current forms, however, it would not be
infeasible to translate them into a form suitable for machine
verification, and to fill in the blanks, resulting in completely
machine-verified proofs. This endeavour would also highlight any
assumptions unknowingly made in this project, and bring to light any
further obligations, particularly those on the mutator or allocator.

\subsection{Run-time Invariant Checking}
\label{sec:conclusion-further-invariants}

The garbage collector is only a part of a larger system, consisting of
the allocator, mutator, and a possible intermediary memory-management
subsystem. Proving the garbage collector correct is all well and good,
but if the other components break the invariants or preconditions upon
which the garbage collector relies, it may fail, no matter how
rigorous the proof of correctness.

Unfortunately, the task of verifying a nontrivial system is very
difficult, and this is the reason why formal verification is so rare
in practice. Thus, it may be worthwhile to implement run-time checking
of invariants into the system, and to either halt or emit an error
when such an invariant is broken.

If this could be implemented efficiently, it would only be a small
overhead for increased reliability, as garbage collection bugs can not
arise in a verified collector where all of the expectations
hold. Furthermore, if this could be implemented such that invariants
are checked after every operation which could break them, it would
produce a powerful debugging tool, as a programmer would know exactly
where errors were introduced into the system.

\subsection{Incremental Garbage Collectors}
\label{sec:conclusion-further-incremental}

I have assumed that all garbage collectors are of the stop-the-world
type, but this is often not the case in practice. Incremental garbage
collectors are very common, especially in systems with need for rapid
response to input, and my formalism cannot reason about them at
all.

It would be desirable to extend the formalism to be able to cope with
incremental collectors, perhaps by modelling the system as a
mutate/collect loop, and determining what invariants are needed for
the collector to not interfere with the mutator and vice versa, whilst
still removing all garbage from the system.

Furthermore, as an incremental collector does not remove all garbage
from the system every time it completes a cycle (as the mutator may
have produced some more in the intervening time), a more flexible
notion of removing all garbage would be required. Possibly something
around the idea that, were the mutator to cease producing any garbage
for some period of time, the collector would remove it all. If this
time could be bounded, then this would also allow more accurate
reasoning about the space characteristics of programs with incremental
garbage collectors.

\subsection{Concurrent Garbage Collectors}
\label{sec:conclusion-further-concurrent}

It is increasingly the case that the garbage collector and mutator run
in separate threads, giving very low delays on removing garbage from
memory. It may even be the case that the collector and mutator run on
separate processor cores continuously, which results in this not just
being a special case of incremental collection.

As it was assumed that collectors are of the stop-the-world type,
there is no scope for reasoning about this in the current
formalism. In order to show correctness, it would be necessary to show
that, for all possible interleavings, the collector and mutator could
not interfere with each other. Whilst with incremental collection we
simply need to show this holds for a fixed mutate/collect cycle, here
it is much more difficult, as mutation can occur simultaneously with
garbage collection.

However, this would be a very difficult problem to approach in
general, and could possibly be attacked by making the problem more
similar to incremental collection by constraining where collection and
mutation occurs, allowing a large class of dangerous states to be
avoided at the cost of some concurrency.

\section{Final Thoughts}
\label{sec:conclusion-thoughts}

This project has achieved its goal, and whilst the motivation (that
there be very little work on generic verification of garbage
collectors) proved to be untrue, the approach developed here is
different from the one discovered and is, I believe, more amenable to
hand-verification of collectors.
