\chapter{Correctness of Copying Collectors}
\label{sec:copying}

This chapter extends the definition of correctness established for
mark-sweep collectors to copying collectors. A copying collector shall
be considered as two separate components, a copying part and an
address updating part, in Sections \ref{sec:copying-copying} and
\ref{sec:copying-address} respectively. Then, the Fenichel/Yochelson
collector shall be considered as an example of the formalism in
Section \ref{sec:copying-example}.

The same definitions of the heap (defn. \ref{def:ms-heap}), cells
(defn. \ref{def:ms-cell}), pointers (defn. \ref{def:ms-pointer}),
reachability (defn. \ref{def:ms-reachable}), and the live cell
invariant (defn. \ref{def:ms-live-cell-invariant}) will be used as in
the prior chapter. Unfortunately, the definition of word preservation
(defn. \ref{def:ms-word-preservation}) will have to change, because a
copying collector renames addresses.

The definition of $\free{c}$ is a bit subtle in the copying collector,
as there is no explicit free list. However, as there is a \textit{free
  pointer}, or equivalent, indicating the start of the free region of
the active semispace, we can consider there to be an implicit free
list, consisting of all cells after that point, and all cells in the
inactive semispace.

\section{The Copying Process}
\label{sec:copying-copying}

Before presenting a definition of correct copying, it is necessary to
highlight the inadequacies of the mark-sweep word
preservation. Firstly, it does not allow for mutation of pointers,
which is essential for a copying collector, but more subtly, it does
not allow for cells to be relocated, as the problem arises in the
$h[w] = h'[w]$ part of the formalism. If cells can be relocated, then
this makes no sense, as $w$ is an index into the heap, and so $w$ in
the new heap is not necessarily the same as $w$ in the old heap. Thus,
we also need to be able to find old $w$ values.

\pagebreak

\begin{definition}[Correct Copying]
  \label{def:c-correct-copying}
  After garbage collection, no allocated cells have been mutated,
  except in the pointer fields, in which case we apply an address
  translation function $f$.

  \begin{align*}
    \forall c \in h,\ \forall w \in c,\ &\lnot\free{c} \implies
    (\type{w} = \typointer \iff h[w] = f(h'[f^{-1}(w)])\\
    &\quad\quad\quad\quad \land \type{w} \neq \typointer
    \iff h[w] = h'[f^{-1}(w)])\\
    &\land \lnot\free{c} \iff f^{-1}(c) \in \reach{h'}{r'}
  \end{align*}
\end{definition}

\section{Address Translation}
\label{sec:copying-address}

In the definition of correct copying, a function $f$ was invoked to
update addresses. Unfortunately, not every function will do here. We
can see that the constant function won't work, as then everything
would be moved to the same place, resulting in data corruption. The
function needs to preserve uniqueness of addresses. Furthermore, in
order to reason about words in the original heap, $h'$, we need to be
able to go backwards: the function must be invertible.

\begin{definition}[Address Translation Function]
  \label{def:c-address-translation-function}
  The address translation function, $f$, is defined across the entire
  heap (total), preserves uniqueness (injective), and is invertible
  (surjective). Thus, $f$ is a bijective endofunction on addresses.

  Furthermore, $f$ preserves relative placement of words within the
  same cell, \[\forall w \in c,\ c - w = f(c) - f(w)\]
\end{definition}

In order to ease reasoning, it may be simpler to mandate that $f$ is
an involution, because showing that $\forall x,\ (f \of f)~x = x$ may
be easier than proving the more abstract properties of injectivity and
surjectivity. Furthermore, the function doesn't strictly need to be
total: it only needs to be defined for words which are still allocated
after collection. However, the totality requirement enables us to use
the array reasoning to build up the function, which is useful.

If we start with $f = \id$, then (for an involution) we simply need to
show that, for every allocated word $w$, the garbage collector
performs $f[w] = w'; f[w'] = w$. Furthermore, we need to show that
contiguous words within the same cell remain in the same relative
place after copying. This further obligation could have been avoided
by defining $f$ as a cell translation function, which moves entire
cells at a time, but then this would have complicated reasoning about
individual words.

Finally, we need to ensure that the roots are correctly translated, as
they may not be correct after things have been moved,

\begin{definition}[Root Translation]
  \label{def:c-root-translation}
  \[roots = \map{f}{roots'}\]

  Where $\map{f}{xs} = \langle f(x)~|~x \in xs \rangle$.
\end{definition}

\section{Case Study: A Garbage Collector for Lisp}
\label{sec:copying-example}

Fenichel and Yochelson\cite{Fenichel69} introduced a simple semispace
copying garbage collector for lists in Lisp systems, and assume the
existence of a correct collector for atoms. The collector works by
storing in the car of each cell a flag indicating that it has been
copied, and a forwarding pointer in the cdr, permitting a very simple
algorithm which has been recast in a Hoare-like form in Figure
\ref{fig:copying-example-algo}.

\begin{figure}[t]
  \centering
  \begin{lstlisting}
gc():
    flipconsspace()
    for k = 0; k < len roots; k ++:
        roots[k] = collect(roots[i])
    flipsemispace()

collect(p):
    if p is atomic:
        return collectatom(p)
    else if car(p) = ALREADYCOPIED:
        return cdr(p)
    else:
        a = car(p)
        b = cdr(p)
        q = cons(NIL, NIL)

        rplaca(p, ALREADYCOPIED)
        rplacd(p, q)

        nrplaca(q, collect(a))
        nrplacd(q, collect(b))

        return q
  \end{lstlisting}
  \caption{Imperative version of the Fenichel/Yochelson garbage collector.}
  \label{fig:copying-example-algo}
\end{figure}

Firstly, it must be noted that pointers are interpreted relative to
the semispaces. A pointer $p$ refers to the location $p$ in the
current semispace. Then, there are some functions which must be
defined: \texttt{flipconsspace} causes cons allocation to happen in
the other semispace; \texttt{flipsemispace} flips the roles of the
semispaces; \texttt{collectatom} is a garbage collection function for
atoms, which we assume to be correct; \texttt{nrplaca} and
\texttt{nrplacd} are like \texttt{rplaca} and \texttt{rplacd} except
they interpret the pointer in the other semispace.

We can thus see that the collector allocates cons cells for every
reachable cons cell in the other semispace, copying over car and cdr
as it goes, and then flips the roles of the semispaces, so that the
mutator uses the new space.

\subsection{Partial Correctness}
\label{sec:copying-example-partial}

Firstly, it is necessary to define what an address for this rather
strange system is. As pointers are interpreted relative to a
semispace, each pointer has two related memory addresses, which means
we can't just talk about $h[p]$. Let's define an address as a pair,
consisting of a semispace to interpret it in, and a pointer, $(s,
p)$. Assuming the two semispaces are contiguous, are numbered 0 and 1,
and pointers range from 0 to the size of the semispace, turning this
into a numeric address is done by the operation $s \times
\mathrm{semispace\_size} + p$. We shall proceed to use $s'$ to refer
to the semispace before garbage collection was initiated, and $s$ to
refer to the current (different) semispace.

Assuming the correctness of \texttt{flipconsspace} and
\texttt{flipsemispace}, the correctness of \texttt{gc} consists in
proving that \texttt{collect}($c$) returns a pointer to a copied
version of $c$, in addition to the loop invariant in Figure
\ref{fig:copying-example-partial-invariant}. In words, the invariant
states, for every call \texttt{collect}($c$), everything (and nothing
more) reachable from $c$ (including itself) is copied, pointers are
translated according to $f$, and non-pointers are preserved. The
invariant is sufficient, as shown in Figure
\ref{fig:copying-example-partial-sufficient}, as after the loop
terminates we have $k = \len \mathtt{roots}$.

\begin{figure}[t]
  \centering
  \begin{align*}
    \forall c \in h,\ \forall w \in c,\ &\lnot\free{c} \implies
    (\type{w} = \typointer \iff h[w] = f(h'[f^{-1}(w)])\\
    &\quad\quad\quad\quad \land \type{w} \neq \typointer
    \iff h[w] = h'[f^{-1}(w)])\\
    &\land \lnot\free{c} \iff c \in
    \reach{h'}{\{\mathtt{roots}[i]~|~\forall 0 \leq i < k\}}
  \end{align*}
  \captionsetup{format=default}
  \caption{Loop invariant for the Fenichel/Yochelson collector}
  \label{fig:copying-example-partial-invariant}
\end{figure}

\begin{figure}[t]
  \centering
  \begin{align*}
    \forall c \in h,\ \forall w \in c,\ &\lnot\free{c} \implies
    (\type{w} = \typointer \iff h[w] = f(h'[f^{-1}(w)])\\
    &\quad\quad\quad\quad \land \type{w} \neq \typointer
    \iff h[w] = h'[f^{-1}(w)])\\
    &\land \lnot\free{c} \iff c \in \reach{h'}{\{\mathtt{roots}[i]~|~\forall
      0 \leq i < \len\mathtt{roots}\}}\\
%
    \forall c \in h,\ \forall w \in c,\ &\lnot\free{c} \implies
    (\type{w} = \typointer \iff h[w] = f(h'[f^{-1}(w)])\\
    &\quad\quad\quad\quad \land \type{w} \neq \typointer
    \iff h[w] = h'[f^{-1}(w)])\\
    &\land \lnot\free{c} \iff c \in \reach{h'}{r'}
  \end{align*}
  \caption{Sufficiency of the loop invariant for the
    Fenichel/Yochelson collector}
  \label{fig:copying-example-partial-sufficient}
\end{figure}

\begin{theorem}[Partial Correctness]
\label{lem:fyc}
  Showing the correctness of \texttt{collect} consists of showing the
  following:

  \begin{enumerate}
  \item after \texttt{collect}($p$) terminates, both $p$ and all cells
    reachable from $p$ have been collected

  \item if a cell has been collected, then collecting it again returns
    a pointer to the previously copied version

  \item \texttt{collect}($p$) returns a pointer to a copied version of
    $p$

  \item \texttt{collect}($p$) updates the car and cdr of $p$ to
    \texttt{collect}(car($p$)) and \texttt{collect}(cdr($p$))
    respectively

  \item \texttt{collect} does not copy any unreachable cells
  \end{enumerate}
\end{theorem}

\begin{proof}
  By (1) and (5), we have $\forall c \in h,\ \lnot\free{c} \iff c \in
  \reach{h'}{\{\mathtt{roots}[i]~|~\forall 0 \leq i < k\}}$. By (2),
  (3), and (4) we have $\forall c \in h,\ \forall w \in c,\ \lnot\free{c}
  \implies \type{w} = \typointer \iff h[w] =
  f(h'[f^{-1}(w)])$

  As the collector does not deal with non-pointer values (atoms), we
  can ignore the lack of mutation of non-pointers entirely, and assume
  that is handled correctly by \texttt{collectatom}, which is
  justifiable as the collector explicitly sets aside atoms, and hands
  them over to another collector. And so, by assumption, we have
  $\forall c \in h,\ \forall w \in c,\ \lnot\free{c} \implies
  \type{w} \neq \typointer \iff h'[f^{-1}(w)]$, giving
  us,

  \begin{align*}
    \forall c \in h,\ \forall w \in c,\ &\lnot\free{c} \implies
    (\type{w} = \typointer \iff h[w] = f(h'[f^{-1}(w)])\\
    &\quad\quad\quad\quad \land \type{w} \neq \typointer
    \iff h[w] = h'[f^{-1}(w)])\\
    &\land \lnot\free{c} \iff c \in \reach{h'}{\{\mathtt{roots}[i]~|~\forall
      0 \leq i < \len\mathtt{roots}\}}
  \end{align*}
\end{proof}

\begin{lemma}[After \texttt{collect}($p$) terminates, both $p$ and all
  cells reachable from $p$ have been collected]
  \label{lem:c-example-reach}
\end{lemma}

\begin{proof}
  There are three cases to consider for $p$,

  \begin{description}
  \item[$p$ is atomic] we assume the correctness of
    \texttt{collectatom}

  \item[car($p$) = \texttt{ALREADYCOPIED}] we assume the correctness
    of the else clause of \texttt{collect}

  \item[otherwise] a new cell is allocated for $p$ in the new
    semispace, and $p$ updated to point to this, then the car and cdr
    of $p$ are recursively \texttt{collect}ed
  \end{description}
\end{proof}

\begin{lemma}[If a cell has been collected, then collecting it again
  returns a pointer to the previously copied version]
  \label{lem:c-example-duplication}
\end{lemma}

\begin{proof}
  By corollary of Lemma \ref{lem:c-example-ret}.
\end{proof}

\begin{lemma}[\texttt{collect}($p$) returns a pointer to a copied
  version of $p$]
  \label{lem:c-example-ret}
\end{lemma}

\begin{proof}
  There are three cases:

  \begin{description}
  \item[$p$ is atomic] we assume the correctness of
    \texttt{collectatom}

  \item[car($p$) = \texttt{ALREADYCOPIED}] we return cdr($p$), and
    assume correctness of the code which updated the cell

  \item[otherwise] The car is set to \texttt{ALREADYCOPIED}, the old
    cell is copied, and the cdr set to its new address.
  \end{description}
\end{proof}

\begin{lemma}[\texttt{collect}($p$) updates the car and cdr of $p$ to
  \texttt{collect}(car($p$)) and\\\texttt{collect}(cdr($p$))
  respectively]
  \label{lem:c-example-update}
\end{lemma}

\begin{proof}
  By inspection.
\end{proof}

\begin{lemma}[\texttt{collect} does not copy any unreachable cells]
  \label{lem:c-example-unreach}
\end{lemma}

\begin{proof}
  We can see that \texttt{gc} only calls \texttt{collect} on
  cells pointed to by roots, which are reachable by definition;
  furthermore we can see that \texttt{collect} only calls
  \texttt{collect} on cells which are either the car or cdr of a cell
  it is currently examining.
\end{proof}

\subsection{Total Correctness}
\label{sec:copying-example-total}

A quick read of the code will reveal that the only place in which
nontermination could possibly occur is in the recursive call in the
else clause of \texttt{collect}. However, equally obviously, we can
see that this will never happen, because the base case of the
recursion is that the car of a cell is \texttt{ALREADYCOPIED}, and as
soon as we hit the else case we do that. Thus, even if we were to
encounter a cyclic structure, reaching again the point at which we
entered it would end the recursion.

\begin{theorem}[Total Correctness]
  The Fenichel/Yochelson collector terminates on all possible heaps.
\end{theorem}

\begin{proof}
  By induction on the number of uncopied and non-atomic cells in the
  semispace we are copying from.

  \begin{description}
  \item[Base case, $n = 0$] In this case, every cell is either an atom
    or is marked as having being copied, in which case there is no
    recursive call, and so \texttt{collect} terminates.

  \item[Inductive case, $n = k + 1$] We have three cases, depending on
    what the cell currently being examined is,

    \begin{description}
    \item[atom] No recursive call: terminates.
    \item[copied] No recursive call: terminates.
    \item[else] We mark the cell as copied, and \texttt{collect} makes
      the recursive call. We now have a heap where $n = k$, which by
      assumption terminates.
    \end{description}
  \end{description}

  We do not need to worry about \texttt{gc}, as the only loop in there
  is over a fixed-size list, which must necessarily terminate.
\end{proof}

\section{Comparison with Myreen's Formalism}
\label{sec:copying-myreen}

Myreen's\cite{Myreen10} notion of copying correctness is somewhat more
abstract than mine, and is defined in terms of cells, rather than
words,

\begin{definition}[Myreen Correctness]
  \label{def:c-myreen-correctness}
  Garbage collection is a relation which filters out unreachable heap
  elements and renames the addresses, \[x \gc y = (\mathrm{filter}~x)
  \translate y\]

  Where filter and translate are defined as follows,

  \begin{prooftree}
    \AxiomC{$f \of f = \id$}
    \UnaryInfC{$(h, roots) \translate (\rename{f}{h}, \map{f}{roots}$}
  \end{prooftree}

  \begin{align*}
    \filter{h}{roots} &= (h \restrict \reach{h}{roots}, roots)\\
    \dom~(\rename{f}{h}) &= \mathrm{image}~f~(\dom~h)\\
    (\rename{f}{h})(f(x)) &= (\map{f}{as}, d) & \mbox{ wherever } h(x)
    = (as, d)\\
  \end{align*}
\end{definition}

That is, a relation is a garbage collection relation if it renames all
addresses in the heap by applying an involution $f$, and removes all
non-reachable cells from the heap.

This sounds rather similar to the formalism developed in this chapter,
but with a few key differences:

\begin{itemize}
\item $f$ is defined on cell addresses, rather than word addresses
\item $f$ is required to be an involution
\end{itemize}

Nothing is explicitly stated about non-pointer words, but the
strictness of the definition doesn't require this to be stated: by not
saying anything about them, it requires that they be preserved.

The greatest difference between the two formalisms is arguably the use
of cell addresses compared with the use of word addresses, and I think
the choice comes down to the different approaches of myself and
Myreen. Myreen's formalism was intended to be used to produce a
collector by refinement, and so a higher-level and simpler
specification was a desirable starting point; whereas my formalism was
more motivated by a desire to prove existing collectors correct, in
which case a necessity to easily reason about individual words arose,
and so a more realistic model of the heap was desirable.

\begin{theorem}[Myreen Correctness is at least as strong]
  We would like to show that Myreen Correctness implies both Correct
  Copying and Root Translation, in which case we know that Myreen
  Correctness is at least as strong as my formalism.
\end{theorem}

\begin{proof}
    Myreen's $f$ is only defined for cell start addresses, as his heap
    is a partial function from addresses to cells, but the function
    can be easily expanded to what we need by filling in the
    blanks. Having done that, let's rearrange to get $h$ and $r$ from
    his definition,

    \begin{align*}
      (h', r') \gc (h, r) &= (\filter{h'}{r'}) \translate (h, r)\\
      &= (\filter{h'}{r'}) \translate (\rename{f}{\filter{h'}{r'}}, \map{f}{r'})\\
      \therefore h &= \rename{f}{\filter{h'}{r'}}\\
      \therefore r &= \map{f}{r'}
    \end{align*}

    $r = \map{f}{r'}$ is exactly the definition of Root Translation.

    We can expand the definition of $h$ and filter to check that only
    reachable cells are in $h$,

    \begin{align*}
      \forall c \in h,\ \lnot\free{c} &\iff c \in \filter{h'}{r'}\\
      \reach{h'}{r'} &\iff \filter{h'}{r'}\\
      \therefore \forall c \in h,\ \lnot\free{c} &\iff c \in \reach{h'}{r'}
    \end{align*}

    Myreen's definition requires that non-pointer words are
    untouched, \[\forall c \in h,\ \forall w \in c,\ \type{w}
    \neq \typointer \iff h[w] = h'[f^{-1}(w)]\]

    And we can expand the definition of rename to ensure that pointers
    are appropriately modified, \[\rename{f}{h} \implies \forall c \in
    w, \forall w \in h,\ \type{w} = \typointer \iff h[w]
    = f(h'[f^{-1}(w)])\]

    Thus, Myreen Correctness implies both Correct Copying and Root
    Translation.
\end{proof}

\begin{theorem}[Correct Copying and Root Translation are weaker]
  We would like to show that Correct Copying and Root Translation, if
  and only if $f$ is an involution, imply Myreen Correctness, in which
  case Correct Copying and Root Translation alone are weaker than
  Myreen Correctness.
\end{theorem}

\begin{proof}
    As shown in the prior proof, the roots after collection in Myreen
    are exactly what is required by Root Translation, thus we only
    need to worry about the heap.

    After collection, the heap contains only allocated cells, which
    have been moved in memory by applying $f$, and all pointers have
    been updated by $f$, giving exactly the combination of filter and
    translate as used by Myreen.

    Thus, Correct Copying and Root Translation, if and only if we also
    mandate that $f$ is an involution, imply Myreen Correctness.
\end{proof}

Having shown both of these, we can conclude that my formalism is
strictly weaker than Myreen's, but with the additional requirement
that $f$ be an involution, they are equivalent.

Myreen\cite{MyreenEmail} agreed that his requirement of $f \of f =
\id$ is probably unnecessarily strict, and said that he used this
property as ``it also works, it was simpler to state and doesn't
require thinking whether one has to apply $f$ or the inverse of
$f$''. This is certainly true as my formalism requires applying $f$ or
$f^{-1}$ in the appropriate places, and results in a possibly less
obvious statement of correctness.

\section{Summary}
\label{sec:copying-summary}

This chapter has highlighted the differences between the formalisms
suitable for mark-sweep and copying collectors: correct copying
replacing word preservation, and the introduction of address
translation for pointer mutation during garbage collection.

It was argued that correctness for a copying collector comes down to
copying exactly the live cells and updating all pointers consistently.

Then the Fenichel/Yochelson\cite{Fenichel69} collector was considered,
and proofs of its partial and total correctness were developed. These
proofs shall not be referenced in future chapters, and merely serve as
an illustration of the utility of the definition of correctness
presented here. Definitions will be reused, however, in producing an
abstract formalism of garbage collection.

Finally, this formalism of copying correctness was compared with that
of Myreen\cite{Myreen10}, the differences were discussed, and then it
was shown that they are equivalent, given an additional lemma.
