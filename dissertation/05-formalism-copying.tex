\chapter{Correctness of Copying Collectors}
\label{sec:copying}

We shall use the same definitions of the \gls{heap}
(defn. \ref{def:ms-heap}), \gls{cell} (defn. \ref{def:ms-cell}),
\gls{pointer} (defn. \ref{def:ms-pointer}), and reachability
(defn. \ref{def:ms-reachable}) as in the prior chapter. Unfortunately,
the definition of word preservation
(defn. \ref{def:ms-word-preservation}) will have to change, because a
\gls{copying} \gls{collector} renames addresses.

\begin{definition}[Word Preservation]
  \label{def:c-word-preservation}
  After garbage collection, no allocated cells have been mutated,
  except in the pointer fields, in which case we apply an address
  translation function $f$.

  \begin{align*}
    \forall c \in h,\ \forall w \in c,\ \alloc{c} & \implies
    \mathrm{type}(w) = \mathrm{pointer} \iff h[w] = f(h'[w])\\
    &\quad\land \mathrm{type}(w) \neq \mathrm{pointer} \iff h[w] = h'[w]
  \end{align*}
\end{definition}

Finally, because we are renaming addresses, we shall have to apply $f$
to the root set, in order for the mutator to still work after
collection, \[roots = \left\{f(r)~|~r \in roots'\right\}\]

\section{Copying}
\label{sec:copying-copying}

\todo{Proof obligations for copying}

\section{Address Translation}
\label{sec:copying-address}

\todo{Proof obligations for address translation}

\section{Case Study: A Garbage Collector for Lisp}
\label{sec:copying-example}

\todo{Slightly informal proof for Fenichel/Yochelson 
collector. Informality is ok, because this is only to show the ideas
are in the right direction.}
