\chapter{Introduction}

As systems become more complicated, we have a greater need for
higher-level languages which abstract away from much of the underlying
machine. A key part of such a language is a run-time memory management
system, and in particular the automatic disposal of unneeded allocated
memory: that is, garbage collection.

\section{Motivation}

Garbage collection is an important part of many modern programming
languages, however there is an implicit assumption that it never goes
wrong. It would not be feasible for the programmer to account for all
the ways in which the run-time support system may fail, and doing so
would remove the benefit of using a higher level language in the first
place. But these systems are not perfect, they are written by humans,
and tested by humans, with formal verification being very uncommon.

The aim of this project is to design and prove correct a garbage
collection algorithm, to then implement it in (unverified) Java, and
to benchmark how it performs in a full JVM. Whilst the implementation
will not be verified, this would not be impossible.

\section{Outline}

\todo{Brief description of each chapter in the report}