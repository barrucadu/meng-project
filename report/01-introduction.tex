\chapter{Introduction}

As systems become more complicated, we have a greater need for
higher-level languages which abstract away from much of the underlying
machine. A key part of such a language is a run-time memory management
system, and in particular the automatic disposal of unneeded allocated
memory: that is, garbage collection.

\section{Motivation}

Garbage collection is an important part of many modern programming
languages, however there is an implicit assumption that it never goes
wrong. It would not be feasible for the programmer to account for all
the ways in which the run-time support system may fail, and doing so
would remove the benefit of using a higher level language in the first
place. But these systems are not perfect, they are written by humans,
and tested by humans, with formal verification being very uncommon.

The aim of this project is to design and prove correct a garbage
collection algorithm, to then implement it in (unverified) Java, and
to benchmark how it performs in a full JVM. Whilst the implementation
will not be verified, this would not be impossible.

\section{Outline}

In Chapter 2, the historical context and recent developments in the
field of the project are presented, and is built upon in Chapter 3
where the full aims and motivations of the project are
presented.

Chapter 4 proceeds to formalise partial correctness for
garbage collection, and justifies the formalism chosen by proving
correct two different collectors.

Chapter 5 develops a garbage collection algorithm, proves it
correct using the formalism, and then shows an implementation of the
algorithm in a language runtime. The relation between algorithm and
code is then discussed.

Finally, Chapters 6 and 7 summarise the contributions of the project
and what remains to be done, as well as discussing limitations of the
work.
