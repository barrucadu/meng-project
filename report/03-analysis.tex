\chapter{Problem Analysis}

\section{Aims Revisited}

Reviewing the literature reveals that much of the recent work in
garbage collection development has been focused on concurrency, and
obtaining real-time guarantees. The topic of verified garbage
collection has received comparatively little interest, and is perhaps
viewed as something which would be nice, but not as important as
making garbage collectors as fast as possible. This is a shame, as
garbage collection bugs can be a very difficult problem to solve
through testing alone.

There have been a few verified garbage collectors, however these tend
to be proofs for specific extant algorithms, or extraction using very
strict definitions of garbage collection. The former cannot,
unfortunately, typically generalise to other collectors, and the
latter makes generalising to types of collectors other than what the
author had in mind very difficult.

This project explores the route of using generic and flexible
formalisms, not tied to any particular class of collector, and which
can be used for the proof or extraction methods. The aim, through
algorithm/formalism co-design, is to produce:

\begin{itemize}
  \item a family of increasingly strict formalisms for garbage
    collection partial correctness, in terms of the amount of work
    done by the collector in each collection;

  \item an illustration of the utility of the formalisms by proofs of
    correctness of two simple mark-sweep collectors: one specialised
    for immutable languages, and one more general;

  \item design, proof, and implementation of a copying collector.
\end{itemize}

\section{Development Methodology}

\todo{How I'm going to develop the algorithm and prove it}

\section{Evaluation Methodology}

\todo{How I'm going to evaluate the work}